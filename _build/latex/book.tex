%% Generated by Sphinx.
\def\sphinxdocclass{jupyterBook}
\documentclass[letterpaper,10pt,english]{jupyterBook}
\ifdefined\pdfpxdimen
   \let\sphinxpxdimen\pdfpxdimen\else\newdimen\sphinxpxdimen
\fi \sphinxpxdimen=.75bp\relax
%% turn off hyperref patch of \index as sphinx.xdy xindy module takes care of
%% suitable \hyperpage mark-up, working around hyperref-xindy incompatibility
\PassOptionsToPackage{hyperindex=false}{hyperref}
%% memoir class requires extra handling
\makeatletter\@ifclassloaded{memoir}
{\ifdefined\memhyperindexfalse\memhyperindexfalse\fi}{}\makeatother

\PassOptionsToPackage{warn}{textcomp}

\catcode`^^^^00a0\active\protected\def^^^^00a0{\leavevmode\nobreak\ }
\usepackage{cmap}
\usepackage{fontspec}
\defaultfontfeatures[\rmfamily,\sffamily,\ttfamily]{}
\usepackage{amsmath,amssymb,amstext}
\usepackage{polyglossia}
\setmainlanguage{english}



\setmainfont{FreeSerif}[
  Extension      = .otf,
  UprightFont    = *,
  ItalicFont     = *Italic,
  BoldFont       = *Bold,
  BoldItalicFont = *BoldItalic
]
\setsansfont{FreeSans}[
  Extension      = .otf,
  UprightFont    = *,
  ItalicFont     = *Oblique,
  BoldFont       = *Bold,
  BoldItalicFont = *BoldOblique,
]
\setmonofont{FreeMono}[
  Extension      = .otf,
  UprightFont    = *,
  ItalicFont     = *Oblique,
  BoldFont       = *Bold,
  BoldItalicFont = *BoldOblique,
]


\usepackage[Bjarne]{fncychap}
\usepackage[,numfigreset=0,mathnumfig]{sphinx}

\fvset{fontsize=\small}
\usepackage{geometry}


% Include hyperref last.
\usepackage{hyperref}
% Fix anchor placement for figures with captions.
\usepackage{hypcap}% it must be loaded after hyperref.
% Set up styles of URL: it should be placed after hyperref.
\urlstyle{same}

\addto\captionsenglish{\renewcommand{\contentsname}{Introdução ao Python}}

\usepackage{sphinxmessages}



        % Start of preamble defined in sphinx-jupyterbook-latex %
         \usepackage[Latin,Greek]{ucharclasses}
        \usepackage{unicode-math}
        % fixing title of the toc
        \addto\captionsenglish{\renewcommand{\contentsname}{Contents}}
        \hypersetup{
            pdfencoding=auto,
            psdextra
        }
        % End of preamble defined in sphinx-jupyterbook-latex %
        

\title{Curso de SymPy}
\date{Jul 15, 2021}
\release{}
\author{Eduardo Adame Salles}
\newcommand{\sphinxlogo}{\vbox{}}
\renewcommand{\releasename}{}
\makeindex
\begin{document}

\pagestyle{empty}
\sphinxmaketitle
\pagestyle{plain}
\sphinxtableofcontents
\pagestyle{normal}
\phantomsection\label{\detokenize{intro::doc}}

\begin{itemize}
\item {} 
\sphinxAtStartPar
Introdução ao Python

\begin{itemize}
\item {} 
\sphinxAtStartPar
{\hyperref[\detokenize{chapters/1::doc}]{\sphinxcrossref{Conhecendo e Preparando o Ambiente}}}

\item {} 
\sphinxAtStartPar
{\hyperref[\detokenize{chapters/2::doc}]{\sphinxcrossref{Básicos da Programação em Python}}}

\end{itemize}
\end{itemize}
\begin{itemize}
\item {} 
\sphinxAtStartPar
Utilizando o SymPy

\begin{itemize}
\item {} 
\sphinxAtStartPar
{\hyperref[\detokenize{chapters/3::doc}]{\sphinxcrossref{Primeiros Passos com o Sympy}}}

\item {} 
\sphinxAtStartPar
{\hyperref[\detokenize{chapters/4::doc}]{\sphinxcrossref{Aplicações em Cálculo Diferencial e Integral}}}

\item {} 
\sphinxAtStartPar
{\hyperref[\detokenize{chapters/5::doc}]{\sphinxcrossref{Criando Gráficos}}}

\item {} 
\sphinxAtStartPar
{\hyperref[\detokenize{chapters/6::doc}]{\sphinxcrossref{Extras}}}

\end{itemize}
\end{itemize}

\begin{DUlineblock}{0em}
\item[] \sphinxstylestrong{\Large Sobre}
\end{DUlineblock}

\sphinxAtStartPar
Esse curso busca ser uma porta de entrada do aluno de Engenharia à programação, mais especificamente, programação com Python.

\sphinxAtStartPar
Contudo, a grande diferença que estou buscando trazer é indroduzi\sphinxhyphen{}lo direto à aplicação. Ou seja, com uma biblioteca que pode ser útil para ele em seus estudos.

\sphinxAtStartPar
A biblioteca Sympy é muito poderosa. Não só no cálculo simbólico, mas também no numérico. Ou seja, ela é capaz de calcular Integrais (indefinidas, definidas, impróprias, etc.), Derivadas, Limites e tudo que há de bom, com extrema facilidade.

\sphinxAtStartPar
Além disso, seu uso com o plot de gráficos e com geometria é incrível.

\sphinxAtStartPar
No final, esse módulo acaba sendo um substituto programático ao Matlab, Wolfram Mathematica, Maple, e outros softwares pagos. Além de ser um bom auxiliar aos estudos com Geogebra.


\bigskip\hrule\bigskip


\sphinxAtStartPar
A ideia principal é utilizar esse repositório como curso, com sua organização pensada para que se aprenda somente com ele. Mas, assim que eu entender que o curso está “100\%”, pretendo distribuí\sphinxhyphen{}lo em versões pdf/html. E ir atualizando\sphinxhyphen{}o à medida que o repositório vá se atualizando.

\begin{DUlineblock}{0em}
\item[] \sphinxstylestrong{\large Licença}
\end{DUlineblock}

\sphinxAtStartPar
Embora a intenção seja distribuir o curso para os alunos do CEFET, ele estará disponível no Github para qualquer um ler e contribuir, alterar e/ou redistribuir. Para isso, a licença GNU GPLv3 foi escolhida para que possam ter liberdade nesses assuntos, e proteger o código de distribuições não \sphinxstyleemphasis{open\sphinxhyphen{}source}.


\part{Introdução ao Python}


\chapter{Conhecendo e Preparando o Ambiente}
\label{\detokenize{chapters/1:conhecendo-e-preparando-o-ambiente}}\label{\detokenize{chapters/1::doc}}

\section{Introdução}
\label{\detokenize{chapters/1:introducao}}
\sphinxAtStartPar
Esse curso busca capacitar seus estudantes no tocante à biblioteca Sympy, do Python. Portanto, há a necessidade do conhecimento do próprio Python base. Com isso, o curso foi dividido em uma ordem lógica de aprendizagem para que o aluno consiga aprender continuamente, sem grandes “degraus” entre os assuntos. O capítulo 1 é uma introdução ao Python base, o que é suficiente para esse curso.

\sphinxAtStartPar
Neste capítulo, temos como objetivo fazer a instalação e configuração do nosso ambiente de desenvolvimento. Se você já é desenvolvedor Python, não acredito que seja necessária sua mudança ao ambiente que será indicado a esse curso. Mas, caso seja iniciante, recomendo que siga as instruções.


\section{Escolhendo um Ambiente}
\label{\detokenize{chapters/1:escolhendo-um-ambiente}}
\sphinxAtStartPar
Há duas opções para seguir esse curso:
\begin{enumerate}
\sphinxsetlistlabels{\arabic}{enumi}{enumii}{}{.}%
\item {} 
\sphinxAtStartPar
Na forma de Notebooks

\item {} 
\sphinxAtStartPar
Na forma de Scripts

\end{enumerate}

\sphinxAtStartPar
No curso, serão utilizado notebooks. Mais especificamente, Jupyter Notebooks, através do ambiente JupyterHub. Mas, não se preocupe, ambas as opções serão eficientes. Embora a utilização do Sympy seja regularmente associada ao desenvolvimento em Jupyter.

\sphinxAtStartPar
Na primeira opção, você só terá que instalar um pacote que virá com tudo incluso (IDE, Pacotes, etc.), e sua programação será mais parecida como um caderno. Onde você pode escrever, colocar imagens, matemática, e separar o código em blocos. Contudo, ele é mais pesado e menos flexível.

\sphinxAtStartPar
Na segunda opção, você terá um ambiente mais semelhante à programação tradicional. Linha após linha e comentários. Você terá que instalar o Python puro, versionar/atualizar seus módulos manualmente (o que é mais trabalhoso, mas dá mais autonomia) e utilizar um Editor de Código/IDE.

\sphinxAtStartPar
Caso não tenha entendido o que fora explicado acima, acredito que, nesse momento inicial, a primeira opção é a melhor para você. Contudo, ela não te ajudará a migrar para outra linguagem de programação no futuro, caso queira ou seja necessário. (Isso não é um problema, ao meu ver).


\section{Instalando o Ambiente (1ª Opção)}
\label{\detokenize{chapters/1:instalando-o-ambiente-1a-opcao}}

\subsection{Windows}
\label{\detokenize{chapters/1:windows}}
\sphinxAtStartPar
No Windows, a instalação é bem simples, utilizaremos o Anaconda. Para isso, baixe o \sphinxhref{https://www.anaconda.com/download/\#windows}{instalador do Anaconda}.

\sphinxAtStartPar
Prossiga a instalação normalmente. Contudo, preste atenção às duas caixas de marcação (checkboxes) que perguntam sobre adicionar o Anaconda ao PATH e torná\sphinxhyphen{}lo padrão. Ambas as opções devem estar marcadas.

\sphinxAtStartPar
Se ele oferecer a instalação de algum outro IDE (como o PyCharm), você pode negar.


\subsection{Linux}
\label{\detokenize{chapters/1:linux}}
\sphinxAtStartPar
A instalação no Linux é um pouco mais complicada. Mas, se você utiliza Linux, possivelmente consegue instalar com certa facilidade.

\sphinxAtStartPar
Verifique as dependências da sua distribuição nesse link: \sphinxurl{https://docs.anaconda.com/anaconda/install/linux/}

\sphinxAtStartPar
E depois siga a instalação com o \sphinxhref{https://www.anaconda.com/download/\#linux}{instalador para Linux}. Ele é um instalador em Bash, então recomendo que utilize os comandos prescritos no site do Anaconda.


\section{Instalando o Ambiente (2ª Opção)}
\label{\detokenize{chapters/1:instalando-o-ambiente-2a-opcao}}
\sphinxAtStartPar
Como dito na seção anterior, para essa opção temos que instalar 2 programas, além dos pacotes separadamente.


\subsection{Windows}
\label{\detokenize{chapters/1:id1}}
\sphinxAtStartPar
No Windows, você terá que baixar e instalar o Python que está disponível nesse link: \sphinxurl{https://www.python.org/downloads/}

\sphinxAtStartPar
A instalação é bem simples, você só deve se certificar que ele será adicionado ao PATH e que o \sphinxcode{\sphinxupquote{pip}} será instalado. (São duas caixinhas de marcar que devem aparecer)

\sphinxAtStartPar
Para testar se a instalação deu certo, abra um prompt de comando (basta inserir ou cmd ou powershell no menu de pesquisa) e digite:

\begin{sphinxVerbatim}[commandchars=\\\{\}]
py \PYGZhy{}\PYGZhy{}version
\end{sphinxVerbatim}

\sphinxAtStartPar
Deve aparecer a versão do Python que você instalou. Caso isso não ocorra, repita o procedimento e veja se fez tudo corretamente.

\sphinxAtStartPar
Também confirme se o pip foi instalado corretamente:

\begin{sphinxVerbatim}[commandchars=\\\{\}]
pip \PYGZhy{}\PYGZhy{}version
\end{sphinxVerbatim}

\sphinxAtStartPar
Após isso, você deve escolher um editor. A minha recomendação é o \sphinxhref{https://code.visualstudio.com/}{Visual Studio Code}.

\sphinxAtStartPar
Mas você pode utilizar outros como o \sphinxhref{https://atom.io/}{Atom} ou \sphinxhref{https://www.jetbrains.com/pt-br/pycharm/download/}{PyCharm}.

\sphinxAtStartPar
A instalação de todos é bem simples. Ao abri\sphinxhyphen{}los, certifique\sphinxhyphen{}se se é necessário instalar uma extensão para suporte ao Python. Caso for, instale\sphinxhyphen{}a.


\subsection{Linux}
\label{\detokenize{chapters/1:id2}}
\sphinxAtStartPar
No Linux, a imensa maioria das distribuições já vêm com o Python instalado e disponível. A única diferença é que algumas adotam o nome \sphinxcode{\sphinxupquote{python3}} e outras somente \sphinxcode{\sphinxupquote{python}}.

\sphinxAtStartPar
Para isso, teste no terminal:

\begin{sphinxVerbatim}[commandchars=\\\{\}]
python \PYGZhy{}\PYGZhy{}version
python3 \PYGZhy{}\PYGZhy{}version
\end{sphinxVerbatim}

\sphinxAtStartPar
E comece a usar o que tem a versão mais atual.

\sphinxAtStartPar
Também certifique\sphinxhyphen{}se se você tem o \sphinxcode{\sphinxupquote{pip}} instalado

\begin{sphinxVerbatim}[commandchars=\\\{\}]
pip \PYGZhy{}\PYGZhy{}version
pip3 \PYGZhy{}\PYGZhy{}version
\end{sphinxVerbatim}

\sphinxAtStartPar
Caso não tenha, veja como instalá\sphinxhyphen{}lo em sua distribuição.

\sphinxAtStartPar
Assim como no Windows, você deve instalar um editor. As minhas recomendações são as mesmas para o Windows. Ou seja, primeiramente o \sphinxhref{https://code.visualstudio.com/}{Visual Studio Code}. E depois ou o \sphinxhref{https://atom.io/}{Atom} ou o \sphinxhref{https://www.jetbrains.com/pt-br/pycharm/download/}{PyCharm}.

\sphinxAtStartPar
Todos têm versão para Linux e a instalação deve ser até mais simples que para windows. No caso do Visual Studio Code, recomendo que utilize a versão flatpak ou snap.


\section{Como acompanhar as explicações do curso}
\label{\detokenize{chapters/1:como-acompanhar-as-explicacoes-do-curso}}
\sphinxAtStartPar
O nosso curso será baseado em Jupyter Notebooks, uma forma de programar em blocos. Esses blocos podem ser tanto de código como de texto.

\sphinxAtStartPar
No caso, a primeira opção de ambiente lhe entregará um ambiente muito parecido com o que eu utilizei para escrever os meus Notebooks. Para a segunda, acredito que será mais difícil. Principalmente, pela ausência de alguns pacotes e softwares que vêm com o Anaconda. Mas, decidi incluí\sphinxhyphen{}la, pois há pessoas que precisam somente de uma consulta rápida, e possivelmente já têm seu ambiente de desenvolvimento, sendo em Anaconda ou não.

\sphinxAtStartPar
Então, para a primeira opção, você deverá abrir o “Anaconda Navigator”. Esse software servirá como gerenciador de pacotes e dos serviços.

\sphinxAtStartPar
Como dito anteriormente, utilizaremos Jupyter Notebooks. Então basta clicar em “Launch” no card do Jupyter Notebook.

\sphinxAtStartPar
Ele abrirá um navegador de arquivos no navegador, semelhante a um site. Basta escolher uma pasta onde gostaria de criar seus notebooks, e criá\sphinxhyphen{}los a partir do botão “New” e então escolha o \sphinxstyleemphasis{Python 3}.

\sphinxAtStartPar
O curso não tem como foco específico ensinar o uso desse ambiente, uma vez que ele é bem simples. Basicamente, você deverá criar blocos (que costumamos chamar de \sphinxstyleemphasis{chunks}) de forma a organizar o entendimento e a saída do seu código. Ao escolher a opção “Code” para seu bloco, você deverá escrever códigos Python nele. No caso de “Markdown”, você deverá escrever texto de acordo com a sintaxe da linguagem de marcação Markdown.

\sphinxAtStartPar
Essa sintaxe é bem simples, dê uma olhada nessa “tabela de cola”: \sphinxurl{https://github.com/luong-komorebi/Markdown-Tutorial/blob/master/README\_pt-BR.md}

\sphinxAtStartPar
Para qualquer tipo de bloco, basta segurar a tecla Ctrl e apertar Enter para compilá\sphinxhyphen{}lo.


\section{Exercícios}
\label{\detokenize{chapters/1:exercicios}}\begin{enumerate}
\sphinxsetlistlabels{\arabic}{enumi}{enumii}{}{.}%
\item {} 
\sphinxAtStartPar
Crie um chunk de markdown e escreva um título com o texto “Olá mundo em Python”.

\item {} 
\sphinxAtStartPar
Abaixo desse chunk, crie outro com o código \sphinxcode{\sphinxupquote{print('Olá mundo')}}.

\end{enumerate}


\section{Próximos passos}
\label{\detokenize{chapters/1:proximos-passos}}
\sphinxAtStartPar
Embora esse capítulo não tenha tido um bloco sequer de código, acredito que agora estão preparados para aprender os básicos de Python


\chapter{Básicos da Programação em Python}
\label{\detokenize{chapters/2:basicos-da-programacao-em-python}}\label{\detokenize{chapters/2::doc}}

\section{Sobre o Python}
\label{\detokenize{chapters/2:sobre-o-python}}
\sphinxAtStartPar
Não é por acaso que a linguagem de programação Python tem se tornado cada vez mais popular. Além do seu uso na computação, que ganhou um novo \sphinxstyleemphasis{hype} por conta da Inteligência Artificial, a linguagem é uma ótima forma de introduzir uma pessoa à programação, de modo geral.

\sphinxAtStartPar
A linguagem apresenta diversas facilidades em sua sintaxe, que não somente contibuem para a facilidade da escrita, mas também contribuem para o entendimento e documentação do código.

\sphinxAtStartPar
Novamente, para não fugir do escopo do curso, não acredito que seja necessário listar as especificações técnicas da linguagem. Acredito que há somente alguns conceitos simples que são necessários para o uso da linguagem.
\begin{itemize}
\item {} 
\sphinxAtStartPar
O Python é uma linguagem interpretada.

\end{itemize}

\sphinxAtStartPar
Ou seja, o software gerado pelo código é o próprio código interpretado durante a execução
\begin{itemize}
\item {} 
\sphinxAtStartPar
O Python interpretará linha após linha, do início ao fim do script (chunk, para nós).

\end{itemize}

\sphinxAtStartPar
Para novatos em programação, isso deve ser fácil de aceitar. Mas pessoas com certa experiência podem questionar essa afirmação. Nesse caso, estou me referindo a forma que vamos utilizar a linguagem no curso e até onde iremos com as estruturas de programação da linguagem.
\begin{itemize}
\item {} 
\sphinxAtStartPar
O Python diferencia letras maíusculas de minúsculas.

\end{itemize}

\sphinxAtStartPar
Isso é chamado de “sensitive case”. Tome cuidado ao reproduzir os exemplos e ao programar, de modo geral.
\begin{itemize}
\item {} 
\sphinxAtStartPar
O Python reconhece espaços vazios.

\end{itemize}

\sphinxAtStartPar
Isso é utilizado para definir o escopo dessa linha de código. Não se preocupe caso não tenha entendido, é mais simples do que aparenta ser. Isso basicamente é pra mostrar o “quão dentro” uma linha está de uma estrutura. Abordaremos isso mais tarde.
\begin{itemize}
\item {} 
\sphinxAtStartPar
Texto precedido por \sphinxcode{\sphinxupquote{\#}} não são interpretadas

\end{itemize}

\sphinxAtStartPar
Tudo que vier após um \sphinxcode{\sphinxupquote{\#}} em uma linha é chamado de comentário. Ele não é compilado e utilizaremos para comentar o código.


\section{Definindo variáveis}
\label{\detokenize{chapters/2:definindo-variaveis}}
\sphinxAtStartPar
Uma variável é uma forma de armazenar certo valor e reutilizá\sphinxhyphen{}lo. Se quiser fazer uma comparação com a Matemática, apenas enxergue ele como um \(x_0\) não um \(x\) somente. Ou seja, ele vai ter um valor específico que pode mudar. Não necessáriamente ele seria “todos os valores ao mesmo tempo” como uma variável na matemática.

\sphinxAtStartPar
Por exemplo, vamos criar a variável \sphinxcode{\sphinxupquote{name}} que vai armazenar meu nome.

\begin{sphinxVerbatim}[commandchars=\\\{\}]
\PYG{n}{name} \PYG{o}{=} \PYG{l+s+s2}{\PYGZdq{}}\PYG{l+s+s2}{Eduardo}\PYG{l+s+s2}{\PYGZdq{}}
\end{sphinxVerbatim}

\sphinxAtStartPar
Agora, eu posso utilizar name em qualquer lugar do meu código. No caso, se eu mudar o valor de name, todo lugar onde name está terá seu valor trocado.

\sphinxAtStartPar
Por exemplo:

\begin{sphinxVerbatim}[commandchars=\\\{\}]
\PYG{n}{name} \PYG{o}{=} \PYG{l+s+s2}{\PYGZdq{}}\PYG{l+s+s2}{Eduardo}\PYG{l+s+s2}{\PYGZdq{}}
\PYG{n}{name}
\end{sphinxVerbatim}

\begin{sphinxVerbatim}[commandchars=\\\{\}]
\PYGZsq{}Eduardo\PYGZsq{}
\end{sphinxVerbatim}

\begin{sphinxVerbatim}[commandchars=\\\{\}]
\PYG{n}{name} \PYG{o}{=} \PYG{l+s+s2}{\PYGZdq{}}\PYG{l+s+s2}{Adame}\PYG{l+s+s2}{\PYGZdq{}}
\PYG{n}{name}
\end{sphinxVerbatim}

\begin{sphinxVerbatim}[commandchars=\\\{\}]
\PYGZsq{}Adame\PYGZsq{}
\end{sphinxVerbatim}

\sphinxAtStartPar
Para ficar mais claro, podemos ver algo como:

\begin{sphinxVerbatim}[commandchars=\\\{\}]
\PYG{n}{name} \PYG{o}{=} \PYG{l+s+s2}{\PYGZdq{}}\PYG{l+s+s2}{Eduardo}\PYG{l+s+s2}{\PYGZdq{}}
\PYG{n}{name} \PYG{o}{=} \PYG{l+s+s2}{\PYGZdq{}}\PYG{l+s+s2}{Adame}\PYG{l+s+s2}{\PYGZdq{}}
\PYG{n}{name}
\end{sphinxVerbatim}

\begin{sphinxVerbatim}[commandchars=\\\{\}]
\PYGZsq{}Adame\PYGZsq{}
\end{sphinxVerbatim}

\sphinxAtStartPar
Como pôde ver no \sphinxstyleemphasis{chunk} acima, embora eu tenha atribuído o valor \sphinxcode{\sphinxupquote{"Eduardo"}} ao \sphinxcode{\sphinxupquote{name}} na primeira linha, o valor dele foi sobrescrito pela linha seguinte.


\section{Tipos de Variáveis}
\label{\detokenize{chapters/2:tipos-de-variaveis}}
\sphinxAtStartPar
Acredito que deve ter notado que colocamos o valor entre \sphinxcode{\sphinxupquote{" "}}. Por quê?

\sphinxAtStartPar
Isso tem a ver com o tipo do valor. No Python, os tipos são dinamicamente interpretados através de sua sintaxe. E, nesse caso, nós inserimos uma \sphinxcode{\sphinxupquote{string}}, que é utilizada para armazenar texto.

\sphinxAtStartPar
Vamos falar sobre os principais tipos de variáveis que já vêm com o Python e como defini\sphinxhyphen{}las.


\subsection{Inteiros}
\label{\detokenize{chapters/2:inteiros}}
\sphinxAtStartPar
Os inteiros são simplesmente definidos ao igualar uma variável a um número que pertença ao conjunto dos números inteiros, seja positivo ou negativo.

\begin{sphinxVerbatim}[commandchars=\\\{\}]
\PYG{n}{my\PYGZus{}integer} \PYG{o}{=} \PYG{l+m+mi}{2}
\PYG{n}{my\PYGZus{}integer}
\end{sphinxVerbatim}

\begin{sphinxVerbatim}[commandchars=\\\{\}]
2
\end{sphinxVerbatim}


\subsection{Floats}
\label{\detokenize{chapters/2:floats}}
\sphinxAtStartPar
Os floats, diferentemente dos inteiros, podem ser qualquer número real. Utilizamos ponto (\sphinxcode{\sphinxupquote{.}}) para separar as casas decimais.

\begin{sphinxVerbatim}[commandchars=\\\{\}]
\PYG{n}{my\PYGZus{}float} \PYG{o}{=} \PYG{l+m+mf}{2.5}
\PYG{n}{my\PYGZus{}float}
\end{sphinxVerbatim}

\begin{sphinxVerbatim}[commandchars=\\\{\}]
2.5
\end{sphinxVerbatim}


\subsection{Strings}
\label{\detokenize{chapters/2:strings}}
\sphinxAtStartPar
Como dito anteriormente, strings são utilizados para textos. Basta envolver seu texto em aspas (\sphinxcode{\sphinxupquote{" "}}) ou (\sphinxcode{\sphinxupquote{' '}}). Para o Python, tanto faz se elas são simples ou duplas.

\begin{sphinxVerbatim}[commandchars=\\\{\}]
\PYG{n}{my\PYGZus{}string} \PYG{o}{=} \PYG{l+s+s2}{\PYGZdq{}}\PYG{l+s+s2}{Textos são bem úteis, não acha?}\PYG{l+s+s2}{\PYGZdq{}}
\PYG{n}{my\PYGZus{}string}
\end{sphinxVerbatim}

\begin{sphinxVerbatim}[commandchars=\\\{\}]
\PYGZsq{}Textos são bem úteis, não acha?\PYGZsq{}
\end{sphinxVerbatim}


\subsection{Booleanos}
\label{\detokenize{chapters/2:booleanos}}
\sphinxAtStartPar
Esse tipo de valor é utilizado para condições. Eles podem ser “Falso” (\sphinxcode{\sphinxupquote{False}}) ou “Verdadeiro” (\sphinxcode{\sphinxupquote{True}}). Preste atenção no sensitive case.

\begin{sphinxVerbatim}[commandchars=\\\{\}]
\PYG{n}{do\PYGZus{}i\PYGZus{}like\PYGZus{}math} \PYG{o}{=} \PYG{k+kc}{True}
\PYG{n}{do\PYGZus{}i\PYGZus{}like\PYGZus{}rlang} \PYG{o}{=} \PYG{k+kc}{False}
\PYG{n}{do\PYGZus{}i\PYGZus{}like\PYGZus{}math}\PYG{p}{,} \PYG{n}{do\PYGZus{}i\PYGZus{}like\PYGZus{}rlang}
\end{sphinxVerbatim}

\begin{sphinxVerbatim}[commandchars=\\\{\}]
(True, False)
\end{sphinxVerbatim}


\subsection{Listas}
\label{\detokenize{chapters/2:listas}}
\sphinxAtStartPar
Nós utilizamos listas para agrupar outros valores. Uma lista pode armazenar qualquer tipo e em qualquer quantidade. Inclusive podemos fazer listas de listas.

\begin{sphinxVerbatim}[commandchars=\\\{\}]
\PYG{n}{names} \PYG{o}{=} \PYG{p}{[}\PYG{l+s+s2}{\PYGZdq{}}\PYG{l+s+s2}{Eduardo}\PYG{l+s+s2}{\PYGZdq{}}\PYG{p}{,} \PYG{l+s+s2}{\PYGZdq{}}\PYG{l+s+s2}{Marcos}\PYG{l+s+s2}{\PYGZdq{}}\PYG{p}{,} \PYG{l+s+s2}{\PYGZdq{}}\PYG{l+s+s2}{Ruan}\PYG{l+s+s2}{\PYGZdq{}}\PYG{p}{,} \PYG{l+s+s2}{\PYGZdq{}}\PYG{l+s+s2}{João}\PYG{l+s+s2}{\PYGZdq{}}\PYG{p}{]}
\PYG{n}{names}
\end{sphinxVerbatim}

\begin{sphinxVerbatim}[commandchars=\\\{\}]
[\PYGZsq{}Eduardo\PYGZsq{}, \PYGZsq{}Marcos\PYGZsq{}, \PYGZsq{}Ruan\PYGZsq{}, \PYGZsq{}João\PYGZsq{}]
\end{sphinxVerbatim}

\begin{sphinxVerbatim}[commandchars=\\\{\}]
\PYG{n}{my\PYGZus{}id} \PYG{o}{=} \PYG{p}{[}\PYG{l+s+s2}{\PYGZdq{}}\PYG{l+s+s2}{Eduardo}\PYG{l+s+s2}{\PYGZdq{}}\PYG{p}{,} \PYG{l+m+mi}{18}\PYG{p}{,} \PYG{l+m+mf}{1.78}\PYG{p}{]}
\PYG{n}{my\PYGZus{}id}
\end{sphinxVerbatim}

\begin{sphinxVerbatim}[commandchars=\\\{\}]
[\PYGZsq{}Eduardo\PYGZsq{}, 18, 1.78]
\end{sphinxVerbatim}

\sphinxAtStartPar
Para acessar um valor em específico na lista utilizamos \sphinxcode{\sphinxupquote{{[}{]}}} com o índice desse valor. A lista começa a contagem em 0. Veremos algumas outras coisas necessárias para o uso de listas posteriormente.

\begin{sphinxVerbatim}[commandchars=\\\{\}]
\PYG{n}{names}\PYG{p}{[}\PYG{l+m+mi}{1}\PYG{p}{]} \PYG{c+c1}{\PYGZsh{} \PYGZdq{}Eduardo\PYGZdq{} é o 0}
\end{sphinxVerbatim}

\begin{sphinxVerbatim}[commandchars=\\\{\}]
\PYGZsq{}Marcos\PYGZsq{}
\end{sphinxVerbatim}

\begin{sphinxVerbatim}[commandchars=\\\{\}]
\PYG{n}{my\PYGZus{}id}\PYG{p}{[}\PYG{l+m+mi}{2}\PYG{p}{]}
\end{sphinxVerbatim}

\begin{sphinxVerbatim}[commandchars=\\\{\}]
1.78
\end{sphinxVerbatim}


\section{Operações}
\label{\detokenize{chapters/2:operacoes}}
\sphinxAtStartPar
Os operadores e as operações para cada tipo são diferentes. Principalmente quando ocorrem entre dois tipos diferentes. Os principais para cada um deles são:


\subsection{Aritmética}
\label{\detokenize{chapters/2:aritmetica}}
\sphinxAtStartPar
Com \sphinxcode{\sphinxupquote{int}} e \sphinxcode{\sphinxupquote{float}} é possível utilizar os operadores comuns da Matemática. Quando ocorre entre eles, o resultado será um \sphinxcode{\sphinxupquote{float}}.

\sphinxAtStartPar
Os principais operadores são:
\begin{itemize}
\item {} 
\sphinxAtStartPar
\sphinxcode{\sphinxupquote{+}}/\sphinxcode{\sphinxupquote{\sphinxhyphen{}}} para adição/sutração.

\item {} 
\sphinxAtStartPar
\sphinxcode{\sphinxupquote{*}}/\sphinxcode{\sphinxupquote{/}} para multiplicação/divisão.

\item {} 
\sphinxAtStartPar
\sphinxcode{\sphinxupquote{**}} para potenciação.

\item {} 
\sphinxAtStartPar
\sphinxcode{\sphinxupquote{\%}} para módulo (resto).

\end{itemize}

\sphinxAtStartPar
Para alguns exemplos, como esse, não criarei variáveis, mas é possível criar para qualquer operação.

\begin{sphinxVerbatim}[commandchars=\\\{\}]
\PYG{l+m+mi}{5} \PYG{o}{+} \PYG{l+m+mi}{2} \PYG{o}{+} \PYG{l+m+mi}{4} \PYG{o}{*} \PYG{l+m+mi}{10} 
\end{sphinxVerbatim}

\begin{sphinxVerbatim}[commandchars=\\\{\}]
47
\end{sphinxVerbatim}

\begin{sphinxVerbatim}[commandchars=\\\{\}]
\PYG{l+m+mi}{2} \PYG{o}{*}\PYG{o}{*} \PYG{p}{(}\PYG{l+m+mi}{1}\PYG{o}{/}\PYG{l+m+mi}{2}\PYG{p}{)} \PYG{c+c1}{\PYGZsh{} Raíz de 2}
\end{sphinxVerbatim}

\begin{sphinxVerbatim}[commandchars=\\\{\}]
1.4142135623730951
\end{sphinxVerbatim}

\sphinxAtStartPar
Você também pode utilizar esses operadores em cima de uma variável. Por exemplo, você tem uma variável e quer acrescer 2 nela.

\begin{sphinxVerbatim}[commandchars=\\\{\}]
\PYG{n}{my\PYGZus{}num} \PYG{o}{=} \PYG{l+m+mi}{14}
\PYG{n}{my\PYGZus{}num} \PYG{o}{=} \PYG{n}{my\PYGZus{}num} \PYG{o}{+} \PYG{l+m+mi}{2}
\PYG{n}{my\PYGZus{}num}
\end{sphinxVerbatim}

\begin{sphinxVerbatim}[commandchars=\\\{\}]
16
\end{sphinxVerbatim}

\sphinxAtStartPar
O exemplo acima é bem convincente e, certamente, bem útil. Mas há uma forma reduzida desse tipo de expressão.

\begin{sphinxVerbatim}[commandchars=\\\{\}]
\PYG{n}{my\PYGZus{}num} \PYG{o}{=} \PYG{l+m+mi}{14}
\PYG{n}{my\PYGZus{}num} \PYG{o}{+}\PYG{o}{=} \PYG{l+m+mi}{2}
\PYG{n}{my\PYGZus{}num}
\end{sphinxVerbatim}

\begin{sphinxVerbatim}[commandchars=\\\{\}]
16
\end{sphinxVerbatim}

\sphinxAtStartPar
Esse tipo de “operador” pode ser criado ao unir um desses operadores listados acima com o símbolo de \sphinxcode{\sphinxupquote{=}}.

\begin{sphinxVerbatim}[commandchars=\\\{\}]
\PYG{n}{my\PYGZus{}num} \PYG{o}{=} \PYG{l+m+mi}{2}
\PYG{n}{my\PYGZus{}num} \PYG{o}{*}\PYG{o}{=} \PYG{l+m+mi}{10}
\PYG{n}{my\PYGZus{}num}
\end{sphinxVerbatim}

\begin{sphinxVerbatim}[commandchars=\\\{\}]
20
\end{sphinxVerbatim}


\subsection{Booleanas}
\label{\detokenize{chapters/2:booleanas}}
\sphinxAtStartPar
As operações booleanas são expressões que retornam ou \sphinxcode{\sphinxupquote{True}} ou \sphinxcode{\sphinxupquote{False}}, são utilizadas para condições.

\sphinxAtStartPar
No Python, costumamos escrever por extenso a maioria das condições.
\begin{itemize}
\item {} 
\sphinxAtStartPar
\sphinxcode{\sphinxupquote{and}} para operação “e”

\item {} 
\sphinxAtStartPar
\sphinxcode{\sphinxupquote{or}} para operação “ou”

\item {} 
\sphinxAtStartPar
\sphinxcode{\sphinxupquote{not}} para operação de negação (inverte o valor)

\item {} 
\sphinxAtStartPar
\sphinxcode{\sphinxupquote{>}}/\sphinxcode{\sphinxupquote{>=}} maior/maior ou igual

\item {} 
\sphinxAtStartPar
\sphinxcode{\sphinxupquote{<}}/\sphinxcode{\sphinxupquote{<=}} menor/menor ou igual

\item {} 
\sphinxAtStartPar
\sphinxcode{\sphinxupquote{==}} igual

\item {} 
\sphinxAtStartPar
\sphinxcode{\sphinxupquote{in}}confere está em uma lista

\end{itemize}

\sphinxAtStartPar
Exemplos:

\begin{sphinxVerbatim}[commandchars=\\\{\}]
\PYG{l+m+mi}{30} \PYG{o}{\PYGZgt{}} \PYG{l+m+mi}{15}
\end{sphinxVerbatim}

\begin{sphinxVerbatim}[commandchars=\\\{\}]
True
\end{sphinxVerbatim}

\begin{sphinxVerbatim}[commandchars=\\\{\}]
\PYG{l+m+mi}{30} \PYG{o}{\PYGZlt{}} \PYG{l+m+mi}{15}
\end{sphinxVerbatim}

\begin{sphinxVerbatim}[commandchars=\\\{\}]
False
\end{sphinxVerbatim}

\begin{sphinxVerbatim}[commandchars=\\\{\}]
\PYG{l+m+mi}{14} \PYG{o}{==} \PYG{l+m+mi}{14}
\end{sphinxVerbatim}

\begin{sphinxVerbatim}[commandchars=\\\{\}]
True
\end{sphinxVerbatim}

\begin{sphinxVerbatim}[commandchars=\\\{\}]
\PYG{l+m+mi}{30} \PYG{o}{\PYGZlt{}} \PYG{l+m+mi}{15} \PYG{o+ow}{and} \PYG{l+m+mi}{14} \PYG{o}{==} \PYG{l+m+mi}{14}
\end{sphinxVerbatim}

\begin{sphinxVerbatim}[commandchars=\\\{\}]
False
\end{sphinxVerbatim}

\begin{sphinxVerbatim}[commandchars=\\\{\}]
\PYG{l+m+mi}{30} \PYG{o}{\PYGZlt{}}\PYG{o}{=} \PYG{l+m+mi}{15} \PYG{o+ow}{or} \PYG{l+m+mi}{14} \PYG{o}{==} \PYG{l+m+mi}{14}
\end{sphinxVerbatim}

\begin{sphinxVerbatim}[commandchars=\\\{\}]
True
\end{sphinxVerbatim}

\begin{sphinxVerbatim}[commandchars=\\\{\}]
\PYG{l+m+mi}{14}\PYG{o}{==}\PYG{l+m+mi}{14} \PYG{o+ow}{and} \PYG{o+ow}{not} \PYG{l+m+mi}{30} \PYG{o}{\PYGZlt{}} \PYG{l+m+mi}{15}
\end{sphinxVerbatim}

\begin{sphinxVerbatim}[commandchars=\\\{\}]
True
\end{sphinxVerbatim}

\begin{sphinxVerbatim}[commandchars=\\\{\}]
\PYG{n}{lst} \PYG{o}{=} \PYG{p}{[}\PYG{l+m+mi}{1}\PYG{p}{,}\PYG{l+m+mi}{2}\PYG{p}{]}
\PYG{l+m+mi}{2} \PYG{o+ow}{in} \PYG{n}{lst}
\end{sphinxVerbatim}

\begin{sphinxVerbatim}[commandchars=\\\{\}]
True
\end{sphinxVerbatim}

\begin{sphinxVerbatim}[commandchars=\\\{\}]
\PYG{k+kc}{True} \PYG{o+ow}{and} \PYG{k+kc}{False}
\end{sphinxVerbatim}

\begin{sphinxVerbatim}[commandchars=\\\{\}]
False
\end{sphinxVerbatim}

\begin{sphinxVerbatim}[commandchars=\\\{\}]
\PYG{k+kc}{False} \PYG{o+ow}{or} \PYG{o+ow}{not} \PYG{k+kc}{False}
\end{sphinxVerbatim}

\begin{sphinxVerbatim}[commandchars=\\\{\}]
True
\end{sphinxVerbatim}


\subsection{Strings}
\label{\detokenize{chapters/2:id1}}
\sphinxAtStartPar
Podemos formatar strings e fazer algumas operações com elas
\begin{itemize}
\item {} 
\sphinxAtStartPar
\sphinxcode{\sphinxupquote{+}} para concatenação

\item {} 
\sphinxAtStartPar
\sphinxcode{\sphinxupquote{*}} para concatenação repetidas vezes

\end{itemize}

\sphinxAtStartPar
Exemplos:

\begin{sphinxVerbatim}[commandchars=\\\{\}]
\PYG{l+s+s1}{\PYGZsq{}}\PYG{l+s+s1}{Eduardo }\PYG{l+s+s1}{\PYGZsq{}} \PYG{o}{+} \PYG{l+s+s1}{\PYGZsq{}}\PYG{l+s+s1}{Adame}\PYG{l+s+s1}{\PYGZsq{}}
\end{sphinxVerbatim}

\begin{sphinxVerbatim}[commandchars=\\\{\}]
\PYGZsq{}Eduardo Adame\PYGZsq{}
\end{sphinxVerbatim}

\begin{sphinxVerbatim}[commandchars=\\\{\}]
\PYG{l+s+s1}{\PYGZsq{}}\PYG{l+s+s1}{du}\PYG{l+s+s1}{\PYGZsq{}}\PYG{o}{*}\PYG{l+m+mi}{2}
\end{sphinxVerbatim}

\begin{sphinxVerbatim}[commandchars=\\\{\}]
\PYGZsq{}dudu\PYGZsq{}
\end{sphinxVerbatim}


\subsubsection{Strings formatadas}
\label{\detokenize{chapters/2:strings-formatadas}}
\sphinxAtStartPar
As strings formatadas são um modo de inserir variáveis em strings. Adicionamos um \sphinxcode{\sphinxupquote{f}} antes das áspas e colocamos a variável entre chaves \sphinxcode{\sphinxupquote{\{\}}}.

\sphinxAtStartPar
Exemplo:

\begin{sphinxVerbatim}[commandchars=\\\{\}]
\PYG{n}{nome} \PYG{o}{=} \PYG{l+s+s1}{\PYGZsq{}}\PYG{l+s+s1}{Eduardo}\PYG{l+s+s1}{\PYGZsq{}}
\PYG{n}{sobrenome} \PYG{o}{=} \PYG{l+s+s1}{\PYGZsq{}}\PYG{l+s+s1}{Adame}\PYG{l+s+s1}{\PYGZsq{}}
\PYG{n}{idade} \PYG{o}{=} \PYG{l+m+mi}{18}
\PYG{l+s+sa}{f}\PYG{l+s+s1}{\PYGZsq{}}\PYG{l+s+s1}{Meu nome é }\PYG{l+s+si}{\PYGZob{}}\PYG{n}{nome}\PYG{l+s+si}{\PYGZcb{}}\PYG{l+s+s1}{ }\PYG{l+s+si}{\PYGZob{}}\PYG{n}{sobrenome}\PYG{l+s+si}{\PYGZcb{}}\PYG{l+s+s1}{ e tenho }\PYG{l+s+si}{\PYGZob{}}\PYG{n}{idade}\PYG{l+s+si}{\PYGZcb{}}\PYG{l+s+s1}{ anos}\PYG{l+s+s1}{\PYGZsq{}}
\end{sphinxVerbatim}

\begin{sphinxVerbatim}[commandchars=\\\{\}]
\PYGZsq{}Meu nome é Eduardo Adame e tenho 18 anos\PYGZsq{}
\end{sphinxVerbatim}


\section{Conversão de Tipos}
\label{\detokenize{chapters/2:conversao-de-tipos}}
\sphinxAtStartPar
Basta utilizar a função (veremos mais tarde funções em específico) que tem o nome do tipo.
\begin{itemize}
\item {} 
\sphinxAtStartPar
\sphinxcode{\sphinxupquote{int()}} para converter para \sphinxcode{\sphinxupquote{inteiro}}

\item {} 
\sphinxAtStartPar
\sphinxcode{\sphinxupquote{str()}} para converter para \sphinxcode{\sphinxupquote{string}}

\item {} 
\sphinxAtStartPar
\sphinxcode{\sphinxupquote{bool()}} para converter para \sphinxcode{\sphinxupquote{booleano}}

\item {} 
\sphinxAtStartPar
\sphinxcode{\sphinxupquote{float()}} para converter para \sphinxcode{\sphinxupquote{float}}

\end{itemize}

\begin{sphinxVerbatim}[commandchars=\\\{\}]
\PYG{n+nb}{int}\PYG{p}{(}\PYG{l+s+s1}{\PYGZsq{}}\PYG{l+s+s1}{10}\PYG{l+s+s1}{\PYGZsq{}}\PYG{p}{)}
\end{sphinxVerbatim}

\begin{sphinxVerbatim}[commandchars=\\\{\}]
10
\end{sphinxVerbatim}

\begin{sphinxVerbatim}[commandchars=\\\{\}]
\PYG{n+nb}{int}\PYG{p}{(}\PYG{l+m+mf}{20.0}\PYG{p}{)}
\end{sphinxVerbatim}

\begin{sphinxVerbatim}[commandchars=\\\{\}]
20
\end{sphinxVerbatim}

\begin{sphinxVerbatim}[commandchars=\\\{\}]
\PYG{n+nb}{float}\PYG{p}{(}\PYG{l+s+s1}{\PYGZsq{}}\PYG{l+s+s1}{12.23}\PYG{l+s+s1}{\PYGZsq{}}\PYG{p}{)}
\end{sphinxVerbatim}

\begin{sphinxVerbatim}[commandchars=\\\{\}]
12.23
\end{sphinxVerbatim}

\begin{sphinxVerbatim}[commandchars=\\\{\}]
\PYG{n+nb}{str}\PYG{p}{(}\PYG{l+m+mf}{30.2}\PYG{p}{)}
\end{sphinxVerbatim}

\begin{sphinxVerbatim}[commandchars=\\\{\}]
\PYGZsq{}30.2\PYGZsq{}
\end{sphinxVerbatim}

\begin{sphinxVerbatim}[commandchars=\\\{\}]
\PYG{n+nb}{bool}\PYG{p}{(}\PYG{l+m+mi}{0}\PYG{p}{)}
\end{sphinxVerbatim}

\begin{sphinxVerbatim}[commandchars=\\\{\}]
False
\end{sphinxVerbatim}

\begin{sphinxVerbatim}[commandchars=\\\{\}]
\PYG{n+nb}{bool}\PYG{p}{(}\PYG{l+m+mi}{1}\PYG{p}{)} \PYG{c+c1}{\PYGZsh{}ou qualquer outro número diferente de 0}
\end{sphinxVerbatim}

\begin{sphinxVerbatim}[commandchars=\\\{\}]
True
\end{sphinxVerbatim}

\begin{sphinxVerbatim}[commandchars=\\\{\}]
\PYG{n+nb}{bool}\PYG{p}{(}\PYG{l+s+s1}{\PYGZsq{}}\PYG{l+s+s1}{\PYGZsq{}}\PYG{p}{)}
\end{sphinxVerbatim}

\begin{sphinxVerbatim}[commandchars=\\\{\}]
False
\end{sphinxVerbatim}

\begin{sphinxVerbatim}[commandchars=\\\{\}]
\PYG{n+nb}{bool}\PYG{p}{(}\PYG{l+s+s1}{\PYGZsq{}}\PYG{l+s+s1}{a}\PYG{l+s+s1}{\PYGZsq{}}\PYG{p}{)} \PYG{c+c1}{\PYGZsh{}ou qualquer string não vazia}
\end{sphinxVerbatim}

\begin{sphinxVerbatim}[commandchars=\\\{\}]
True
\end{sphinxVerbatim}


\section{Recebendo entrada do usuário}
\label{\detokenize{chapters/2:recebendo-entrada-do-usuario}}
\sphinxAtStartPar
Para isso, utilizamos a função \sphinxcode{\sphinxupquote{input()}}, que recebe uma \sphinxcode{\sphinxupquote{string}} como parâmetro para a mensagem impressa, e retorna a entrada como uma \sphinxcode{\sphinxupquote{string}} (devemos converter, se necessário). Para imprimir uma mensagem utilizamos \sphinxcode{\sphinxupquote{print()}} que imprime qualquer coisa que for passada, costuma\sphinxhyphen{}se utilizar as operações vistas anteriormente.

\sphinxAtStartPar
Exemplo:

\begin{sphinxVerbatim}[commandchars=\\\{\}]
\PYG{n}{number} \PYG{o}{=} \PYG{n+nb}{input}\PYG{p}{(}\PYG{l+s+s1}{\PYGZsq{}}\PYG{l+s+s1}{Insira um número: }\PYG{l+s+s1}{\PYGZsq{}}\PYG{p}{)}
\PYG{n}{number} \PYG{o}{*} \PYG{l+m+mi}{2}
\end{sphinxVerbatim}

\begin{sphinxVerbatim}[commandchars=\\\{\}]
Insira um número:  10
\end{sphinxVerbatim}

\begin{sphinxVerbatim}[commandchars=\\\{\}]
\PYGZsq{}1010\PYGZsq{}
\end{sphinxVerbatim}

\sphinxAtStartPar
Por que ao invés de \sphinxcode{\sphinxupquote{20}} recebemos \sphinxcode{\sphinxupquote{'1010'}}? Porque \sphinxcode{\sphinxupquote{10}} foi recebido como \sphinxcode{\sphinxupquote{'10'}}, ou seja, uma \sphinxcode{\sphinxupquote{string}}.

\sphinxAtStartPar
Logo, o exemplo correto seria:

\begin{sphinxVerbatim}[commandchars=\\\{\}]
\PYG{n}{number} \PYG{o}{=} \PYG{n+nb}{float}\PYG{p}{(}\PYG{n+nb}{input}\PYG{p}{(}\PYG{l+s+s1}{\PYGZsq{}}\PYG{l+s+s1}{Insira um número: }\PYG{l+s+s1}{\PYGZsq{}}\PYG{p}{)}\PYG{p}{)} \PYG{c+c1}{\PYGZsh{}float para aceitar qualquer número}
\PYG{n}{number} \PYG{o}{*} \PYG{l+m+mi}{2}
\end{sphinxVerbatim}

\begin{sphinxVerbatim}[commandchars=\\\{\}]
Insira um número:  10
\end{sphinxVerbatim}

\begin{sphinxVerbatim}[commandchars=\\\{\}]
20.0
\end{sphinxVerbatim}


\section{Condições}
\label{\detokenize{chapters/2:condicoes}}
\sphinxAtStartPar
Em Python, como na maioria das linguagens de programação podemos utilizar estruturas condicionais para interpretar certas linhas de código somente se certa condição for atendida.

\sphinxAtStartPar
No caso, aqui faremos o uso das palavras\sphinxhyphen{}chave \sphinxcode{\sphinxupquote{if}}, \sphinxcode{\sphinxupquote{else}} e \sphinxcode{\sphinxupquote{elif}}.

\sphinxAtStartPar
A partir de agora, veremos que os espaços em branco são importantes para denotar se certa linha está dentro de uma estrutura.


\subsection{If}
\label{\detokenize{chapters/2:if}}
\begin{sphinxVerbatim}[commandchars=\\\{\}]
\PYG{n}{word} \PYG{o}{=} \PYG{l+s+s2}{\PYGZdq{}}\PYG{l+s+s2}{Renato}\PYG{l+s+s2}{\PYGZdq{}}
\PYG{k}{if} \PYG{l+m+mi}{2} \PYG{o}{\PYGZgt{}} \PYG{l+m+mi}{1}\PYG{p}{:}
    \PYG{n}{word} \PYG{o}{=} \PYG{l+s+s2}{\PYGZdq{}}\PYG{l+s+s2}{Eduardo}\PYG{l+s+s2}{\PYGZdq{}}
\PYG{n}{word}
\end{sphinxVerbatim}

\begin{sphinxVerbatim}[commandchars=\\\{\}]
\PYGZsq{}Eduardo\PYGZsq{}
\end{sphinxVerbatim}

\sphinxAtStartPar
Como visto acima, o if verificou se o booleano (resultado da operação) era verdadeiro. Como era, executou a linha que definia o novo valor para \sphinxcode{\sphinxupquote{word}}. Caso fosse falso teríamos o seguinte resultado:

\begin{sphinxVerbatim}[commandchars=\\\{\}]
\PYG{n}{word} \PYG{o}{=} \PYG{l+s+s2}{\PYGZdq{}}\PYG{l+s+s2}{Renato}\PYG{l+s+s2}{\PYGZdq{}}
\PYG{k}{if} \PYG{l+m+mi}{1} \PYG{o}{\PYGZgt{}} \PYG{l+m+mi}{2}\PYG{p}{:}
    \PYG{n}{word} \PYG{o}{=} \PYG{l+s+s2}{\PYGZdq{}}\PYG{l+s+s2}{Eduardo}\PYG{l+s+s2}{\PYGZdq{}}
\PYG{n}{word}
\end{sphinxVerbatim}

\begin{sphinxVerbatim}[commandchars=\\\{\}]
\PYGZsq{}Renato\PYGZsq{}
\end{sphinxVerbatim}

\sphinxAtStartPar
Agora, se quisermos fazer algo se tal condição for real (como vimos acima) mas, se caso contrário, fazer outra ação? Utilizaremos o \sphinxcode{\sphinxupquote{else}}.

\begin{sphinxVerbatim}[commandchars=\\\{\}]
\PYG{n}{word} \PYG{o}{=} \PYG{l+s+s2}{\PYGZdq{}}\PYG{l+s+s2}{Renato}\PYG{l+s+s2}{\PYGZdq{}}
\PYG{k}{if} \PYG{l+m+mi}{1}\PYG{o}{\PYGZgt{}}\PYG{l+m+mi}{2}\PYG{p}{:}
    \PYG{n}{word} \PYG{o}{=} \PYG{l+s+s2}{\PYGZdq{}}\PYG{l+s+s2}{Eduardo}\PYG{l+s+s2}{\PYGZdq{}}
\PYG{k}{else}\PYG{p}{:}
    \PYG{n}{word} \PYG{o}{=} \PYG{l+s+s2}{\PYGZdq{}}\PYG{l+s+s2}{Flávio}\PYG{l+s+s2}{\PYGZdq{}}
\PYG{n}{word}
\end{sphinxVerbatim}

\begin{sphinxVerbatim}[commandchars=\\\{\}]
\PYGZsq{}Flávio\PYGZsq{}
\end{sphinxVerbatim}

\sphinxAtStartPar
A estrutura acima é muito simples e frequentemente utilizada. Ela executa o primeiro bloco caso a condição for verdadeira, caso contrário, executa a segunda.

\sphinxAtStartPar
Mas se quisermos criar condições intermediárias (ou específicas) utilizamos o \sphinxcode{\sphinxupquote{elif}}. Ele significa algo como: “Se o anterior for falso, verifica se essa outra condição é verdadeira antes de ir pro \sphinxcode{\sphinxupquote{else}}”.

\begin{sphinxVerbatim}[commandchars=\\\{\}]
\PYG{n}{lst} \PYG{o}{=} \PYG{p}{[}\PYG{l+m+mi}{0}\PYG{p}{,}\PYG{l+m+mi}{1}\PYG{p}{,}\PYG{l+m+mi}{2}\PYG{p}{]}

\PYG{k}{if} \PYG{l+m+mi}{3} \PYG{o+ow}{in} \PYG{n}{lst}\PYG{p}{:}
    \PYG{n+nb}{print}\PYG{p}{(}\PYG{l+s+s2}{\PYGZdq{}}\PYG{l+s+s2}{3 está!}\PYG{l+s+s2}{\PYGZdq{}}\PYG{p}{)} \PYG{c+c1}{\PYGZsh{} Em notebooks o print é desnecessário, mas esse é um exemplo geral.}
\PYG{k}{elif} \PYG{l+m+mi}{2} \PYG{o+ow}{in} \PYG{n}{lst}\PYG{p}{:}
    \PYG{n+nb}{print}\PYG{p}{(}\PYG{l+s+s2}{\PYGZdq{}}\PYG{l+s+s2}{2 está!}\PYG{l+s+s2}{\PYGZdq{}}\PYG{p}{)}
\PYG{k}{else}\PYG{p}{:}
    \PYG{n+nb}{print}\PYG{p}{(}\PYG{l+s+s2}{\PYGZdq{}}\PYG{l+s+s2}{3 e 2 não estão!}\PYG{l+s+s2}{\PYGZdq{}}\PYG{p}{)}
\end{sphinxVerbatim}

\begin{sphinxVerbatim}[commandchars=\\\{\}]
2 está!
\end{sphinxVerbatim}


\section{Métodos}
\label{\detokenize{chapters/2:metodos}}
\sphinxAtStartPar
Métodos são funções chamadas a partir de um objeto de um tipo específico. Novamente, não precisa se preocupar caso não tenha entendido, é mais fácil do que aparenta.

\sphinxAtStartPar
Esses métodos são acessados através de \sphinxcode{\sphinxupquote{.}}. E cada tipo terá os seus, ou seja, \sphinxcode{\sphinxupquote{str}} tem seus próprios métodos, assim como \sphinxcode{\sphinxupquote{list}} tem seus próprios.

\sphinxAtStartPar
Não é viável tratar cada um deles aqui, mas é interessante que sempre busque pela documentação quando sentir necessidade. Por exemplo, aqui estão \sphinxhref{https://docs.python.org/3/tutorial/datastructures.html}{todos os métodos de \sphinxcode{\sphinxupquote{list}}.}

\sphinxAtStartPar
Alguns exemplos:

\begin{sphinxVerbatim}[commandchars=\\\{\}]
\PYG{c+c1}{\PYGZsh{} Todas as letras maiúsculas em uma string}
\PYG{n}{word} \PYG{o}{=} \PYG{l+s+s2}{\PYGZdq{}}\PYG{l+s+s2}{eDuArdO}\PYG{l+s+s2}{\PYGZdq{}}
\PYG{n}{word} \PYG{o}{=} \PYG{n}{word}\PYG{o}{.}\PYG{n}{upper}\PYG{p}{(}\PYG{p}{)}
\PYG{n}{word}
\end{sphinxVerbatim}

\begin{sphinxVerbatim}[commandchars=\\\{\}]
\PYGZsq{}EDUARDO\PYGZsq{}
\end{sphinxVerbatim}

\begin{sphinxVerbatim}[commandchars=\\\{\}]
\PYG{c+c1}{\PYGZsh{} Todas as letras minúsculas em uma string}
\PYG{n}{word} \PYG{o}{=} \PYG{l+s+s2}{\PYGZdq{}}\PYG{l+s+s2}{eDuArdO}\PYG{l+s+s2}{\PYGZdq{}}
\PYG{n}{word} \PYG{o}{=} \PYG{n}{word}\PYG{o}{.}\PYG{n}{lower}\PYG{p}{(}\PYG{p}{)}
\PYG{n}{word}
\end{sphinxVerbatim}

\begin{sphinxVerbatim}[commandchars=\\\{\}]
\PYGZsq{}eduardo\PYGZsq{}
\end{sphinxVerbatim}

\begin{sphinxVerbatim}[commandchars=\\\{\}]
\PYG{c+c1}{\PYGZsh{} Somente a primeira letra maiúscula em uma string}
\PYG{n}{word} \PYG{o}{=} \PYG{l+s+s2}{\PYGZdq{}}\PYG{l+s+s2}{eDuArdO adaME}\PYG{l+s+s2}{\PYGZdq{}}
\PYG{n}{word} \PYG{o}{=} \PYG{n}{word}\PYG{o}{.}\PYG{n}{title}\PYG{p}{(}\PYG{p}{)}
\PYG{n}{word}
\end{sphinxVerbatim}

\begin{sphinxVerbatim}[commandchars=\\\{\}]
\PYGZsq{}Eduardo Adame\PYGZsq{}
\end{sphinxVerbatim}

\sphinxAtStartPar
Esses métodos de strings são muito úteis em condições. Note que \sphinxcode{\sphinxupquote{'eduardo'}} é diferente de \sphinxcode{\sphinxupquote{'Eduardo'}}. Outros exemplos são mais úteis para limpeza de dados e/ou correção, como o abaixo.

\begin{sphinxVerbatim}[commandchars=\\\{\}]
\PYG{c+c1}{\PYGZsh{} Substitui todas ocorrências de uma substring}
\PYG{n}{phrase} \PYG{o}{=} \PYG{l+s+s2}{\PYGZdq{}}\PYG{l+s+s2}{Penso, logo existo}\PYG{l+s+s2}{\PYGZdq{}}
\PYG{n}{phrase} \PYG{o}{=} \PYG{n}{phrase}\PYG{o}{.}\PYG{n}{replace}\PYG{p}{(}\PYG{l+s+s1}{\PYGZsq{}}\PYG{l+s+s1}{logo}\PYG{l+s+s1}{\PYGZsq{}}\PYG{p}{,} \PYG{l+s+s1}{\PYGZsq{}}\PYG{l+s+s1}{portanto}\PYG{l+s+s1}{\PYGZsq{}}\PYG{p}{)}
\PYG{n}{phrase}
\end{sphinxVerbatim}

\begin{sphinxVerbatim}[commandchars=\\\{\}]
\PYGZsq{}Penso, portanto existo\PYGZsq{}
\end{sphinxVerbatim}

\sphinxAtStartPar
Para listas, os seus métodos são suas principais “operações”.

\begin{sphinxVerbatim}[commandchars=\\\{\}]
\PYG{n}{lst} \PYG{o}{=} \PYG{p}{[}\PYG{l+m+mi}{0}\PYG{p}{,}\PYG{l+m+mi}{2}\PYG{p}{,}\PYG{l+m+mi}{4}\PYG{p}{]}
\PYG{c+c1}{\PYGZsh{} Adiciona um item a uma lista}
\PYG{n}{lst}\PYG{o}{.}\PYG{n}{append}\PYG{p}{(}\PYG{l+m+mi}{6}\PYG{p}{)}
\PYG{n}{lst} \PYG{c+c1}{\PYGZsh{} Note que ele altera a variável diretamente}
\end{sphinxVerbatim}

\begin{sphinxVerbatim}[commandchars=\\\{\}]
[0, 2, 4, 6]
\end{sphinxVerbatim}

\begin{sphinxVerbatim}[commandchars=\\\{\}]
\PYG{k}{if} \PYG{l+m+mi}{2} \PYG{o+ow}{in} \PYG{n}{lst}\PYG{p}{:} \PYG{c+c1}{\PYGZsh{} Forma de evitar erros}
    \PYG{n}{lst}\PYG{o}{.}\PYG{n}{remove}\PYG{p}{(}\PYG{l+m+mi}{2}\PYG{p}{)} \PYG{c+c1}{\PYGZsh{} Remove um item da lista}
\PYG{n}{lst}
\end{sphinxVerbatim}

\begin{sphinxVerbatim}[commandchars=\\\{\}]
[0, 4, 6]
\end{sphinxVerbatim}

\begin{sphinxVerbatim}[commandchars=\\\{\}]
\PYG{c+c1}{\PYGZsh{} Remover um item pelo seu índice}
\PYG{n}{lst}\PYG{o}{.}\PYG{n}{pop}\PYG{p}{(}\PYG{l+m+mi}{2}\PYG{p}{)}
\PYG{n}{lst}
\end{sphinxVerbatim}

\begin{sphinxVerbatim}[commandchars=\\\{\}]
[0, 4]
\end{sphinxVerbatim}

\begin{sphinxVerbatim}[commandchars=\\\{\}]
\PYG{c+c1}{\PYGZsh{} Extende a lista com outra (semelhante a uma soma)}
\PYG{n}{lst}\PYG{o}{.}\PYG{n}{extend}\PYG{p}{(}\PYG{p}{[}\PYG{l+m+mi}{2}\PYG{p}{,}\PYG{l+m+mi}{6}\PYG{p}{]}\PYG{p}{)}
\PYG{n}{lst}
\end{sphinxVerbatim}

\begin{sphinxVerbatim}[commandchars=\\\{\}]
[0, 4, 2, 6]
\end{sphinxVerbatim}

\begin{sphinxVerbatim}[commandchars=\\\{\}]
\PYG{c+c1}{\PYGZsh{} Limpa a lista}
\PYG{n}{lst}\PYG{o}{.}\PYG{n}{clear}\PYG{p}{(}\PYG{p}{)}
\PYG{n}{lst}
\end{sphinxVerbatim}

\begin{sphinxVerbatim}[commandchars=\\\{\}]
[]
\end{sphinxVerbatim}


\section{Estruturas de Repetição}
\label{\detokenize{chapters/2:estruturas-de-repeticao}}
\sphinxAtStartPar
Esse tipo de estrutura, como diz o seu nome, é utilizada para repetir certo bloco de código. No caso, há duas formas de criar esse tipo de \sphinxstyleemphasis{loop}.
\begin{itemize}
\item {} 
\sphinxAtStartPar
\sphinxcode{\sphinxupquote{while}}: Repete o bloco enquanto certa condição verdadeira

\item {} 
\sphinxAtStartPar
\sphinxcode{\sphinxupquote{for}}: Repete o bloco para cada item em uma lista

\end{itemize}

\sphinxAtStartPar
Nesse ponto do curso gostaria de fazer uma menção sobre uma decisão que tomei quanto ao método de ensino. Utilizar listas não é a unica forma de agrupar elementos, por exemplo, existem os \sphinxcode{\sphinxupquote{sets}} e as \sphinxcode{\sphinxupquote{tuples}}. Contudo, todos eles são iteráveis. Ou seja, podemos navegar por seus elementos. Portanto, tudo que se refere a iterabilidade da lista, também serve para esses outros tipos (que pretendo falar posteriomente).

\sphinxAtStartPar
Exemplos:

\begin{sphinxVerbatim}[commandchars=\\\{\}]
\PYG{n}{entry} \PYG{o}{=} \PYG{n+nb}{int}\PYG{p}{(}\PYG{n+nb}{input}\PYG{p}{(}\PYG{l+s+s1}{\PYGZsq{}}\PYG{l+s+s1}{Digite um inteiro: }\PYG{l+s+s1}{\PYGZsq{}}\PYG{p}{)}\PYG{p}{)}
\PYG{k}{while} \PYG{n}{entry} \PYG{o}{!=} \PYG{l+m+mi}{7}\PYG{p}{:} \PYG{c+c1}{\PYGZsh{} Verifica se a entrada é igual a 7.}
    \PYG{n}{entry} \PYG{o}{=} \PYG{n+nb}{int}\PYG{p}{(}\PYG{n+nb}{input}\PYG{p}{(}\PYG{l+s+s1}{\PYGZsq{}}\PYG{l+s+s1}{Digite um inteiro: }\PYG{l+s+s1}{\PYGZsq{}}\PYG{p}{)}\PYG{p}{)}
\PYG{n+nb}{print}\PYG{p}{(}\PYG{l+s+s2}{\PYGZdq{}}\PYG{l+s+s2}{Loop encerrado!}\PYG{l+s+s2}{\PYGZdq{}}\PYG{p}{)}
\end{sphinxVerbatim}

\begin{sphinxVerbatim}[commandchars=\\\{\}]
Digite um inteiro:  2
Digite um inteiro:  3
Digite um inteiro:  4
Digite um inteiro:  8
Digite um inteiro:  7
\end{sphinxVerbatim}

\begin{sphinxVerbatim}[commandchars=\\\{\}]
Loop encerrado!
\end{sphinxVerbatim}

\sphinxAtStartPar
Tome cuidado com loops infinitos! A estrutura \sphinxcode{\sphinxupquote{while}} é bem propícia a isso.

\begin{sphinxVerbatim}[commandchars=\\\{\}]
\PYG{n}{lst} \PYG{o}{=} \PYG{p}{[}\PYG{l+m+mi}{0}\PYG{p}{,}\PYG{l+m+mi}{2}\PYG{p}{,}\PYG{l+m+mi}{4}\PYG{p}{,}\PYG{l+m+mi}{6}\PYG{p}{]}
\PYG{k}{for} \PYG{n}{item} \PYG{o+ow}{in} \PYG{n}{lst}\PYG{p}{:} \PYG{c+c1}{\PYGZsh{} Para cada item da lista}
    \PYG{n+nb}{print}\PYG{p}{(}\PYG{n}{item} \PYG{o}{*}\PYG{o}{*} \PYG{l+m+mi}{2}\PYG{p}{)} \PYG{c+c1}{\PYGZsh{} Imprima o dobro do item}
\end{sphinxVerbatim}

\begin{sphinxVerbatim}[commandchars=\\\{\}]
0
4
16
36
\end{sphinxVerbatim}

\sphinxAtStartPar
É possível criar coisas muito poderozas com o for, inclusive na visualização de dados (em texto).

\begin{sphinxVerbatim}[commandchars=\\\{\}]
\PYG{n}{lst} \PYG{o}{=} \PYG{p}{[}\PYG{l+m+mi}{0}\PYG{p}{,}\PYG{l+m+mi}{2}\PYG{p}{,}\PYG{l+m+mi}{3}\PYG{p}{,}\PYG{l+m+mi}{5}\PYG{p}{,}\PYG{l+m+mi}{6}\PYG{p}{]}
\PYG{k}{for} \PYG{n}{item} \PYG{o+ow}{in} \PYG{n}{lst}\PYG{p}{:} \PYG{c+c1}{\PYGZsh{} Para cada item da lista}
    \PYG{n}{sq} \PYG{o}{=} \PYG{n}{item} \PYG{o}{*}\PYG{o}{*}\PYG{l+m+mi}{2}
    \PYG{k}{if} \PYG{n}{item} \PYG{o}{\PYGZpc{}} \PYG{l+m+mi}{2} \PYG{o}{==} \PYG{l+m+mi}{0}\PYG{p}{:} \PYG{c+c1}{\PYGZsh{} Checa se o número é divisível por 2 (par)}
        \PYG{n}{is\PYGZus{}even} \PYG{o}{=} \PYG{l+s+s2}{\PYGZdq{}}\PYG{l+s+s2}{par}\PYG{l+s+s2}{\PYGZdq{}}
    \PYG{k}{else}\PYG{p}{:}
        \PYG{n}{is\PYGZus{}even} \PYG{o}{=} \PYG{l+s+s2}{\PYGZdq{}}\PYG{l+s+s2}{ímpar}\PYG{l+s+s2}{\PYGZdq{}}
    \PYG{n+nb}{print}\PYG{p}{(}\PYG{l+s+sa}{f}\PYG{l+s+s1}{\PYGZsq{}}\PYG{l+s+s1}{O número }\PYG{l+s+si}{\PYGZob{}}\PYG{n}{item}\PYG{l+s+si}{\PYGZcb{}}\PYG{l+s+s1}{ é }\PYG{l+s+si}{\PYGZob{}}\PYG{n}{is\PYGZus{}even}\PYG{l+s+si}{\PYGZcb{}}\PYG{l+s+s1}{, e seu quadrado é }\PYG{l+s+si}{\PYGZob{}}\PYG{n}{sq}\PYG{l+s+si}{\PYGZcb{}}\PYG{l+s+s1}{\PYGZsq{}}\PYG{p}{)} \PYG{c+c1}{\PYGZsh{} Imprima o dobro do item}
\end{sphinxVerbatim}

\begin{sphinxVerbatim}[commandchars=\\\{\}]
O número 0 é par, e seu quadrado é 0
O número 2 é par, e seu quadrado é 4
O número 3 é ímpar, e seu quadrado é 9
O número 5 é ímpar, e seu quadrado é 25
O número 6 é par, e seu quadrado é 36
\end{sphinxVerbatim}


\section{Funções}
\label{\detokenize{chapters/2:funcoes}}
\sphinxAtStartPar
Funções são blocos de código que podem ser reutilizados. Elas podem ser definidas de duas formas:
\begin{itemize}
\item {} 
\sphinxAtStartPar
Utilizando \sphinxcode{\sphinxupquote{def}}

\item {} 
\sphinxAtStartPar
Utilizando \sphinxcode{\sphinxupquote{lambda}} (chamadas funções anônimas)

\end{itemize}

\sphinxAtStartPar
O uso mais comum, entretanto, é utilizando o \sphinxcode{\sphinxupquote{def}}.

\sphinxAtStartPar
Exemplo:

\begin{sphinxVerbatim}[commandchars=\\\{\}]
\PYG{k}{def} \PYG{n+nf}{square}\PYG{p}{(}\PYG{n}{x}\PYG{p}{)}\PYG{p}{:}
    \PYG{k}{return} \PYG{n}{x}\PYG{o}{*}\PYG{o}{*}\PYG{l+m+mi}{2}

\PYG{n}{square}\PYG{p}{(}\PYG{l+m+mi}{89}\PYG{p}{)}
\end{sphinxVerbatim}

\begin{sphinxVerbatim}[commandchars=\\\{\}]
7921
\end{sphinxVerbatim}

\sphinxAtStartPar
Note que, ao utilizar \sphinxcode{\sphinxupquote{def}}, nós criamos um bloco. Nós devemos nomear uma função e colocar entre parênteses seus parâmetros (ou deixá\sphinxhyphen{}lo vazio). É possível definir valores padrões para certos parâmetros. As variáveis criadas dentro do bloco só existirão dentro dele (inclusive sobrescrevendo temporariamente uma variável global).

\sphinxAtStartPar
O termo return se refere a linha que define o valor retornado por essa função. Quando utilizamos notebooks, o Jupyter imprime o retorno da última linha, por isso não utilizamos print. Mas, utilizando esse exemplo, se estivéssemos no desenvolvimento regular, seria algo como \sphinxcode{\sphinxupquote{print(square(89))}}.

\sphinxAtStartPar
Alguns exemplos com \sphinxcode{\sphinxupquote{def}}:

\begin{sphinxVerbatim}[commandchars=\\\{\}]
\PYG{k}{def} \PYG{n+nf}{count\PYGZus{}even}\PYG{p}{(}\PYG{n}{lst}\PYG{p}{)}\PYG{p}{:}
    \PYG{n}{count} \PYG{o}{=} \PYG{l+m+mi}{0}
    \PYG{k}{for} \PYG{n}{item} \PYG{o+ow}{in} \PYG{n}{lst}\PYG{p}{:}
        \PYG{k}{if} \PYG{n}{item} \PYG{o}{\PYGZpc{}} \PYG{l+m+mi}{2} \PYG{o}{==} \PYG{l+m+mi}{0}\PYG{p}{:}
            \PYG{n}{count} \PYG{o}{+}\PYG{o}{=} \PYG{l+m+mi}{1}
    \PYG{k}{return} \PYG{n}{count}

\PYG{n}{count\PYGZus{}even}\PYG{p}{(}\PYG{p}{[}\PYG{l+m+mi}{1}\PYG{p}{,}\PYG{l+m+mi}{2}\PYG{p}{,}\PYG{l+m+mi}{4}\PYG{p}{,}\PYG{l+m+mi}{5}\PYG{p}{,}\PYG{l+m+mi}{8}\PYG{p}{]}\PYG{p}{)}
\end{sphinxVerbatim}

\begin{sphinxVerbatim}[commandchars=\\\{\}]
3
\end{sphinxVerbatim}

\begin{sphinxVerbatim}[commandchars=\\\{\}]
\PYG{c+c1}{\PYGZsh{} Como utilizar variáveis globais}

\PYG{n}{breads} \PYG{o}{=} \PYG{l+m+mi}{100}
\PYG{n}{currency} \PYG{o}{=} \PYG{l+m+mi}{0}
\PYG{k}{def} \PYG{n+nf}{sell}\PYG{p}{(}\PYG{p}{)}\PYG{p}{:}
    \PYG{k}{global} \PYG{n}{breads}\PYG{p}{,} \PYG{n}{currency} \PYG{c+c1}{\PYGZsh{} Crie as variáveis com a palavra\PYGZhy{}chave global}
    \PYG{n}{breads} \PYG{o}{\PYGZhy{}}\PYG{o}{=}\PYG{l+m+mi}{1}
    \PYG{n}{currency} \PYG{o}{+}\PYG{o}{=} \PYG{l+m+mf}{.5}
    
\PYG{n}{sell}\PYG{p}{(}\PYG{p}{)}

\PYG{n}{breads}\PYG{p}{,}\PYG{n}{currency}
\end{sphinxVerbatim}

\begin{sphinxVerbatim}[commandchars=\\\{\}]
(99, 0.5)
\end{sphinxVerbatim}

\begin{sphinxVerbatim}[commandchars=\\\{\}]
\PYG{k}{def} \PYG{n+nf}{print\PYGZus{}id}\PYG{p}{(}\PYG{n}{name}\PYG{p}{,} \PYG{n}{age}\PYG{p}{,} \PYG{n}{course} \PYG{o}{=} \PYG{l+s+s1}{\PYGZsq{}}\PYG{l+s+s1}{\PYGZsq{}}\PYG{p}{)}\PYG{p}{:}
    \PYG{n}{result} \PYG{o}{=} \PYG{l+s+sa}{f}\PYG{l+s+s1}{\PYGZsq{}}\PYG{l+s+s1}{ Nome: }\PYG{l+s+si}{\PYGZob{}}\PYG{n}{name}\PYG{l+s+si}{\PYGZcb{}}\PYG{l+s+s1}{ }\PYG{l+s+se}{\PYGZbs{}n}\PYG{l+s+s1}{ Idade: }\PYG{l+s+si}{\PYGZob{}}\PYG{n}{age}\PYG{l+s+si}{\PYGZcb{}}\PYG{l+s+s1}{\PYGZsq{}} \PYG{c+c1}{\PYGZsh{} \PYGZbs{}n para quebrar a linha}
    \PYG{k}{if} \PYG{n}{course}\PYG{p}{:}
        \PYG{n}{result} \PYG{o}{+}\PYG{o}{=} \PYG{l+s+sa}{f}\PYG{l+s+s1}{\PYGZsq{}}\PYG{l+s+s1}{ }\PYG{l+s+se}{\PYGZbs{}n}\PYG{l+s+s1}{ Curso: }\PYG{l+s+si}{\PYGZob{}}\PYG{n}{course}\PYG{l+s+si}{\PYGZcb{}}\PYG{l+s+s1}{\PYGZsq{}}
    \PYG{k}{return} \PYG{n+nb}{print}\PYG{p}{(}\PYG{n}{result}\PYG{p}{)} \PYG{c+c1}{\PYGZsh{} Necessário para \PYGZbs{}n funcionar}

\PYG{n}{print\PYGZus{}id}\PYG{p}{(}\PYG{l+s+s1}{\PYGZsq{}}\PYG{l+s+s1}{Eduardo}\PYG{l+s+s1}{\PYGZsq{}}\PYG{p}{,} \PYG{l+m+mi}{18}\PYG{p}{)}
\end{sphinxVerbatim}

\begin{sphinxVerbatim}[commandchars=\\\{\}]
 Nome: Eduardo 
 Idade: 18
\end{sphinxVerbatim}

\begin{sphinxVerbatim}[commandchars=\\\{\}]
\PYG{n}{print\PYGZus{}id}\PYG{p}{(}\PYG{l+s+s1}{\PYGZsq{}}\PYG{l+s+s1}{Eduardo}\PYG{l+s+s1}{\PYGZsq{}}\PYG{p}{,} \PYG{l+m+mi}{18}\PYG{p}{,} \PYG{l+s+s1}{\PYGZsq{}}\PYG{l+s+s1}{Ciência de Dados}\PYG{l+s+s1}{\PYGZsq{}}\PYG{p}{)}
\end{sphinxVerbatim}

\begin{sphinxVerbatim}[commandchars=\\\{\}]
 Nome: Eduardo 
 Idade: 18 
 Curso: Ciência de Dados
\end{sphinxVerbatim}

\sphinxAtStartPar
Para funções de linha única, o \sphinxcode{\sphinxupquote{lambda}} pode ser uma opção melhor. Ou quando uma função recebe outra como parâmetro. Muitos cursos não abordam esse tipo de função, mas acredito que passar rapidamente por ela pode ser bom.

\sphinxAtStartPar
Anteriormente definimos a função \sphinxcode{\sphinxupquote{square()}}, podemos definir uma função que faz o mesmo trabalho da seguinte forma:

\begin{sphinxVerbatim}[commandchars=\\\{\}]
\PYG{n}{sqr\PYGZus{}anon} \PYG{o}{=}  \PYG{k}{lambda} \PYG{n}{x}\PYG{p}{:} \PYG{n}{x}\PYG{o}{*}\PYG{o}{*}\PYG{l+m+mi}{2}

\PYG{n}{sqr\PYGZus{}anon}\PYG{p}{(}\PYG{l+m+mi}{89}\PYG{p}{)} \PYG{c+c1}{\PYGZsh{} É invocada da mesma forma.}
\end{sphinxVerbatim}

\begin{sphinxVerbatim}[commandchars=\\\{\}]
7921
\end{sphinxVerbatim}

\sphinxAtStartPar
Nesse caso, a criamos como se fosse um valor atribuído. O que está entre \sphinxcode{\sphinxupquote{lambda}} e \sphinxcode{\sphinxupquote{:}} são os parâmetros e o que está depois dos \sphinxcode{\sphinxupquote{:}} é o retorno. Nós perderemos o poder de criar o parâmetro opcional como em \sphinxcode{\sphinxupquote{def}}, mas podemos fazer algo parecido com \sphinxcode{\sphinxupquote{print\_id()}}.

\begin{sphinxVerbatim}[commandchars=\\\{\}]
\PYG{n}{id\PYGZus{}anon} \PYG{o}{=} \PYG{k}{lambda} \PYG{n}{name}\PYG{p}{,} \PYG{n}{age}\PYG{p}{:} \PYG{l+s+sa}{f}\PYG{l+s+s1}{\PYGZsq{}}\PYG{l+s+s1}{ Nome: }\PYG{l+s+si}{\PYGZob{}}\PYG{n}{name}\PYG{l+s+si}{\PYGZcb{}}\PYG{l+s+s1}{ }\PYG{l+s+se}{\PYGZbs{}n}\PYG{l+s+s1}{ Idade: }\PYG{l+s+si}{\PYGZob{}}\PYG{n}{age}\PYG{l+s+si}{\PYGZcb{}}\PYG{l+s+s1}{\PYGZsq{}} \PYG{c+c1}{\PYGZsh{} Decidi deixar o print de fora.}
\PYG{n+nb}{print}\PYG{p}{(}\PYG{n}{id\PYGZus{}anon}\PYG{p}{(}\PYG{l+s+s1}{\PYGZsq{}}\PYG{l+s+s1}{Eduardo}\PYG{l+s+s1}{\PYGZsq{}}\PYG{p}{,} \PYG{l+m+mi}{18}\PYG{p}{)}\PYG{p}{)}
\end{sphinxVerbatim}

\begin{sphinxVerbatim}[commandchars=\\\{\}]
 Nome: Eduardo 
 Idade: 18
\end{sphinxVerbatim}


\section{Exercícios}
\label{\detokenize{chapters/2:exercicios}}
\sphinxAtStartPar
Agora você sabe mais que o suficiente para utilizar o \sphinxcode{\sphinxupquote{sympy}}. Para verificar que absorveu o aprendido aqui, tente resolver os seguintes exercícios:
\begin{enumerate}
\sphinxsetlistlabels{\arabic}{enumi}{enumii}{}{.}%
\item {} 
\sphinxAtStartPar
Crie uma função que recebe dois números (floats) e retorna o menor elevado pelo maior.

\item {} 
\sphinxAtStartPar
Crie uma função que recebe duas listas, uma com o ponto inicial e outra com o ponto final \sphinxcode{\sphinxupquote{{[}x,y{]}}}, e calcule a distância entre eles.

\item {} 
\sphinxAtStartPar
Crie uma função que recebe números (floats) de quantidade indefinida (pesquise sobre \sphinxcode{\sphinxupquote{*args}}) e retorne a soma deles.

\item {} 
\sphinxAtStartPar
Crie um loop que faz o mesmo que a função acima, contudo, ele deverá encontrar o valor total quando o usuário inserir \sphinxcode{\sphinxupquote{'s'}}.

\item {} 
\sphinxAtStartPar
Crie uma função que calcula as raízes reais de uma equação do segundo grau a partir dos coeficientes \sphinxcode{\sphinxupquote{a}}, \sphinxcode{\sphinxupquote{b}}, \sphinxcode{\sphinxupquote{c}}.

\end{enumerate}


\part{Utilizando o SymPy}


\chapter{Primeiros Passos com o Sympy}
\label{\detokenize{chapters/3:primeiros-passos-com-o-sympy}}\label{\detokenize{chapters/3::doc}}

\section{Instalação}
\label{\detokenize{chapters/3:instalacao}}
\sphinxAtStartPar
Você possivelmente deve estar se perguntando como instalar o \sphinxcode{\sphinxupquote{Sympy}}. Se você já utilizou algum outro módulo em Python, possivelmente imaginou em instalá\sphinxhyphen{}lo utilizando o \sphinxcode{\sphinxupquote{pip}} (software que pedimos que garantisse sua sua instalação no capítulo 0).

\sphinxAtStartPar
Contudo, se você utiliza está utilizando notebooks com o Anaconda (nossa recomendação para esse módulo), o \sphinxcode{\sphinxupquote{Sympy}} já está instalado, basta carregá\sphinxhyphen{}lo.

\sphinxAtStartPar
Caso esteja desenvolvendo em outro ambiente, uma forma de instalar é com o pip, por exemplo:

\begin{sphinxVerbatim}[commandchars=\\\{\}]
pip install sympy
\end{sphinxVerbatim}


\section{Carregando o Módulo}
\label{\detokenize{chapters/3:carregando-o-modulo}}
\sphinxAtStartPar
Para utilizar os comandos do \sphinxcode{\sphinxupquote{Sympy}} de forma nativa em nossos scripts, precisamos importá\sphinxhyphen{}lo globalmente. Para isso utilizamos as palavras\sphinxhyphen{}chave \sphinxcode{\sphinxupquote{import}} e \sphinxcode{\sphinxupquote{from}}. Isso não foi abordado no capítulo anterior devido a sua complexidade, mas essa é uma forma de importar módulos em Python.

\sphinxAtStartPar
Portanto, basta criar e executar a seguinte \sphinxstyleemphasis{chunk}:

\begin{sphinxVerbatim}[commandchars=\\\{\}]
\PYG{k+kn}{from} \PYG{n+nn}{sympy} \PYG{k+kn}{import} \PYG{o}{*}
\PYG{n}{init\PYGZus{}printing}\PYG{p}{(}\PYG{n}{use\PYGZus{}unicode}\PYG{o}{=}\PYG{k+kc}{True}\PYG{p}{,} \PYG{n}{use\PYGZus{}latex}\PYG{o}{=}\PYG{l+s+s1}{\PYGZsq{}}\PYG{l+s+s1}{mathjax}\PYG{l+s+s1}{\PYGZsq{}}\PYG{p}{)} \PYG{c+c1}{\PYGZsh{} Para imprimir LaTeX}
\end{sphinxVerbatim}

\sphinxAtStartPar
O \sphinxcode{\sphinxupquote{*}} significa que estamos importando o módulo por completo.


\section{Trabalhando com expressões matemáticas}
\label{\detokenize{chapters/3:trabalhando-com-expressoes-matematicas}}
\sphinxAtStartPar
Como o \sphinxcode{\sphinxupquote{Sympy}} tem como objetivo o cálculo simbólico, tudo é baseado a partir dos simbólos. Ou seja, as nossas querídas variáveis (como \(x\), \(y\), e \(z\)) sendo interpretadas com suas propriedades matemáticas.

\sphinxAtStartPar
Portanto, para utilizá\sphinxhyphen{}las, precisamos criar seus símbolos. Por enquanto, vamos utilizar somente o \(x\). Então, para o \sphinxcode{\sphinxupquote{x}} do Python significar a variável \(x\) fazemos:

\begin{sphinxVerbatim}[commandchars=\\\{\}]
\PYG{n}{x} \PYG{o}{=} \PYG{n}{symbols}\PYG{p}{(}\PYG{l+s+s1}{\PYGZsq{}}\PYG{l+s+s1}{x}\PYG{l+s+s1}{\PYGZsq{}}\PYG{p}{)}
\PYG{n}{x}
\end{sphinxVerbatim}
\begin{equation*}
\begin{split}\displaystyle x\end{split}
\end{equation*}
\sphinxAtStartPar
Note que nossa saída matemática será processada por um compilador \(\LaTeX\) para facilitar a leitura.

\sphinxAtStartPar
No caso, você pode utilizar \sphinxcode{\sphinxupquote{x}} como um número, e as expressões aparecerão normalmente (sem igualdade).

\begin{sphinxVerbatim}[commandchars=\\\{\}]
\PYG{n}{x}\PYG{o}{*}\PYG{o}{*}\PYG{l+m+mi}{2} \PYG{o}{\PYGZhy{}} \PYG{l+m+mi}{4}\PYG{o}{*}\PYG{n}{x} \PYG{o}{+} \PYG{l+m+mi}{3}
\end{sphinxVerbatim}
\begin{equation*}
\begin{split}\displaystyle x^{2} - 4 x + 3\end{split}
\end{equation*}
\sphinxAtStartPar
Caso você queira a solução de uma expressão que seja igual a 0 (ou suas raízes, em outras palavras), em respeito a uma variável, você pode usar a função \sphinxcode{\sphinxupquote{solve()}}. Ela recebe dois parâmetros obrigatórios, sua expressão e a variável que você quer a solução.

\begin{sphinxVerbatim}[commandchars=\\\{\}]
\PYG{n}{solve}\PYG{p}{(}\PYG{n}{x}\PYG{o}{*}\PYG{o}{*}\PYG{l+m+mi}{2} \PYG{o}{\PYGZhy{}} \PYG{l+m+mi}{4}\PYG{o}{*}\PYG{n}{x} \PYG{o}{+} \PYG{l+m+mi}{3}\PYG{p}{,} \PYG{n}{x}\PYG{p}{)}
\end{sphinxVerbatim}
\begin{equation*}
\begin{split}\displaystyle \left[ 1, \  3\right]\end{split}
\end{equation*}
\begin{sphinxVerbatim}[commandchars=\\\{\}]
\PYG{n}{solve}\PYG{p}{(}\PYG{n}{sqrt}\PYG{p}{(}\PYG{n}{x}\PYG{p}{)} \PYG{o}{\PYGZhy{}} \PYG{p}{(}\PYG{n}{x}\PYG{o}{/}\PYG{l+m+mi}{2}\PYG{p}{)}\PYG{p}{,}\PYG{n}{x}\PYG{p}{)}
\end{sphinxVerbatim}
\begin{equation*}
\begin{split}\displaystyle \left[ 0, \  4\right]\end{split}
\end{equation*}
\sphinxAtStartPar
Inclusive, caso queira uma resposta como costumamos escrever no papel, ou seja, em forma de conjunto e suas condições, podemos utilizar o \sphinxcode{\sphinxupquote{solveset()}}. A maior diferença é que ele pode receber o conjunto numérico onde você quer trabalhar através do parâmetro \sphinxcode{\sphinxupquote{domain}}. Na maioria das vezes podemos utilizar \sphinxcode{\sphinxupquote{domain=S.Reals}} ou \sphinxcode{\sphinxupquote{domain=S.Complexes}}.

\begin{sphinxVerbatim}[commandchars=\\\{\}]
\PYG{n}{solveset}\PYG{p}{(}\PYG{n}{x}\PYG{o}{*}\PYG{o}{*}\PYG{l+m+mi}{2} \PYG{o}{\PYGZhy{}} \PYG{l+m+mi}{4}\PYG{o}{*}\PYG{n}{x}  \PYG{o}{+}\PYG{l+m+mi}{20}\PYG{p}{,}\PYG{n}{x}\PYG{p}{,} \PYG{n}{domain}\PYG{o}{=}\PYG{n}{S}\PYG{o}{.}\PYG{n}{Reals}\PYG{p}{)}
\end{sphinxVerbatim}
\begin{equation*}
\begin{split}\displaystyle \emptyset\end{split}
\end{equation*}
\begin{sphinxVerbatim}[commandchars=\\\{\}]
\PYG{n}{solveset}\PYG{p}{(}\PYG{n}{x}\PYG{o}{*}\PYG{o}{*}\PYG{l+m+mi}{2} \PYG{o}{\PYGZhy{}} \PYG{l+m+mi}{4}\PYG{o}{*}\PYG{n}{x}  \PYG{o}{+}\PYG{l+m+mi}{20}\PYG{p}{,}\PYG{n}{x}\PYG{p}{,} \PYG{n}{domain}\PYG{o}{=}\PYG{n}{S}\PYG{o}{.}\PYG{n}{Complexes}\PYG{p}{)}
\end{sphinxVerbatim}
\begin{equation*}
\begin{split}\displaystyle \left\{2 - 4 i, 2 + 4 i\right\}\end{split}
\end{equation*}
\begin{sphinxVerbatim}[commandchars=\\\{\}]
\PYG{n}{solveset}\PYG{p}{(}\PYG{n}{tan}\PYG{p}{(}\PYG{n}{x}\PYG{p}{)}\PYG{p}{,} \PYG{n}{x}\PYG{p}{,} \PYG{n}{domain}\PYG{o}{=}\PYG{n}{S}\PYG{o}{.}\PYG{n}{Reals}\PYG{p}{)}
\end{sphinxVerbatim}
\begin{equation*}
\begin{split}\displaystyle \left\{2 n \pi\; |\; n \in \mathbb{Z}\right\} \cup \left\{2 n \pi + \pi\; |\; n \in \mathbb{Z}\right\}\end{split}
\end{equation*}
\sphinxAtStartPar
Vamos criar uma variável para armazenar essa primeira expressão para mostrar outros exemplos

\begin{sphinxVerbatim}[commandchars=\\\{\}]
\PYG{n}{expr} \PYG{o}{=} \PYG{n}{x}\PYG{o}{*}\PYG{o}{*}\PYG{l+m+mi}{2} \PYG{o}{\PYGZhy{}} \PYG{l+m+mi}{4}\PYG{o}{*}\PYG{n}{x} \PYG{o}{+} \PYG{l+m+mi}{3}
\end{sphinxVerbatim}

\sphinxAtStartPar
Podemos achar valores utilizando o método \sphinxcode{\sphinxupquote{subs()}}. Novamente, devemos especficiar a variável.

\begin{sphinxVerbatim}[commandchars=\\\{\}]
\PYG{n}{expr}\PYG{o}{.}\PYG{n}{subs}\PYG{p}{(}\PYG{n}{x}\PYG{p}{,}\PYG{l+m+mi}{2}\PYG{p}{)} \PYG{c+c1}{\PYGZsh{} Se y = expr, esse é o valor de y quando x = 2.}
\end{sphinxVerbatim}
\begin{equation*}
\begin{split}\displaystyle -1\end{split}
\end{equation*}
\begin{sphinxVerbatim}[commandchars=\\\{\}]
\PYG{n}{expr}\PYG{o}{.}\PYG{n}{subs}\PYG{p}{(}\PYG{n}{x}\PYG{p}{,}\PYG{l+m+mi}{1}\PYG{p}{)}
\end{sphinxVerbatim}
\begin{equation*}
\begin{split}\displaystyle 0\end{split}
\end{equation*}
\sphinxAtStartPar
Se tivermos uma expressão numérica não\sphinxhyphen{}inteira e quisermos achar a solução em um ponto flutuante (\sphinxcode{\sphinxupquote{float}}), podemos usar o método \sphinxcode{\sphinxupquote{evalf()}}.

\begin{sphinxVerbatim}[commandchars=\\\{\}]
\PYG{n}{my\PYGZus{}sqrt} \PYG{o}{=} \PYG{n}{sqrt}\PYG{p}{(}\PYG{l+m+mi}{8}\PYG{p}{)}
\PYG{n}{my\PYGZus{}sqrt}
\end{sphinxVerbatim}
\begin{equation*}
\begin{split}\displaystyle 2 \sqrt{2}\end{split}
\end{equation*}
\begin{sphinxVerbatim}[commandchars=\\\{\}]
\PYG{n}{my\PYGZus{}sqrt}\PYG{o}{.}\PYG{n}{evalf}\PYG{p}{(}\PYG{p}{)}
\end{sphinxVerbatim}
\begin{equation*}
\begin{split}\displaystyle 2.82842712474619\end{split}
\end{equation*}
\sphinxAtStartPar
Como viu acima, possivelmente há uma função do \sphinxcode{\sphinxupquote{SymPy}} que represente uma operação ou função matemática. Por exemplo, temos \sphinxcode{\sphinxupquote{sqrt()}}, \sphinxcode{\sphinxupquote{log()}}, \sphinxcode{\sphinxupquote{exp()}}, \sphinxcode{\sphinxupquote{sin()}} e etc. Quando sentir necessidade de utilizar uma dessa, tente antes de consultar a documentação. Caso não consiga “adivinhar”, faça uma consulta que, com toda certeza, haverá uma função que te atenderá.

\sphinxAtStartPar
Existem algumas funções que “fazem Álgebra” por si só. Veja alguns exemplos:

\begin{sphinxVerbatim}[commandchars=\\\{\}]
\PYG{c+c1}{\PYGZsh{} Simplifica}
\PYG{n}{simplify}\PYG{p}{(}\PYG{p}{(}\PYG{n}{x}\PYG{o}{*}\PYG{o}{*}\PYG{l+m+mi}{2} \PYG{o}{+} \PYG{n}{x}\PYG{p}{)}\PYG{o}{/}\PYG{n}{x}\PYG{p}{)} 
\end{sphinxVerbatim}
\begin{equation*}
\begin{split}\displaystyle x + 1\end{split}
\end{equation*}
\begin{sphinxVerbatim}[commandchars=\\\{\}]
\PYG{c+c1}{\PYGZsh{} Fatora}
\PYG{n}{factor}\PYG{p}{(}\PYG{l+m+mi}{1}\PYG{o}{\PYGZhy{}}\PYG{l+m+mi}{1}\PYG{o}{/}\PYG{n}{x}\PYG{p}{)}
\end{sphinxVerbatim}
\begin{equation*}
\begin{split}\displaystyle \frac{x - 1}{x}\end{split}
\end{equation*}
\begin{sphinxVerbatim}[commandchars=\\\{\}]
\PYG{c+c1}{\PYGZsh{} Expande}
\PYG{n}{expand}\PYG{p}{(}\PYG{p}{(}\PYG{n}{x}\PYG{o}{*}\PYG{o}{*}\PYG{l+m+mi}{2} \PYG{o}{+} \PYG{l+m+mi}{3}\PYG{o}{*}\PYG{n}{x}\PYG{p}{)}\PYG{o}{*}\PYG{o}{*}\PYG{l+m+mi}{3}\PYG{p}{)}
\end{sphinxVerbatim}
\begin{equation*}
\begin{split}\displaystyle x^{6} + 9 x^{5} + 27 x^{4} + 27 x^{3}\end{split}
\end{equation*}
\begin{sphinxVerbatim}[commandchars=\\\{\}]
\PYG{c+c1}{\PYGZsh{} Agrupa potências de uma variável (que vai como segundo parâmetro)}
\PYG{n}{collect}\PYG{p}{(}\PYG{n}{x}\PYG{o}{*}\PYG{o}{*}\PYG{l+m+mi}{2} \PYG{o}{+} \PYG{l+m+mi}{4}\PYG{o}{*}\PYG{n}{x} \PYG{o}{\PYGZhy{}} \PYG{l+m+mi}{2}\PYG{o}{*}\PYG{n}{x}\PYG{o}{*}\PYG{o}{*}\PYG{l+m+mi}{2} \PYG{o}{+} \PYG{n}{x} \PYG{o}{\PYGZhy{}}\PYG{l+m+mi}{20} \PYG{o}{+} \PYG{n}{x}\PYG{o}{*}\PYG{o}{*}\PYG{l+m+mi}{3} \PYG{o}{+} \PYG{l+m+mi}{2}\PYG{p}{,} \PYG{n}{x}\PYG{p}{)}
\end{sphinxVerbatim}
\begin{equation*}
\begin{split}\displaystyle x^{3} - x^{2} + 5 x - 18\end{split}
\end{equation*}
\begin{sphinxVerbatim}[commandchars=\\\{\}]
\PYG{c+c1}{\PYGZsh{} Separa fração em frações parciais}
\PYG{n}{apart}\PYG{p}{(}\PYG{p}{(}\PYG{n}{x}\PYG{o}{*}\PYG{o}{*}\PYG{l+m+mi}{2} \PYG{o}{+} \PYG{l+m+mi}{8}\PYG{o}{*}\PYG{n}{x}\PYG{o}{\PYGZhy{}}\PYG{l+m+mi}{18}\PYG{p}{)}\PYG{o}{/}\PYG{p}{(}\PYG{n}{x}\PYG{o}{*}\PYG{o}{*}\PYG{l+m+mi}{3} \PYG{o}{+} \PYG{l+m+mi}{3}\PYG{o}{*}\PYG{n}{x}\PYG{o}{*}\PYG{o}{*}\PYG{l+m+mi}{2}\PYG{p}{)}\PYG{p}{)}
\end{sphinxVerbatim}
\begin{equation*}
\begin{split}\displaystyle - \frac{11}{3 \left(x + 3\right)} + \frac{14}{3 x} - \frac{6}{x^{2}}\end{split}
\end{equation*}
\sphinxAtStartPar
Além desses principais, ainda há \sphinxcode{\sphinxupquote{trigsimp()}} e \sphinxcode{\sphinxupquote{expand\_trig()}} que simplificam e expandem funções trigonométricas (a partir das identidades de adição de arco). E outras que fazem o mesmo para potências, logaritmos e outros tipos de funções. Nesse caso, acho que vale a pena dar uma olhada na documentação. Elas todas são bem parecidas.

\sphinxAtStartPar
Finalizando esse tópico inicial, temos como substituir uma função em termos de outra. Por exemplo:

\begin{sphinxVerbatim}[commandchars=\\\{\}]
\PYG{n}{sin}\PYG{p}{(}\PYG{n}{x}\PYG{p}{)}\PYG{o}{.}\PYG{n}{rewrite}\PYG{p}{(}\PYG{n}{cos}\PYG{p}{)}
\end{sphinxVerbatim}
\begin{equation*}
\begin{split}\displaystyle \cos{\left(x - \frac{\pi}{2} \right)}\end{split}
\end{equation*}
\begin{sphinxVerbatim}[commandchars=\\\{\}]
\PYG{p}{(}\PYG{n}{x}\PYG{o}{*}\PYG{o}{*}\PYG{l+m+mi}{3}\PYG{p}{)}\PYG{o}{.}\PYG{n}{rewrite}\PYG{p}{(}\PYG{n}{exp}\PYG{p}{)}
\end{sphinxVerbatim}
\begin{equation*}
\begin{split}\displaystyle e^{3 \log{\left(x \right)}}\end{split}
\end{equation*}

\subsection{Equações}
\label{\detokenize{chapters/3:equacoes}}
\sphinxAtStartPar
Ok, nós vimos como utilizar expressões. Mas, como tratamos equações? Como vimos no capítulo anterior \sphinxcode{\sphinxupquote{=}} significa atribuição e \sphinxcode{\sphinxupquote{==}} é uma operação booleana (ou seja, recebemos \sphinxcode{\sphinxupquote{True}} ou \sphinxcode{\sphinxupquote{False}}).

\sphinxAtStartPar
Para equações criamos uma classe \sphinxcode{\sphinxupquote{Eq()}}. Não se preocupe com a nomenclatura, é só uma forma de criar um objeto do tipo \sphinxcode{\sphinxupquote{Eq}}. Para criar esse objeto, passamos dois argumentos: cada lado da equação, respectivamente. Veja:

\begin{sphinxVerbatim}[commandchars=\\\{\}]
\PYG{n}{eq} \PYG{o}{=} \PYG{n}{Eq}\PYG{p}{(}\PYG{n}{x}\PYG{o}{*}\PYG{o}{*}\PYG{l+m+mi}{2}\PYG{p}{,} \PYG{l+m+mi}{2}\PYG{p}{)}
\PYG{n}{eq}
\end{sphinxVerbatim}
\begin{equation*}
\begin{split}\displaystyle x^{2} = 2\end{split}
\end{equation*}
\sphinxAtStartPar
Podemos utilizar o mesmo método para encontrar suas raízes.

\begin{sphinxVerbatim}[commandchars=\\\{\}]
\PYG{n}{solve}\PYG{p}{(}\PYG{n}{eq}\PYG{p}{,}\PYG{n}{x}\PYG{p}{)}
\end{sphinxVerbatim}
\begin{equation*}
\begin{split}\displaystyle \left[ - \sqrt{2}, \  \sqrt{2}\right]\end{split}
\end{equation*}

\subsection{Igualdade}
\label{\detokenize{chapters/3:igualdade}}
\sphinxAtStartPar
Como verificar igualdade entre duas expressões? O \sphinxcode{\sphinxupquote{==}} só servirá para expressões identicas (não somente em valor, mas também nos termos expressos). Para isso, utilizamos o método \sphinxcode{\sphinxupquote{equals()}}. Veja:

\begin{sphinxVerbatim}[commandchars=\\\{\}]
\PYG{n}{expr\PYGZus{}1} \PYG{o}{=} \PYG{n}{sin}\PYG{p}{(}\PYG{n}{x}\PYG{p}{)}\PYG{o}{*}\PYG{o}{*}\PYG{l+m+mi}{2}
\end{sphinxVerbatim}

\begin{sphinxVerbatim}[commandchars=\\\{\}]
\PYG{n}{expr\PYGZus{}2} \PYG{o}{=} \PYG{l+m+mf}{.5}\PYG{o}{*}\PYG{p}{(}\PYG{l+m+mi}{1} \PYG{o}{\PYGZhy{}}\PYG{n}{cos}\PYG{p}{(}\PYG{l+m+mi}{2}\PYG{o}{*}\PYG{n}{x}\PYG{p}{)}\PYG{p}{)}
\end{sphinxVerbatim}

\sphinxAtStartPar
Como veremos abaixo, pelo \sphinxcode{\sphinxupquote{==}} as expressões seriam diferentes.

\begin{sphinxVerbatim}[commandchars=\\\{\}]
\PYG{n}{expr\PYGZus{}1} \PYG{o}{==} \PYG{n}{expr\PYGZus{}2}
\end{sphinxVerbatim}

\begin{sphinxVerbatim}[commandchars=\\\{\}]
False
\end{sphinxVerbatim}

\sphinxAtStartPar
Vejamos pelo método \sphinxcode{\sphinxupquote{equals()}}:

\begin{sphinxVerbatim}[commandchars=\\\{\}]
\PYG{n}{expr\PYGZus{}1}\PYG{o}{.}\PYG{n}{equals}\PYG{p}{(}\PYG{n}{expr\PYGZus{}2}\PYG{p}{)}
\end{sphinxVerbatim}

\begin{sphinxVerbatim}[commandchars=\\\{\}]
True
\end{sphinxVerbatim}

\sphinxAtStartPar
Podemos visualizar a igualdade da seguinte forma:

\begin{sphinxVerbatim}[commandchars=\\\{\}]
\PYG{n}{Eq}\PYG{p}{(}\PYG{n}{expr\PYGZus{}1}\PYG{p}{,}\PYG{n}{expr\PYGZus{}2}\PYG{p}{)}
\end{sphinxVerbatim}
\begin{equation*}
\begin{split}\displaystyle \sin^{2}{\left(x \right)} = 0.5 - 0.5 \cos{\left(2 x \right)}\end{split}
\end{equation*}

\subsection{Sistemas de Equações}
\label{\detokenize{chapters/3:sistemas-de-equacoes}}
\sphinxAtStartPar
Podemos, também utilizando \sphinxcode{\sphinxupquote{solve()}} encontrar as soluções de um sistema de equações. Basta passar uma lista com as equações como parâmetro.

\begin{sphinxVerbatim}[commandchars=\\\{\}]
\PYG{n}{y}\PYG{p}{,} \PYG{n}{z} \PYG{o}{=} \PYG{n}{symbols}\PYG{p}{(}\PYG{l+s+s1}{\PYGZsq{}}\PYG{l+s+s1}{y z}\PYG{l+s+s1}{\PYGZsq{}}\PYG{p}{)} \PYG{c+c1}{\PYGZsh{} Criando o y simbólico}
\PYG{n}{Eqs} \PYG{o}{=} \PYG{p}{[}\PYG{p}{]}
\PYG{n}{Eqs}\PYG{o}{.}\PYG{n}{append}\PYG{p}{(}\PYG{n}{Eq}\PYG{p}{(}\PYG{l+m+mi}{3}\PYG{o}{*}\PYG{n}{x} \PYG{o}{\PYGZhy{}} \PYG{l+m+mi}{3}\PYG{o}{*}\PYG{n}{y}\PYG{p}{,} \PYG{l+m+mi}{20}\PYG{p}{)}\PYG{p}{)}
\PYG{n}{Eqs}\PYG{o}{.}\PYG{n}{append}\PYG{p}{(}\PYG{n}{Eq}\PYG{p}{(}\PYG{o}{\PYGZhy{}}\PYG{l+m+mi}{7}\PYG{o}{*}\PYG{n}{x} \PYG{o}{+} \PYG{l+m+mi}{9}\PYG{o}{*}\PYG{n}{y}\PYG{p}{,} \PYG{o}{\PYGZhy{}}\PYG{l+m+mi}{10}\PYG{p}{)}\PYG{p}{)}
\PYG{n}{solve}\PYG{p}{(}\PYG{n}{Eqs}\PYG{p}{,} \PYG{n}{x}\PYG{p}{,} \PYG{n}{y}\PYG{p}{)}
\end{sphinxVerbatim}
\begin{equation*}
\begin{split}\displaystyle \left\{ x : 25, \  y : \frac{55}{3}\right\}\end{split}
\end{equation*}
\sphinxAtStartPar
Caso esteja tentando resolver um sistema linear (como o acima), é possível utilizar o \sphinxcode{\sphinxupquote{linsolve()}}.

\begin{sphinxVerbatim}[commandchars=\\\{\}]
\PYG{n}{linsolve}\PYG{p}{(}\PYG{n}{Eqs}\PYG{p}{,} \PYG{n}{x}\PYG{p}{,} \PYG{n}{y}\PYG{p}{)}
\end{sphinxVerbatim}
\begin{equation*}
\begin{split}\displaystyle \left\{\left( 25, \  \frac{55}{3}\right)\right\}\end{split}
\end{equation*}
\begin{sphinxVerbatim}[commandchars=\\\{\}]
\PYG{n}{Eqs} \PYG{o}{=} \PYG{p}{[}\PYG{p}{]}
\PYG{n}{Eqs}\PYG{o}{.}\PYG{n}{append}\PYG{p}{(}\PYG{n}{Eq}\PYG{p}{(}\PYG{l+m+mi}{3}\PYG{o}{*}\PYG{n}{x} \PYG{o}{\PYGZhy{}} \PYG{l+m+mi}{3}\PYG{o}{*}\PYG{n}{y} \PYG{o}{+} \PYG{l+m+mi}{2}\PYG{o}{*}\PYG{n}{z}\PYG{p}{,} \PYG{l+m+mi}{20}\PYG{p}{)}\PYG{p}{)}
\PYG{n}{Eqs}\PYG{o}{.}\PYG{n}{append}\PYG{p}{(}\PYG{n}{Eq}\PYG{p}{(}\PYG{o}{\PYGZhy{}}\PYG{l+m+mi}{7}\PYG{o}{*}\PYG{n}{x} \PYG{o}{+} \PYG{l+m+mi}{9}\PYG{o}{*}\PYG{n}{y} \PYG{o}{\PYGZhy{}}\PYG{l+m+mi}{4}\PYG{o}{*}\PYG{n}{z}\PYG{p}{,} \PYG{o}{\PYGZhy{}}\PYG{l+m+mi}{10}\PYG{p}{)}\PYG{p}{)}
\PYG{n}{Eqs}\PYG{o}{.}\PYG{n}{append}\PYG{p}{(}\PYG{n}{Eq}\PYG{p}{(}\PYG{o}{\PYGZhy{}}\PYG{l+m+mi}{7}\PYG{o}{*}\PYG{n}{x} \PYG{o}{+} \PYG{l+m+mi}{9}\PYG{o}{*}\PYG{n}{y} \PYG{o}{+} \PYG{l+m+mi}{5}\PYG{o}{*}\PYG{n}{z}\PYG{p}{,} \PYG{l+m+mi}{40}\PYG{p}{)}\PYG{p}{)}
\PYG{n}{linsolve}\PYG{p}{(}\PYG{n}{Eqs}\PYG{p}{,} \PYG{n}{x}\PYG{p}{,} \PYG{n}{y}\PYG{p}{,}\PYG{n}{z}\PYG{p}{)}
\end{sphinxVerbatim}
\begin{equation*}
\begin{split}\displaystyle \left\{\left( \frac{175}{9}, \  \frac{445}{27}, \  \frac{50}{9}\right)\right\}\end{split}
\end{equation*}
\begin{sphinxVerbatim}[commandchars=\\\{\}]
\PYG{n}{Eqs} \PYG{o}{=} \PYG{p}{[}\PYG{p}{]}
\PYG{n}{Eqs}\PYG{o}{.}\PYG{n}{append}\PYG{p}{(}\PYG{n}{Eq}\PYG{p}{(}\PYG{n}{x}\PYG{o}{*}\PYG{o}{*}\PYG{l+m+mi}{2} \PYG{o}{+} \PYG{n}{y}\PYG{o}{*}\PYG{o}{*}\PYG{l+m+mi}{2}\PYG{p}{,} \PYG{l+m+mi}{18}\PYG{p}{)}\PYG{p}{)} \PYG{c+c1}{\PYGZsh{}  Elipse de centro (0,0) e R\PYGZca{}2 = 18}
\PYG{n}{Eqs}\PYG{o}{.}\PYG{n}{append}\PYG{p}{(}\PYG{n}{Eq}\PYG{p}{(}\PYG{n}{x}\PYG{p}{,} \PYG{n}{y}\PYG{p}{)}\PYG{p}{)} \PYG{c+c1}{\PYGZsh{} Reta identidade}
\PYG{n}{nonlinsolve}\PYG{p}{(}\PYG{n}{Eqs}\PYG{p}{,} \PYG{n}{x}\PYG{p}{,} \PYG{n}{y}\PYG{p}{)} \PYG{c+c1}{\PYGZsh{} Sistema não\PYGZhy{}linear (solve também serve). No nosso caso, 2 pontos de intersecção.}
\end{sphinxVerbatim}
\begin{equation*}
\begin{split}\displaystyle \left\{\left( -3, \  -3\right), \left( 3, \  3\right)\right\}\end{split}
\end{equation*}

\section{Matrizes}
\label{\detokenize{chapters/3:matrizes}}
\sphinxAtStartPar
É bem trivial trabalhar com matrizes no Sympy. De modo geral, basta criar um objeto a partir da classe \sphinxcode{\sphinxupquote{Matrix}}. E passamos uma lista de listas, sendo cada uma das listas uma linha. Veja:

\begin{sphinxVerbatim}[commandchars=\\\{\}]
\PYG{n}{Matrix}\PYG{p}{(}\PYG{p}{[}\PYG{p}{[}\PYG{l+m+mi}{1}\PYG{p}{,}\PYG{l+m+mi}{2}\PYG{p}{,}\PYG{l+m+mi}{3}\PYG{p}{]}\PYG{p}{,}\PYG{p}{[}\PYG{l+m+mi}{2}\PYG{p}{,}\PYG{l+m+mi}{3}\PYG{p}{,}\PYG{l+m+mi}{1}\PYG{p}{]}\PYG{p}{]}\PYG{p}{)}
\end{sphinxVerbatim}
\begin{equation*}
\begin{split}\displaystyle \left[\begin{matrix}1 & 2 & 3\\2 & 3 & 1\end{matrix}\right]\end{split}
\end{equation*}
\sphinxAtStartPar
E podemos manipulá\sphinxhyphen{}las normalmente, com as operações comuns. Além disso, há algumas outras operações especiais.

\begin{sphinxVerbatim}[commandchars=\\\{\}]
\PYG{n}{A} \PYG{o}{=} \PYG{n}{Matrix}\PYG{p}{(}\PYG{p}{[}\PYG{p}{[}\PYG{l+m+mi}{1}\PYG{p}{,}\PYG{l+m+mi}{2}\PYG{p}{,}\PYG{l+m+mi}{3}\PYG{p}{]}\PYG{p}{,}\PYG{p}{[}\PYG{l+m+mi}{2}\PYG{p}{,}\PYG{l+m+mi}{3}\PYG{p}{,}\PYG{l+m+mi}{1}\PYG{p}{]}\PYG{p}{]}\PYG{p}{)}
\PYG{n}{B} \PYG{o}{=} \PYG{n}{Matrix}\PYG{p}{(}\PYG{p}{[}\PYG{p}{[}\PYG{l+m+mi}{3}\PYG{p}{,}\PYG{l+m+mi}{2}\PYG{p}{]}\PYG{p}{,} \PYG{p}{[}\PYG{l+m+mi}{2}\PYG{p}{,}\PYG{l+m+mi}{2}\PYG{p}{]}\PYG{p}{,} \PYG{p}{[}\PYG{l+m+mi}{1}\PYG{p}{,}\PYG{l+m+mi}{4}\PYG{p}{]}\PYG{p}{]}\PYG{p}{)}
\PYG{n}{A}\PYG{p}{,} \PYG{n}{B}
\end{sphinxVerbatim}
\begin{equation*}
\begin{split}\displaystyle \left( \left[\begin{matrix}1 & 2 & 3\\2 & 3 & 1\end{matrix}\right], \  \left[\begin{matrix}3 & 2\\2 & 2\\1 & 4\end{matrix}\right]\right)\end{split}
\end{equation*}
\begin{sphinxVerbatim}[commandchars=\\\{\}]
\PYG{n}{A} \PYG{o}{*} \PYG{n}{B} \PYG{c+c1}{\PYGZsh{} Multiplicação}
\end{sphinxVerbatim}
\begin{equation*}
\begin{split}\displaystyle \left[\begin{matrix}10 & 18\\13 & 14\end{matrix}\right]\end{split}
\end{equation*}
\begin{sphinxVerbatim}[commandchars=\\\{\}]
\PYG{n}{B}\PYG{o}{.}\PYG{n}{T} \PYG{c+c1}{\PYGZsh{} Transpor}
\end{sphinxVerbatim}
\begin{equation*}
\begin{split}\displaystyle \left[\begin{matrix}3 & 2 & 1\\2 & 2 & 4\end{matrix}\right]\end{split}
\end{equation*}
\begin{sphinxVerbatim}[commandchars=\\\{\}]
\PYG{n}{A} \PYG{o}{+} \PYG{n}{B}\PYG{o}{.}\PYG{n}{T} \PYG{c+c1}{\PYGZsh{} Soma}
\end{sphinxVerbatim}
\begin{equation*}
\begin{split}\displaystyle \left[\begin{matrix}4 & 4 & 4\\4 & 5 & 5\end{matrix}\right]\end{split}
\end{equation*}
\begin{sphinxVerbatim}[commandchars=\\\{\}]
\PYG{n}{A}\PYG{o}{.}\PYG{n}{row}\PYG{p}{(}\PYG{l+m+mi}{1}\PYG{p}{)} \PYG{c+c1}{\PYGZsh{} Começa em 0}
\end{sphinxVerbatim}
\begin{equation*}
\begin{split}\displaystyle \left[\begin{matrix}2 & 3 & 1\end{matrix}\right]\end{split}
\end{equation*}
\begin{sphinxVerbatim}[commandchars=\\\{\}]
\PYG{n}{B}\PYG{o}{.}\PYG{n}{col}\PYG{p}{(}\PYG{l+m+mi}{0}\PYG{p}{)} \PYG{c+c1}{\PYGZsh{} Também começa em 0}
\end{sphinxVerbatim}
\begin{equation*}
\begin{split}\displaystyle \left[\begin{matrix}3\\2\\1\end{matrix}\right]\end{split}
\end{equation*}
\begin{sphinxVerbatim}[commandchars=\\\{\}]
\PYG{n}{B} \PYG{o}{=} \PYG{n}{B}\PYG{o}{.}\PYG{n}{col\PYGZus{}insert}\PYG{p}{(}\PYG{l+m+mi}{2}\PYG{p}{,}\PYG{n}{Matrix}\PYG{p}{(}\PYG{p}{[}\PYG{l+m+mi}{2}\PYG{p}{,}\PYG{l+m+mi}{3}\PYG{p}{,}\PYG{l+m+mi}{2}\PYG{p}{]}\PYG{p}{)}\PYG{p}{)} \PYG{c+c1}{\PYGZsh{} Insere Coluna}
\PYG{n}{B}
\end{sphinxVerbatim}
\begin{equation*}
\begin{split}\displaystyle \left[\begin{matrix}3 & 2 & 2\\2 & 2 & 3\\1 & 4 & 2\end{matrix}\right]\end{split}
\end{equation*}
\begin{sphinxVerbatim}[commandchars=\\\{\}]
\PYG{n}{B}\PYG{o}{*}\PYG{o}{*}\PYG{o}{\PYGZhy{}}\PYG{l+m+mi}{1}
\end{sphinxVerbatim}
\begin{equation*}
\begin{split}\displaystyle \left[\begin{matrix}\frac{4}{7} & - \frac{2}{7} & - \frac{1}{7}\\\frac{1}{14} & - \frac{2}{7} & \frac{5}{14}\\- \frac{3}{7} & \frac{5}{7} & - \frac{1}{7}\end{matrix}\right]\end{split}
\end{equation*}

\section{Exercícios}
\label{\detokenize{chapters/3:exercicios}}
\sphinxAtStartPar
Utilizando o que aprendeu nesse capítulo, tente resolver os seguintes exercícios:
\begin{enumerate}
\sphinxsetlistlabels{\arabic}{enumi}{enumii}{}{.}%
\item {} 
\sphinxAtStartPar
Encontre as raízes de cada uma das expressões abaixo. Depois encontre um par \((x,y)\) para cada:

\end{enumerate}
\begin{equation*}
\begin{split}x^3 - 8x^2 + 4x + 3 \end{split}
\end{equation*}\begin{equation*}
\begin{split}\sin(x) + 2\cos(x)\end{split}
\end{equation*}\begin{equation*}
\begin{split}\log\left|\dfrac{x^2 - x}{2}\right|\end{split}
\end{equation*}\begin{equation*}
\begin{split}e^{-x^3 + 5x^2 -x} -1\end{split}
\end{equation*}\begin{enumerate}
\sphinxsetlistlabels{\arabic}{enumi}{enumii}{}{.}%
\item {} 
\sphinxAtStartPar
Encontre as soluções das equações abaixo:

\end{enumerate}
\begin{equation*}
\begin{split}x^4 - 4x^3 + x^2 - 30 = -x^2 + x - 40\end{split}
\end{equation*}\begin{equation*}
\begin{split}2^{x^2 -x} = 3^x\end{split}
\end{equation*}\begin{equation*}
\begin{split}\log\left|x^3 - 2x^2 +x\right| = \log\left|x^2 + 6x\right|\end{split}
\end{equation*}\begin{enumerate}
\sphinxsetlistlabels{\arabic}{enumi}{enumii}{}{.}%
\item {} 
\sphinxAtStartPar
Verifique se as igualdades são verdadeiras:

\end{enumerate}
\begin{equation*}
\begin{split}\Gamma\left(\dfrac{3}{2}\right) = \dfrac{\sqrt{\pi}}{2}\end{split}
\end{equation*}\begin{equation*}
\begin{split}\sin(2x^2) - \cos(x^2 +x) = \sin{\left(x \right)} \sin{\left(x^{2} \right)} + 2 \sin{\left(x^{2} \right)} \cos{\left(x^{2} \right)} - \cos{\left(x \right)} \cos{\left(x^{2} \right)}\end{split}
\end{equation*}\begin{equation*}
\begin{split}(x^2 -3x)(2x^4 + x^3 - x)(-4x^2)= 8 x^{8} + 20 x^{7} + 12 x^{6} + 4 x^{5} - 12 x^{4}\end{split}
\end{equation*}\begin{enumerate}
\sphinxsetlistlabels{\arabic}{enumi}{enumii}{}{.}%
\item {} 
\sphinxAtStartPar
Encontre as soluções do sistema:

\end{enumerate}
\begin{equation*}
\begin{split}\begin{cases}4x - 3y + 2z = 60\\ (x - 10)^2 + y^2 + z^2 = 72\\ 2x + 9y +z = 20 \end{cases}\end{split}
\end{equation*}

\chapter{Aplicações em Cálculo Diferencial e Integral}
\label{\detokenize{chapters/4:aplicacoes-em-calculo-diferencial-e-integral}}\label{\detokenize{chapters/4::doc}}
\sphinxAtStartPar
De modo geral, o principal objetivo do curso é garantir que seus alunos estejam proeficientes no uso de \sphinxcode{\sphinxupquote{SymPy}} no Cálculo. Na minha opinião, esse é o capítulo mais importante do curso. Dê seu máximo para absorver o conteúdo aqui apresentado.

\sphinxAtStartPar
Antes de comerçarmos, certifique\sphinxhyphen{}se que fez as devidas importações e atribuições:

\begin{sphinxVerbatim}[commandchars=\\\{\}]
\PYG{k+kn}{from} \PYG{n+nn}{sympy} \PYG{k+kn}{import} \PYG{o}{*}
\PYG{n}{x}\PYG{p}{,} \PYG{n}{y}\PYG{p}{,} \PYG{n}{z} \PYG{o}{=} \PYG{n}{symbols}\PYG{p}{(}\PYG{l+s+s1}{\PYGZsq{}}\PYG{l+s+s1}{x y z}\PYG{l+s+s1}{\PYGZsq{}}\PYG{p}{)}
\PYG{n}{init\PYGZus{}printing}\PYG{p}{(}\PYG{n}{use\PYGZus{}unicode}\PYG{o}{=}\PYG{k+kc}{True}\PYG{p}{,} \PYG{n}{use\PYGZus{}latex}\PYG{o}{=}\PYG{l+s+s1}{\PYGZsq{}}\PYG{l+s+s1}{mathjax}\PYG{l+s+s1}{\PYGZsq{}}\PYG{p}{)}
\end{sphinxVerbatim}


\section{Intervalos}
\label{\detokenize{chapters/4:intervalos}}
\sphinxAtStartPar
Nós sabemos que o Cálculo é, genericamente, o estudo das mudanças. E nós costumamos definir intervalos para trabalhar com nossas funções e expressões. É bem simples de criá\sphinxhyphen{}los e utilizá\sphinxhyphen{}los no \sphinxcode{\sphinxupquote{SymPy}}.

\sphinxAtStartPar
Para criar um intervalo, criamos um objeto a partir da classe \sphinxcode{\sphinxupquote{Interval}} e/ou um método seu para definir se está aberto em algum dos lados. Veja os exemplos:

\begin{sphinxVerbatim}[commandchars=\\\{\}]
\PYG{c+c1}{\PYGZsh{} Intervalo Fechado}
\PYG{n}{Interval}\PYG{p}{(}\PYG{l+m+mi}{0}\PYG{p}{,}\PYG{l+m+mi}{10}\PYG{p}{)}
\end{sphinxVerbatim}
\begin{equation*}
\begin{split}\displaystyle \left[0, 10\right]\end{split}
\end{equation*}
\begin{sphinxVerbatim}[commandchars=\\\{\}]
\PYG{c+c1}{\PYGZsh{} Intervalo Aberto}
\PYG{n}{Interval}\PYG{o}{.}\PYG{n}{open}\PYG{p}{(}\PYG{o}{\PYGZhy{}}\PYG{l+m+mi}{10}\PYG{p}{,} \PYG{l+m+mi}{20}\PYG{p}{)}
\end{sphinxVerbatim}
\begin{equation*}
\begin{split}\displaystyle \left(-10, 20\right)\end{split}
\end{equation*}
\begin{sphinxVerbatim}[commandchars=\\\{\}]
\PYG{c+c1}{\PYGZsh{} Intervalo Aberto em um dos lados}
\PYG{n}{Interval}\PYG{o}{.}\PYG{n}{Ropen}\PYG{p}{(}\PYG{l+m+mi}{10}\PYG{p}{,}\PYG{l+m+mi}{30}\PYG{p}{)} \PYG{c+c1}{\PYGZsh{} R \PYGZhy{} Direita}
\end{sphinxVerbatim}
\begin{equation*}
\begin{split}\displaystyle \left[10, 30\right)\end{split}
\end{equation*}
\begin{sphinxVerbatim}[commandchars=\\\{\}]
\PYG{n}{Interval}\PYG{o}{.}\PYG{n}{Lopen}\PYG{p}{(}\PYG{l+m+mi}{10}\PYG{p}{,}\PYG{l+m+mi}{30}\PYG{p}{)} \PYG{c+c1}{\PYGZsh{} L \PYGZhy{} Direita}
\end{sphinxVerbatim}
\begin{equation*}
\begin{split}\displaystyle \left(10, 30\right]\end{split}
\end{equation*}
\begin{sphinxVerbatim}[commandchars=\\\{\}]
\PYG{n}{Interval}\PYG{p}{(}\PYG{l+m+mi}{0}\PYG{p}{,}\PYG{n}{oo}\PYG{p}{)} \PYG{c+c1}{\PYGZsh{} oo representa o infinito em sympy. Note que onde oo estiver será aberto.}
\end{sphinxVerbatim}
\begin{equation*}
\begin{split}\displaystyle \left[0, \infty\right)\end{split}
\end{equation*}

\section{Análises de Domínio/Intervalo}
\label{\detokenize{chapters/4:analises-de-dominio-intervalo}}
\sphinxAtStartPar
Existem diversas funções embutidas no \sphinxcode{\sphinxupquote{SymPy}} para avaliar o comportamento das funções/expressões ao longo de seu domínio ou de um intervalo específico. Normalmente elas retornarão um booleano.


\subsection{Verificar se é crescente ou decrescente.}
\label{\detokenize{chapters/4:verificar-se-e-crescente-ou-decrescente}}
\begin{sphinxVerbatim}[commandchars=\\\{\}]
\PYG{c+c1}{\PYGZsh{}\PYGZsh{} x² em seu domínio não é crescente}
\PYG{n}{is\PYGZus{}increasing}\PYG{p}{(}\PYG{n}{x}\PYG{o}{*}\PYG{o}{*}\PYG{l+m+mi}{2}\PYG{p}{)}
\end{sphinxVerbatim}

\begin{sphinxVerbatim}[commandchars=\\\{\}]
False
\end{sphinxVerbatim}

\begin{sphinxVerbatim}[commandchars=\\\{\}]
\PYG{c+c1}{\PYGZsh{}\PYGZsh{} x² em (0, oo) é crescente}
\PYG{n}{is\PYGZus{}increasing}\PYG{p}{(}\PYG{n}{x}\PYG{o}{*}\PYG{o}{*}\PYG{l+m+mi}{2}\PYG{p}{,} \PYG{n}{Interval}\PYG{o}{.}\PYG{n}{open}\PYG{p}{(}\PYG{l+m+mi}{0}\PYG{p}{,} \PYG{n}{oo}\PYG{p}{)}\PYG{p}{)}
\end{sphinxVerbatim}

\begin{sphinxVerbatim}[commandchars=\\\{\}]
True
\end{sphinxVerbatim}

\begin{sphinxVerbatim}[commandchars=\\\{\}]
\PYG{c+c1}{\PYGZsh{}\PYGZsh{} O contrário vale para decreasing}
\PYG{n}{is\PYGZus{}decreasing}\PYG{p}{(}\PYG{n}{x}\PYG{o}{*}\PYG{o}{*}\PYG{l+m+mi}{2}\PYG{p}{,} \PYG{n}{Interval}\PYG{o}{.}\PYG{n}{open}\PYG{p}{(}\PYG{o}{\PYGZhy{}}\PYG{n}{oo}\PYG{p}{,} \PYG{l+m+mi}{0}\PYG{p}{)}\PYG{p}{)}
\end{sphinxVerbatim}

\begin{sphinxVerbatim}[commandchars=\\\{\}]
True
\end{sphinxVerbatim}

\sphinxAtStartPar
Podemos verificar também se ela é estritamente crescente ou decrescente, ou seja, se ela é injetiva.

\begin{sphinxVerbatim}[commandchars=\\\{\}]
\PYG{c+c1}{\PYGZsh{}\PYGZsh{} x³ é crescente em todo seu domínio. (d/dx = 3x² \PYGZgt{}= 0)}
\PYG{n}{is\PYGZus{}increasing}\PYG{p}{(}\PYG{n}{x}\PYG{o}{*}\PYG{o}{*}\PYG{l+m+mi}{3}\PYG{p}{)}
\end{sphinxVerbatim}

\begin{sphinxVerbatim}[commandchars=\\\{\}]
True
\end{sphinxVerbatim}

\begin{sphinxVerbatim}[commandchars=\\\{\}]
\PYG{c+c1}{\PYGZsh{}\PYGZsh{} x³ não é estritamente crescente em seu domínio (3*0² = 0)}
\PYG{n}{is\PYGZus{}strictly\PYGZus{}increasing}\PYG{p}{(}\PYG{n}{x}\PYG{o}{*}\PYG{o}{*}\PYG{l+m+mi}{3}\PYG{p}{)}
\end{sphinxVerbatim}

\begin{sphinxVerbatim}[commandchars=\\\{\}]
\PYG{c+c1}{\PYGZsh{}\PYGZsh{} 1/(e\PYGZca{}x) é estritamente decrescente em seu domínio}
\PYG{n}{is\PYGZus{}strictly\PYGZus{}decreasing}\PYG{p}{(}\PYG{l+m+mi}{1}\PYG{o}{/}\PYG{p}{(}\PYG{n}{exp}\PYG{p}{(}\PYG{n}{x}\PYG{p}{)}\PYG{p}{)}\PYG{p}{)}
\end{sphinxVerbatim}

\begin{sphinxVerbatim}[commandchars=\\\{\}]
True
\end{sphinxVerbatim}

\sphinxAtStartPar
Podemos também verificar se ela é monótona com \sphinxcode{\sphinxupquote{is\_monotonic()}}. Para finalizar, podemos verificar se há pontos (e quais são) com singularidades. Ou seja, que requerem certa atenção. Normalmente, são pontos que não têm limite.

\begin{sphinxVerbatim}[commandchars=\\\{\}]
\PYG{n}{singularities}\PYG{p}{(}\PYG{l+m+mi}{1}\PYG{o}{/}\PYG{n}{x}\PYG{p}{,}\PYG{n}{x}\PYG{p}{)}
\end{sphinxVerbatim}
\begin{equation*}
\begin{split}\displaystyle \left\{0\right\}\end{split}
\end{equation*}

\section{Limites}
\label{\detokenize{chapters/4:limites}}
\sphinxAtStartPar
Assim como veremos posteriormente nas derivadas e nas integrais, há duas formas de criar e calcular limites no \sphinxcode{\sphinxupquote{SymPy}}. A primeria forma é através da classe \sphinxcode{\sphinxupquote{Limit}}, que criará um limite e não calculará seu valor. Ou seja, utilize ela para armazenar a expressão do limite. Caso queira somente calcular o limite. Utilizamos a função \sphinxcode{\sphinxupquote{limit()}}.

\begin{sphinxVerbatim}[commandchars=\\\{\}]
\PYG{n}{Limit}\PYG{p}{(}\PYG{n}{sin}\PYG{p}{(}\PYG{n}{x}\PYG{p}{)}\PYG{o}{/}\PYG{n}{x}\PYG{p}{,} \PYG{n}{x}\PYG{p}{,} \PYG{l+m+mi}{0}\PYG{p}{,} \PYG{l+s+s1}{\PYGZsq{}}\PYG{l+s+s1}{+}\PYG{l+s+s1}{\PYGZsq{}}\PYG{p}{)} \PYG{c+c1}{\PYGZsh{}\PYGZsh{} sin(x)/x, x \PYGZhy{}\PYGZgt{} 0+}
\end{sphinxVerbatim}
\begin{equation*}
\begin{split}\displaystyle \lim_{x \to 0^+}\left(\frac{\sin{\left(x \right)}}{x}\right)\end{split}
\end{equation*}
\begin{sphinxVerbatim}[commandchars=\\\{\}]
\PYG{n}{Limit}\PYG{p}{(}\PYG{l+m+mi}{1}\PYG{o}{/}\PYG{n}{x}\PYG{p}{,} \PYG{n}{x}\PYG{p}{,} \PYG{l+m+mi}{0}\PYG{p}{,} \PYG{l+s+s1}{\PYGZsq{}}\PYG{l+s+s1}{\PYGZhy{}}\PYG{l+s+s1}{\PYGZsq{}}\PYG{p}{)} \PYG{c+c1}{\PYGZsh{}\PYGZsh{} sin(x)/x, x \PYGZhy{}\PYGZgt{} 0\PYGZhy{}}
\end{sphinxVerbatim}
\begin{equation*}
\begin{split}\displaystyle \lim_{x \to 0^-} \frac{1}{x}\end{split}
\end{equation*}
\begin{sphinxVerbatim}[commandchars=\\\{\}]
\PYG{n}{limit}\PYG{p}{(}\PYG{n}{sin}\PYG{p}{(}\PYG{n}{x}\PYG{p}{)}\PYG{o}{/}\PYG{n}{x}\PYG{p}{,} \PYG{n}{x}\PYG{p}{,} \PYG{l+m+mi}{0}\PYG{p}{,} \PYG{l+s+s1}{\PYGZsq{}}\PYG{l+s+s1}{+}\PYG{l+s+s1}{\PYGZsq{}}\PYG{p}{)}
\end{sphinxVerbatim}
\begin{equation*}
\begin{split}\displaystyle 1\end{split}
\end{equation*}
\begin{sphinxVerbatim}[commandchars=\\\{\}]
\PYG{n}{limit}\PYG{p}{(}\PYG{l+m+mi}{1}\PYG{o}{/}\PYG{n}{x}\PYG{p}{,} \PYG{n}{x}\PYG{p}{,} \PYG{l+m+mi}{0}\PYG{p}{)} \PYG{c+c1}{\PYGZsh{} \PYGZsq{}+\PYGZsq{} por padrão}
\end{sphinxVerbatim}
\begin{equation*}
\begin{split}\displaystyle \infty\end{split}
\end{equation*}
\begin{sphinxVerbatim}[commandchars=\\\{\}]
\PYG{n}{limit}\PYG{p}{(}\PYG{l+m+mi}{1}\PYG{o}{/}\PYG{n}{x}\PYG{p}{,} \PYG{n}{x}\PYG{p}{,} \PYG{l+m+mi}{0}\PYG{p}{,} \PYG{l+s+s1}{\PYGZsq{}}\PYG{l+s+s1}{\PYGZhy{}}\PYG{l+s+s1}{\PYGZsq{}}\PYG{p}{)}
\end{sphinxVerbatim}
\begin{equation*}
\begin{split}\displaystyle -\infty\end{split}
\end{equation*}
\begin{sphinxVerbatim}[commandchars=\\\{\}]
\PYG{n}{limit}\PYG{p}{(}\PYG{l+m+mi}{1}\PYG{o}{/}\PYG{n}{x}\PYG{p}{,} \PYG{n}{x}\PYG{p}{,} \PYG{l+m+mi}{0}\PYG{p}{,} \PYG{l+s+s1}{\PYGZsq{}}\PYG{l+s+s1}{+\PYGZhy{}}\PYG{l+s+s1}{\PYGZsq{}}\PYG{p}{)} \PYG{c+c1}{\PYGZsh{} Dois lados}
\end{sphinxVerbatim}
\begin{equation*}
\begin{split}\displaystyle \tilde{\infty}\end{split}
\end{equation*}
\begin{sphinxVerbatim}[commandchars=\\\{\}]
\PYG{n}{my\PYGZus{}sin} \PYG{o}{=} \PYG{n}{Limit}\PYG{p}{(}\PYG{n}{sin}\PYG{p}{(}\PYG{n}{x}\PYG{p}{)}\PYG{o}{/}\PYG{n}{x}\PYG{p}{,} \PYG{n}{x}\PYG{p}{,} \PYG{l+m+mi}{0}\PYG{p}{,} \PYG{l+s+s1}{\PYGZsq{}}\PYG{l+s+s1}{+}\PYG{l+s+s1}{\PYGZsq{}}\PYG{p}{)}
\PYG{n}{my\PYGZus{}sin}\PYG{o}{.}\PYG{n}{doit}\PYG{p}{(}\PYG{p}{)} \PYG{c+c1}{\PYGZsh{} Método doit() calcula uma expressão.}
\end{sphinxVerbatim}
\begin{equation*}
\begin{split}\displaystyle 1\end{split}
\end{equation*}

\section{Derivadas}
\label{\detokenize{chapters/4:derivadas}}
\sphinxAtStartPar
Assim com os limites, podemos criar a derivada (sem calculá\sphinxhyphen{}la) através da classe \sphinxcode{\sphinxupquote{Derivative()}}. E calcular diretamente através da \sphinxcode{\sphinxupquote{diff()}}.

\begin{sphinxVerbatim}[commandchars=\\\{\}]
\PYG{n}{Derivative}\PYG{p}{(}\PYG{n}{exp}\PYG{p}{(}\PYG{l+m+mi}{2}\PYG{o}{*}\PYG{n}{x}\PYG{o}{*}\PYG{o}{*}\PYG{l+m+mi}{3}\PYG{p}{)}\PYG{p}{,}\PYG{n}{x}\PYG{p}{)}
\end{sphinxVerbatim}
\begin{equation*}
\begin{split}\displaystyle \frac{d}{d x} e^{2 x^{3}}\end{split}
\end{equation*}
\begin{sphinxVerbatim}[commandchars=\\\{\}]
\PYG{n}{diff}\PYG{p}{(}\PYG{n}{sin}\PYG{p}{(}\PYG{n}{x}\PYG{o}{*}\PYG{o}{*}\PYG{l+m+mi}{2}\PYG{p}{)}\PYG{p}{,}\PYG{n}{x}\PYG{p}{)}
\end{sphinxVerbatim}
\begin{equation*}
\begin{split}\displaystyle 2 x \cos{\left(x^{2} \right)}\end{split}
\end{equation*}
\begin{sphinxVerbatim}[commandchars=\\\{\}]
\PYG{n}{diff}\PYG{p}{(}\PYG{n}{sin}\PYG{p}{(}\PYG{n}{x}\PYG{o}{*}\PYG{o}{*}\PYG{l+m+mi}{2}\PYG{p}{)}\PYG{p}{,}\PYG{n}{x}\PYG{p}{,} \PYG{n}{x}\PYG{p}{)} \PYG{c+c1}{\PYGZsh{}\PYGZsh{} Calcular a segunda derivada}
\end{sphinxVerbatim}
\begin{equation*}
\begin{split}\displaystyle 2 \left(- 2 x^{2} \sin{\left(x^{2} \right)} + \cos{\left(x^{2} \right)}\right)\end{split}
\end{equation*}
\begin{sphinxVerbatim}[commandchars=\\\{\}]
\PYG{n}{diff}\PYG{p}{(}\PYG{n}{sin}\PYG{p}{(}\PYG{n}{x}\PYG{o}{*}\PYG{o}{*}\PYG{l+m+mi}{2}\PYG{p}{)}\PYG{p}{,}\PYG{n}{x}\PYG{p}{,} \PYG{n}{x}\PYG{p}{,} \PYG{n}{x}\PYG{p}{)} \PYG{c+c1}{\PYGZsh{}\PYGZsh{} Calcular a terceira derivada}
\end{sphinxVerbatim}
\begin{equation*}
\begin{split}\displaystyle - 4 x \left(2 x^{2} \cos{\left(x^{2} \right)} + 3 \sin{\left(x^{2} \right)}\right)\end{split}
\end{equation*}
\begin{sphinxVerbatim}[commandchars=\\\{\}]
\PYG{n}{diff}\PYG{p}{(}\PYG{n}{sin}\PYG{p}{(}\PYG{n}{x}\PYG{o}{*}\PYG{o}{*}\PYG{l+m+mi}{2}\PYG{p}{)}\PYG{p}{,}\PYG{n}{x}\PYG{p}{,} \PYG{l+m+mi}{3}\PYG{p}{)} \PYG{c+c1}{\PYGZsh{}\PYGZsh{} Calcular a terceira derivada de outra forma}
\end{sphinxVerbatim}
\begin{equation*}
\begin{split}\displaystyle - 4 x \left(2 x^{2} \cos{\left(x^{2} \right)} + 3 \sin{\left(x^{2} \right)}\right)\end{split}
\end{equation*}
\begin{sphinxVerbatim}[commandchars=\\\{\}]
\PYG{n}{diff}\PYG{p}{(}\PYG{n}{sin}\PYG{p}{(}\PYG{n}{x}\PYG{o}{*}\PYG{o}{*}\PYG{l+m+mi}{2}\PYG{p}{)}\PYG{p}{,}\PYG{n}{x}\PYG{p}{,} \PYG{l+m+mi}{10}\PYG{p}{)} \PYG{c+c1}{\PYGZsh{}\PYGZsh{} Calcular a décima derivada}
\end{sphinxVerbatim}
\begin{equation*}
\begin{split}\displaystyle 32 \left(- 32 x^{10} \sin{\left(x^{2} \right)} + 720 x^{8} \cos{\left(x^{2} \right)} + 5040 x^{6} \sin{\left(x^{2} \right)} - 12600 x^{4} \cos{\left(x^{2} \right)} - 9450 x^{2} \sin{\left(x^{2} \right)} + 945 \cos{\left(x^{2} \right)}\right)\end{split}
\end{equation*}
\begin{sphinxVerbatim}[commandchars=\\\{\}]
\PYG{n}{my\PYGZus{}deriv} \PYG{o}{=} \PYG{n}{Derivative}\PYG{p}{(}\PYG{n}{exp}\PYG{p}{(}\PYG{l+m+mi}{2}\PYG{o}{*}\PYG{n}{x}\PYG{o}{*}\PYG{o}{*}\PYG{l+m+mi}{3}\PYG{p}{)}\PYG{p}{,}\PYG{n}{x}\PYG{p}{)}
\PYG{n}{my\PYGZus{}deriv}\PYG{o}{.}\PYG{n}{doit}\PYG{p}{(}\PYG{p}{)}
\end{sphinxVerbatim}
\begin{equation*}
\begin{split}\displaystyle 6 x^{2} e^{2 x^{3}}\end{split}
\end{equation*}
\begin{sphinxVerbatim}[commandchars=\\\{\}]
\PYG{c+c1}{\PYGZsh{}\PYGZsh{} Podemos, com o método diff()}
\PYG{n}{expr} \PYG{o}{=} \PYG{n}{exp}\PYG{p}{(}\PYG{l+m+mi}{2}\PYG{o}{*}\PYG{n}{x}\PYG{o}{*}\PYG{o}{*}\PYG{l+m+mi}{3}\PYG{p}{)}
\PYG{n}{expr}\PYG{o}{.}\PYG{n}{diff}\PYG{p}{(}\PYG{n}{x}\PYG{p}{)}
\end{sphinxVerbatim}
\begin{equation*}
\begin{split}\displaystyle 6 x^{2} e^{2 x^{3}}\end{split}
\end{equation*}
\begin{sphinxVerbatim}[commandchars=\\\{\}]
\PYG{n}{expr}\PYG{o}{.}\PYG{n}{diff}\PYG{p}{(}\PYG{n}{x}\PYG{p}{,}\PYG{l+m+mi}{3}\PYG{p}{)} \PYG{c+c1}{\PYGZsh{} Terceira derivada}
\end{sphinxVerbatim}
\begin{equation*}
\begin{split}\displaystyle 12 \left(18 x^{6} + 18 x^{3} + 1\right) e^{2 x^{3}}\end{split}
\end{equation*}

\section{Integrais}
\label{\detokenize{chapters/4:integrais}}
\sphinxAtStartPar
Assim como os Limites e as Derivadas que vimos acima, podemos criar uma Integral através da classe \sphinxcode{\sphinxupquote{Integral()}} caso queiramos ter somente a expressão, e caso queiramos o resultado de uma Integral, basta utilizar a função \sphinxcode{\sphinxupquote{integrate()}}.

\sphinxAtStartPar
O \sphinxcode{\sphinxupquote{Sympy}} não acresce a constante de integração nas Integrais Indefinidas, então é importante se lembrar delaa quando for resolver algum exercício.

\begin{sphinxVerbatim}[commandchars=\\\{\}]
\PYG{n}{Integral}\PYG{p}{(}\PYG{l+m+mi}{1}\PYG{o}{/}\PYG{n}{x}\PYG{p}{,} \PYG{n}{x}\PYG{p}{)}
\end{sphinxVerbatim}
\begin{equation*}
\begin{split}\displaystyle \int \frac{1}{x}\, dx\end{split}
\end{equation*}
\begin{sphinxVerbatim}[commandchars=\\\{\}]
\PYG{n}{Integral}\PYG{p}{(}\PYG{l+m+mi}{1}\PYG{o}{/}\PYG{n}{x}\PYG{p}{,} \PYG{p}{(}\PYG{n}{x}\PYG{p}{,} \PYG{l+m+mi}{1}\PYG{p}{,} \PYG{l+m+mi}{10}\PYG{p}{)}\PYG{p}{)} \PYG{c+c1}{\PYGZsh{} Note que passamos (simbolo, inf, sup)}
\end{sphinxVerbatim}
\begin{equation*}
\begin{split}\displaystyle \int\limits_{1}^{10} \frac{1}{x}\, dx\end{split}
\end{equation*}
\begin{sphinxVerbatim}[commandchars=\\\{\}]
\PYG{n}{integrate}\PYG{p}{(}\PYG{l+m+mi}{1}\PYG{o}{/}\PYG{n}{x}\PYG{p}{,} \PYG{p}{(}\PYG{n}{x}\PYG{p}{,}\PYG{l+m+mi}{1}\PYG{p}{,}\PYG{l+m+mi}{10}\PYG{p}{)}\PYG{p}{)}
\end{sphinxVerbatim}
\begin{equation*}
\begin{split}\displaystyle \log{\left(10 \right)}\end{split}
\end{equation*}
\begin{sphinxVerbatim}[commandchars=\\\{\}]
\PYG{n}{my\PYGZus{}integral} \PYG{o}{=} \PYG{n}{Integral}\PYG{p}{(}\PYG{l+m+mi}{1}\PYG{o}{/}\PYG{n}{x} \PYG{o}{+} \PYG{l+m+mi}{1}\PYG{o}{/}\PYG{n}{y}\PYG{p}{,} \PYG{p}{(}\PYG{n}{x}\PYG{p}{,} \PYG{l+m+mi}{1}\PYG{p}{,} \PYG{l+m+mi}{10}\PYG{p}{)}\PYG{p}{,} \PYG{p}{(}\PYG{n}{y}\PYG{p}{,} \PYG{l+m+mi}{1}\PYG{p}{,} \PYG{l+m+mi}{10}\PYG{p}{)}\PYG{p}{)} \PYG{c+c1}{\PYGZsh{} Integral dupla, duas variáveis.}
\PYG{n}{my\PYGZus{}integral}
\end{sphinxVerbatim}
\begin{equation*}
\begin{split}\displaystyle \int\limits_{1}^{10}\int\limits_{1}^{10} \left(\frac{1}{y} + \frac{1}{x}\right)\, dx\, dy\end{split}
\end{equation*}
\begin{sphinxVerbatim}[commandchars=\\\{\}]
\PYG{n}{my\PYGZus{}integral}\PYG{o}{.}\PYG{n}{doit}\PYG{p}{(}\PYG{p}{)}
\end{sphinxVerbatim}
\begin{equation*}
\begin{split}\displaystyle 18 \log{\left(10 \right)}\end{split}
\end{equation*}
\begin{sphinxVerbatim}[commandchars=\\\{\}]
\PYG{n}{Integral}\PYG{p}{(}\PYG{n}{exp}\PYG{p}{(}\PYG{n}{x}\PYG{o}{*}\PYG{o}{*}\PYG{l+m+mi}{2} \PYG{o}{\PYGZhy{}} \PYG{l+m+mi}{10}\PYG{p}{)}\PYG{p}{,} \PYG{n}{x}\PYG{p}{,}\PYG{n}{x}\PYG{p}{)} \PYG{c+c1}{\PYGZsh{} Integral dupla indefinida, mesma variável}
\end{sphinxVerbatim}
\begin{equation*}
\begin{split}\displaystyle \iint e^{x^{2} - 10}\, dx\, dx\end{split}
\end{equation*}

\section{Outras funções}
\label{\detokenize{chapters/4:outras-funcoes}}

\subsection{Séries}
\label{\detokenize{chapters/4:series}}
\sphinxAtStartPar
Você pode utilizar o método \sphinxcode{\sphinxupquote{series()}} em uma expressão para fazer sua expansão em série.

\begin{sphinxVerbatim}[commandchars=\\\{\}]
\PYG{n}{asin}\PYG{p}{(}\PYG{n}{x}\PYG{p}{)}\PYG{o}{.}\PYG{n}{series}\PYG{p}{(}\PYG{n}{x}\PYG{p}{,}\PYG{l+m+mi}{0}\PYG{p}{,} \PYG{l+m+mi}{10}\PYG{p}{)} \PYG{c+c1}{\PYGZsh{} (x, x\PYGZus{}0, n)}
\end{sphinxVerbatim}
\begin{equation*}
\begin{split}\displaystyle x + \frac{x^{3}}{6} + \frac{3 x^{5}}{40} + \frac{5 x^{7}}{112} + \frac{35 x^{9}}{1152} + O\left(x^{10}\right)\end{split}
\end{equation*}

\subsection{Equações Diferenciais}
\label{\detokenize{chapters/4:equacoes-diferenciais}}
\sphinxAtStartPar
Ao criar uma função simbólica, você pode utilizar derivadas e a função \sphinxcode{\sphinxupquote{dsolve()}} para encontrar a solução de uma expressão e ou equação diferencial. Nesse caso, o \sphinxcode{\sphinxupquote{SymPy}} insere as constantes quando necessário.

\begin{sphinxVerbatim}[commandchars=\\\{\}]
\PYG{n}{f} \PYG{o}{=} \PYG{n}{Function}\PYG{p}{(}\PYG{l+s+s1}{\PYGZsq{}}\PYG{l+s+s1}{f}\PYG{l+s+s1}{\PYGZsq{}}\PYG{p}{)}
\PYG{n}{my\PYGZus{}deq} \PYG{o}{=} \PYG{n}{Eq}\PYG{p}{(}\PYG{n}{Derivative}\PYG{p}{(}\PYG{n}{f}\PYG{p}{(}\PYG{n}{x}\PYG{p}{)}\PYG{p}{,}\PYG{n}{x}\PYG{p}{,}\PYG{l+m+mi}{2}\PYG{p}{)}\PYG{p}{,}\PYG{n}{f}\PYG{p}{(}\PYG{n}{x}\PYG{p}{)}\PYG{p}{)}
\PYG{n}{my\PYGZus{}deq}
\end{sphinxVerbatim}
\begin{equation*}
\begin{split}\displaystyle \frac{d^{2}}{d x^{2}} f{\left(x \right)} = f{\left(x \right)}\end{split}
\end{equation*}
\begin{sphinxVerbatim}[commandchars=\\\{\}]
\PYG{n}{dsolve}\PYG{p}{(}\PYG{n}{my\PYGZus{}deq}\PYG{p}{)}
\end{sphinxVerbatim}
\begin{equation*}
\begin{split}\displaystyle f{\left(x \right)} = C_{1} e^{- x} + C_{2} e^{x}\end{split}
\end{equation*}

\section{Exercícios}
\label{\detokenize{chapters/4:exercicios}}
\sphinxAtStartPar
Como nos últimos capítulos, resolva os seguintes exercícios com o que aprendeu ao longo do curso.
\begin{enumerate}
\sphinxsetlistlabels{\arabic}{enumi}{enumii}{}{.}%
\item {} 
\sphinxAtStartPar
Para cada uma das funções abaixo, encontre:

\end{enumerate}

\sphinxAtStartPar
a) O domínio da função;

\sphinxAtStartPar
b) As assíntotas horizontais e verticais, caso existam;

\sphinxAtStartPar
c) Sua derivada, os intervalos de crescimento e decrescimento de f,os pontos de máximo e mínimo, caso existam;

\sphinxAtStartPar
d) Os intervalos onde o gráfico da f é côncavo para cima e onde é côncavo para baixo.
\begin{equation*}
\begin{split}f(x) = \dfrac{x^3}{x + 4}\end{split}
\end{equation*}\begin{equation*}
\begin{split}g(x) = \dfrac{x^3 - 4x^2 + 5}{x^2 - x}\end{split}
\end{equation*}\begin{equation*}
\begin{split}h(x) = x^2 - 4x + \dfrac{x}{x-10}\end{split}
\end{equation*}\begin{enumerate}
\sphinxsetlistlabels{\arabic}{enumi}{enumii}{}{.}%
\item {} 
\sphinxAtStartPar
Calcule:

\end{enumerate}
\begin{equation*}
\begin{split}\int x^3 - 2x^2 + 3x + 10 \,\ dx\end{split}
\end{equation*}\begin{equation*}
\begin{split}\int x^3\cdot\sin(2x) \,\ dx\end{split}
\end{equation*}\begin{equation*}
\begin{split}\int_0^{10} \tan^3(x)\sec^3(x) \,\ dx \end{split}
\end{equation*}\begin{equation*}
\begin{split}\int_1^{\infty} -\dfrac{1}{x^2} \,\ dx \end{split}
\end{equation*}\begin{enumerate}
\sphinxsetlistlabels{\arabic}{enumi}{enumii}{}{.}%
\item {} 
\sphinxAtStartPar
Qual a menor distância vertical entre as funções \(f(x) = 32x^2\) e \(g(x) = -\dfrac{8}{x^2}\)?

\end{enumerate}


\chapter{Criando Gráficos}
\label{\detokenize{chapters/5:criando-graficos}}\label{\detokenize{chapters/5::doc}}
\sphinxAtStartPar
O uso de gráficos na matemática, principalmente no Cálculo e na Geometria Analítica, é de extrema importância. Embora esse não seja o foco do \sphinxcode{\sphinxupquote{SymPy}}, é possível criar os chamados \sphinxstyleemphasis{plots} com certa facilidade. Inclusive, como é utilizado o \sphinxcode{\sphinxupquote{matplotlib}} para criar os gráficos, há uma grande margem para alterações. Isso permite gráficos quase totalmente customizáveis.

\sphinxAtStartPar
Os tipos de plots abordados nesse capítulos serão os criados pelas funções:
\begin{itemize}
\item {} 
\sphinxAtStartPar
plot()

\item {} 
\sphinxAtStartPar
plot\_implicit()

\item {} 
\sphinxAtStartPar
plot3d()

\end{itemize}

\sphinxAtStartPar
O seu uso é facilitado mais ainda pelo ambiente Jupyter, e podemos trabalhar com eles de forma semelhante ao que fora abordado no último capítulo. Ou seja, podemos criar um objeto para trabalharmos com ele, e depois ver o resultado. Ou, podemos simplesmente criar um plot.

\sphinxAtStartPar
Antes de começarmos, vamos a importação e as definições essenciais:

\begin{sphinxVerbatim}[commandchars=\\\{\}]
\PYG{k+kn}{from} \PYG{n+nn}{sympy} \PYG{k+kn}{import} \PYG{o}{*}
\PYG{n}{x}\PYG{p}{,} \PYG{n}{y}\PYG{p}{,} \PYG{n}{z} \PYG{o}{=} \PYG{n}{symbols}\PYG{p}{(}\PYG{l+s+s1}{\PYGZsq{}}\PYG{l+s+s1}{x y z}\PYG{l+s+s1}{\PYGZsq{}}\PYG{p}{)}
\PYG{n}{init\PYGZus{}printing}\PYG{p}{(}\PYG{n}{use\PYGZus{}unicode}\PYG{o}{=}\PYG{k+kc}{True}\PYG{p}{,} \PYG{n}{use\PYGZus{}latex}\PYG{o}{=}\PYG{l+s+s1}{\PYGZsq{}}\PYG{l+s+s1}{mathjax}\PYG{l+s+s1}{\PYGZsq{}}\PYG{p}{)}
\end{sphinxVerbatim}


\section{Plot 2D}
\label{\detokenize{chapters/5:plot-2d}}
\sphinxAtStartPar
As duas primeiras funções que trabalharemos criam plots em 2D. Para fazer um plot de uma função, basta utilizar a função \sphinxcode{\sphinxupquote{plot()}}. Confira abaixo:


\subsection{Funções}
\label{\detokenize{chapters/5:funcoes}}
\begin{sphinxVerbatim}[commandchars=\\\{\}]
\PYG{n}{plot}\PYG{p}{(}\PYG{n}{x}\PYG{o}{*}\PYG{o}{*}\PYG{l+m+mi}{3}\PYG{p}{)}
\end{sphinxVerbatim}

\noindent\sphinxincludegraphics{{5_3_0}.png}

\begin{sphinxVerbatim}[commandchars=\\\{\}]
\PYGZlt{}sympy.plotting.plot.Plot at 0x7f2220cd9710\PYGZgt{}
\end{sphinxVerbatim}

\sphinxAtStartPar
Podemos, inclusive, fazer vários plots unidos.

\begin{sphinxVerbatim}[commandchars=\\\{\}]
\PYG{n}{plot}\PYG{p}{(}\PYG{n}{x}\PYG{o}{*}\PYG{o}{*}\PYG{l+m+mi}{3}\PYG{p}{,} \PYG{n}{x}\PYG{o}{*}\PYG{o}{*}\PYG{l+m+mi}{2}\PYG{p}{)}
\end{sphinxVerbatim}

\noindent\sphinxincludegraphics{{5_5_0}.png}

\begin{sphinxVerbatim}[commandchars=\\\{\}]
\PYGZlt{}sympy.plotting.plot.Plot at 0x7f21f6b30ad0\PYGZgt{}
\end{sphinxVerbatim}

\sphinxAtStartPar
Obviamente, o gráfico acima está longe de ser o melhor possível. Mas com apenas uma função fizemos algo relevante. Mas, se armazenarmos, note o que podemos fazer.

\begin{sphinxVerbatim}[commandchars=\\\{\}]
\PYG{n}{my\PYGZus{}plot} \PYG{o}{=} \PYG{n}{plot}\PYG{p}{(}\PYG{n}{x}\PYG{o}{*}\PYG{o}{*}\PYG{l+m+mi}{3}\PYG{p}{,} \PYG{n}{x}\PYG{o}{*}\PYG{o}{*}\PYG{l+m+mi}{2}\PYG{p}{,} \PYG{n}{show}\PYG{o}{=}\PYG{k+kc}{False}\PYG{p}{)} \PYG{c+c1}{\PYGZsh{} show=False para não plotar.}
\PYG{n}{my\PYGZus{}plot}\PYG{o}{.}\PYG{n}{show}\PYG{p}{(}\PYG{p}{)} \PYG{c+c1}{\PYGZsh{} .show() para exibir}
\end{sphinxVerbatim}

\noindent\sphinxincludegraphics{{5_7_0}.png}

\sphinxAtStartPar
Por enquanto, nada mudou. Mas podemos ir trabalhando em seus atributos assim.

\begin{sphinxVerbatim}[commandchars=\\\{\}]
\PYG{n}{my\PYGZus{}plot}\PYG{p}{[}\PYG{l+m+mi}{1}\PYG{p}{]}\PYG{o}{.}\PYG{n}{line\PYGZus{}color} \PYG{o}{=} \PYG{l+s+s1}{\PYGZsq{}}\PYG{l+s+s1}{red}\PYG{l+s+s1}{\PYGZsq{}} \PYG{c+c1}{\PYGZsh{} x²}
\PYG{n}{my\PYGZus{}plot}\PYG{o}{.}\PYG{n}{show}\PYG{p}{(}\PYG{p}{)}
\end{sphinxVerbatim}

\noindent\sphinxincludegraphics{{5_9_0}.png}

\sphinxAtStartPar
Estamos melhorando. Veja que \sphinxcode{\sphinxupquote{my\_plot}} armazena as funções em sequência. Trabalhamos em \(x^2\) individualmente quando usamos \sphinxcode{\sphinxupquote{my\_plot{[}1{]}}}.

\begin{sphinxVerbatim}[commandchars=\\\{\}]
\PYG{n}{my\PYGZus{}plot}\PYG{o}{.}\PYG{n}{legend} \PYG{o}{=} \PYG{k+kc}{True} \PYG{c+c1}{\PYGZsh{} Legenda}
\PYG{n}{my\PYGZus{}plot}\PYG{o}{.}\PYG{n}{xlim} \PYG{o}{=} \PYG{p}{(}\PYG{o}{\PYGZhy{}}\PYG{l+m+mi}{10}\PYG{p}{,}\PYG{l+m+mi}{10}\PYG{p}{)} \PYG{c+c1}{\PYGZsh{} Esse é o máximo por padrão, veremos como alterar isso mais tarde}
\PYG{n}{my\PYGZus{}plot}\PYG{o}{.}\PYG{n}{ylim} \PYG{o}{=} \PYG{p}{(}\PYG{o}{\PYGZhy{}}\PYG{l+m+mi}{100}\PYG{p}{,}\PYG{l+m+mi}{100}\PYG{p}{)}
\PYG{n}{my\PYGZus{}plot}\PYG{o}{.}\PYG{n}{size} \PYG{o}{=} \PYG{p}{(}\PYG{l+m+mi}{8}\PYG{p}{,}\PYG{l+m+mi}{6}\PYG{p}{)} \PYG{c+c1}{\PYGZsh{} Tamanho da figura}
\PYG{n}{my\PYGZus{}plot}\PYG{o}{.}\PYG{n}{show}\PYG{p}{(}\PYG{p}{)}
\end{sphinxVerbatim}

\noindent\sphinxincludegraphics{{5_11_0}.png}

\sphinxAtStartPar
Acho que conseguimos um bom resultado. Contudo, você concorda que é um pouco cansativo acessar cada um desses valores e ir alterando, certo? E, se não definíssemos os limites em \(x\) para \([-10,10]\), você veria que a função não continuaria. Isso ocorre pois não definimos o alcance de nosso plot ao criá\sphinxhyphen{}lo.

\sphinxAtStartPar
Aliado a isso, é possível criar dois plots diferentes e uní\sphinxhyphen{}los com o método \sphinxcode{\sphinxupquote{.extend()}}.

\sphinxAtStartPar
Com o que aprendemos podemos fazer, com um único comando, um bom plot de uma única função. Caso queiramos mais de uma (como fizemos acima), basta alterar as cores de suas linhas.

\sphinxAtStartPar
Caso você conheça \sphinxcode{\sphinxupquote{matplotlib}}, é possível utilizar todos os parâmetros para anotações e etc. Não abordarei isso aqui pois acredito que foge do nosso objetivo, mas é algo interessante.

\begin{sphinxVerbatim}[commandchars=\\\{\}]
\PYG{n}{p1} \PYG{o}{=} \PYG{n}{plot}\PYG{p}{(}\PYG{n}{exp}\PYG{p}{(}\PYG{n}{x}\PYG{p}{)}\PYG{p}{,} \PYG{n}{log}\PYG{p}{(}\PYG{n}{x}\PYG{p}{)}\PYG{p}{,}\PYG{n}{x}\PYG{p}{,} \PYG{p}{(}\PYG{n}{x}\PYG{p}{,} \PYG{o}{\PYGZhy{}}\PYG{l+m+mi}{2}\PYG{p}{,} \PYG{l+m+mi}{10}\PYG{p}{)}\PYG{p}{,} \PYG{n}{ylim} \PYG{o}{=} \PYG{p}{(}\PYG{o}{\PYGZhy{}}\PYG{l+m+mi}{2}\PYG{p}{,}\PYG{l+m+mi}{10}\PYG{p}{)}\PYG{p}{,} \PYG{n}{legend} \PYG{o}{=} \PYG{k+kc}{True}\PYG{p}{,} \PYG{n}{size} \PYG{o}{=} \PYG{p}{(}\PYG{l+m+mi}{6}\PYG{p}{,}\PYG{l+m+mi}{6}\PYG{p}{)}\PYG{p}{,} \PYG{n}{show}\PYG{o}{=}\PYG{k+kc}{False}\PYG{p}{)}
\PYG{n}{p1}\PYG{p}{[}\PYG{l+m+mi}{1}\PYG{p}{]}\PYG{o}{.}\PYG{n}{line\PYGZus{}color} \PYG{o}{=} \PYG{l+s+s1}{\PYGZsq{}}\PYG{l+s+s1}{red}\PYG{l+s+s1}{\PYGZsq{}}
\PYG{n}{p1}\PYG{p}{[}\PYG{l+m+mi}{2}\PYG{p}{]}\PYG{o}{.}\PYG{n}{line\PYGZus{}color} \PYG{o}{=} \PYG{l+s+s1}{\PYGZsq{}}\PYG{l+s+s1}{green}\PYG{l+s+s1}{\PYGZsq{}}
\PYG{n}{p1}\PYG{o}{.}\PYG{n}{show}\PYG{p}{(}\PYG{p}{)}
\end{sphinxVerbatim}

\noindent\sphinxincludegraphics{{5_13_0}.png}


\subsection{Equações Implícitas}
\label{\detokenize{chapters/5:equacoes-implicitas}}
\sphinxAtStartPar
Quando temos uma equação com variável implícita, utilizamos a \sphinxcode{\sphinxupquote{plot\_implicit()}}. Ela segue a mesma lógica de \sphinxcode{\sphinxupquote{plot()}} em sua construção. Veja uma circunferência.

\begin{sphinxVerbatim}[commandchars=\\\{\}]
\PYG{n}{cir} \PYG{o}{=} \PYG{n}{Eq}\PYG{p}{(}\PYG{n}{x}\PYG{o}{*}\PYG{o}{*}\PYG{l+m+mi}{2} \PYG{o}{+} \PYG{n}{y}\PYG{o}{*}\PYG{o}{*}\PYG{l+m+mi}{2}\PYG{p}{,} \PYG{l+m+mi}{16}\PYG{p}{)} \PYG{c+c1}{\PYGZsh{} R = 2}
\PYG{n}{plot\PYGZus{}implicit}\PYG{p}{(}\PYG{n}{cir}\PYG{p}{)}
\end{sphinxVerbatim}

\noindent\sphinxincludegraphics{{5_15_0}.png}

\begin{sphinxVerbatim}[commandchars=\\\{\}]
\PYGZlt{}sympy.plotting.plot.Plot at 0x7f21f4425ad0\PYGZgt{}
\end{sphinxVerbatim}

\sphinxAtStartPar
Também podemos elaborá\sphinxhyphen{}lo.

\begin{sphinxVerbatim}[commandchars=\\\{\}]
\PYG{n}{plot\PYGZus{}implicit}\PYG{p}{(}\PYG{n}{Eq}\PYG{p}{(}\PYG{n}{x}\PYG{o}{*}\PYG{o}{*}\PYG{l+m+mi}{2}\PYG{o}{/}\PYG{l+m+mi}{9} \PYG{o}{+} \PYG{n}{y}\PYG{o}{*}\PYG{o}{*}\PYG{l+m+mi}{2}\PYG{o}{/}\PYG{l+m+mi}{4}\PYG{p}{,} \PYG{l+m+mi}{1}\PYG{p}{)}\PYG{p}{,} \PYG{p}{(}\PYG{n}{x}\PYG{p}{,} \PYG{o}{\PYGZhy{}}\PYG{l+m+mi}{4}\PYG{p}{,} \PYG{l+m+mi}{4}\PYG{p}{)}\PYG{p}{,} \PYG{n}{ylim} \PYG{o}{=} \PYG{p}{(}\PYG{o}{\PYGZhy{}}\PYG{l+m+mi}{4}\PYG{p}{,}\PYG{l+m+mi}{4}\PYG{p}{)}\PYG{p}{,} \PYG{n}{size} \PYG{o}{=} \PYG{p}{(}\PYG{l+m+mi}{6}\PYG{p}{,}\PYG{l+m+mi}{6}\PYG{p}{)}\PYG{p}{)} \PYG{c+c1}{\PYGZsh{} Elípse}
\PYG{c+c1}{\PYGZsh{} axis = False, deixa sem os eixos}
\end{sphinxVerbatim}

\noindent\sphinxincludegraphics{{5_17_0}.png}

\begin{sphinxVerbatim}[commandchars=\\\{\}]
\PYGZlt{}sympy.plotting.plot.Plot at 0x7f21f4553b50\PYGZgt{}
\end{sphinxVerbatim}


\section{Plot 3D}
\label{\detokenize{chapters/5:plot-3d}}
\sphinxAtStartPar
Utilizando funções de duas variáveis, nós conseguimos fazer plots em 3d com a função \sphinxcode{\sphinxupquote{plot3d()}}.

\begin{sphinxVerbatim}[commandchars=\\\{\}]
\PYG{k+kn}{from} \PYG{n+nn}{sympy}\PYG{n+nn}{.}\PYG{n+nn}{plotting} \PYG{k+kn}{import} \PYG{n}{plot3d} \PYG{c+c1}{\PYGZsh{} Caso queira importá\PYGZhy{}la diretamente}
\PYG{n}{plot3d}\PYG{p}{(}\PYG{n}{x}\PYG{o}{*}\PYG{o}{*}\PYG{l+m+mi}{2} \PYG{o}{+} \PYG{n}{y}\PYG{o}{*}\PYG{o}{*}\PYG{l+m+mi}{2}\PYG{p}{)}
\end{sphinxVerbatim}

\noindent\sphinxincludegraphics{{5_19_0}.png}

\begin{sphinxVerbatim}[commandchars=\\\{\}]
\PYGZlt{}sympy.plotting.plot.Plot at 0x7f21f43d7510\PYGZgt{}
\end{sphinxVerbatim}

\sphinxAtStartPar
Os parâmetros são muito parecidos com a função \sphinxcode{\sphinxupquote{plot()}}

\begin{sphinxVerbatim}[commandchars=\\\{\}]
\PYG{n}{plot3d}\PYG{p}{(}\PYG{l+m+mi}{100}\PYG{o}{*}\PYG{p}{(}\PYG{n}{sin}\PYG{p}{(}\PYG{n}{x}\PYG{p}{)} \PYG{o}{+} \PYG{n}{cos}\PYG{p}{(}\PYG{n}{y}\PYG{p}{)}\PYG{p}{)}\PYG{p}{,} \PYG{n}{size} \PYG{o}{=} \PYG{p}{(}\PYG{l+m+mi}{6}\PYG{p}{,}\PYG{l+m+mi}{6}\PYG{p}{)}\PYG{p}{)}
\end{sphinxVerbatim}

\noindent\sphinxincludegraphics{{5_21_0}.png}

\begin{sphinxVerbatim}[commandchars=\\\{\}]
\PYGZlt{}sympy.plotting.plot.Plot at 0x7f21f41bd8d0\PYGZgt{}
\end{sphinxVerbatim}

\sphinxAtStartPar
Além desses plots, há os plots paramétricos. Recomendo que dê uma olhada na documentação. No mais, é realmente simples criar plots no Sympy.


\section{Exercícios}
\label{\detokenize{chapters/5:exercicios}}\begin{enumerate}
\sphinxsetlistlabels{\arabic}{enumi}{enumii}{}{.}%
\item {} 
\sphinxAtStartPar
Faça o plot das seguintes funções, escolhendo os melhores valores para os parâmetros:

\end{enumerate}
\begin{equation*}
\begin{split}f(x) = 4x^2 - 3x + 26\end{split}
\end{equation*}\begin{equation*}
\begin{split}g(x) = \log(x^2 + 10)\end{split}
\end{equation*}\begin{equation*}
\begin{split}h(x) = 10x^4 + 7x^3 - 10x + 20\end{split}
\end{equation*}\begin{equation*}
\begin{split}p(x) = \sin(x^2 - x\cdot \pi) + \cos\left(x + \dfrac{\pi}{6}\right)\end{split}
\end{equation*}\begin{enumerate}
\sphinxsetlistlabels{\arabic}{enumi}{enumii}{}{.}%
\item {} 
\sphinxAtStartPar
Faça o plot das seguintes equações, escolhendo os melhores valores para os parâmetros:

\end{enumerate}
\begin{equation*}
\begin{split} 2x - 5y = 20\end{split}
\end{equation*}\begin{equation*}
\begin{split}x^2 + y^2 = 60\end{split}
\end{equation*}\begin{equation*}
\begin{split}\dfrac{x^2}{10} - \dfrac{y^2}{8} = 1\end{split}
\end{equation*}\begin{equation*}
\begin{split}2x - 5y^2 = 10\end{split}
\end{equation*}

\chapter{Extras}
\label{\detokenize{chapters/6:extras}}\label{\detokenize{chapters/6::doc}}
\sphinxAtStartPar
Como a principal intenção do curso é prepará\sphinxhyphen{}los para o uso de \sphinxcode{\sphinxupquote{SymPy}} nas disciplinas que envolvem Cálculo, ao finalizar o capítulo anterior você deve estar pronto para resolver seus problemas utilizando esse módulo. Contudo, eu acredito que há muito a se falar sobre esse módulo. E, portanto, esse capítulo fará uma abordagem rápida sobre algumas coisas que são possíveis com ele.


\section{Geometria}
\label{\detokenize{chapters/6:geometria}}
\sphinxAtStartPar
Isso mesmo, nós podemos resolver problemas de Geometria tanto de forma simbólica, como de forma numérica. A ideia principal não é ficar criando plots com o sistema completo, mas sim trabalhar matemáticamente (indo para o lado da Geometria Analítica).

\sphinxAtStartPar
Antes de começarmos essa seção e as próximas, faremos as devidas importações e definições:

\begin{sphinxVerbatim}[commandchars=\\\{\}]
\PYG{k+kn}{from} \PYG{n+nn}{sympy} \PYG{k+kn}{import} \PYG{o}{*}
\PYG{k+kn}{from} \PYG{n+nn}{sympy}\PYG{n+nn}{.}\PYG{n+nn}{geometry} \PYG{k+kn}{import} \PYG{o}{*} \PYG{c+c1}{\PYGZsh{} Importante garantir que foi importado corretamente}
\PYG{n}{x}\PYG{p}{,} \PYG{n}{y}\PYG{p}{,} \PYG{n}{z} \PYG{o}{=} \PYG{n}{symbols}\PYG{p}{(}\PYG{l+s+s1}{\PYGZsq{}}\PYG{l+s+s1}{x y z}\PYG{l+s+s1}{\PYGZsq{}}\PYG{p}{)}
\PYG{n}{init\PYGZus{}printing}\PYG{p}{(}\PYG{n}{use\PYGZus{}unicode}\PYG{o}{=}\PYG{k+kc}{True}\PYG{p}{,} \PYG{n}{use\PYGZus{}latex}\PYG{o}{=}\PYG{l+s+s1}{\PYGZsq{}}\PYG{l+s+s1}{mathjax}\PYG{l+s+s1}{\PYGZsq{}}\PYG{p}{)}
\end{sphinxVerbatim}


\subsection{2D}
\label{\detokenize{chapters/6:d}}
\sphinxAtStartPar
Começando pela geometria em 2D, podemos seguir o processo de criar os pontos, as linhas (a partir dos pontos) e as formas 2D a partir dos segmentos. É bem simples e intuitivo, veja:

\begin{sphinxVerbatim}[commandchars=\\\{\}]
\PYG{n}{O} \PYG{o}{=} \PYG{n}{Point}\PYG{p}{(}\PYG{l+m+mi}{0}\PYG{p}{,}\PYG{l+m+mi}{0}\PYG{p}{)}
\PYG{n}{A} \PYG{o}{=} \PYG{n}{Point}\PYG{p}{(}\PYG{l+m+mi}{1}\PYG{p}{,}\PYG{l+m+mi}{2}\PYG{p}{)}
\PYG{n}{B} \PYG{o}{=} \PYG{n}{Point}\PYG{p}{(}\PYG{l+m+mi}{3}\PYG{p}{,}\PYG{o}{\PYGZhy{}}\PYG{l+m+mi}{4}\PYG{p}{)}
\PYG{n}{C} \PYG{o}{=} \PYG{n}{Point}\PYG{p}{(}\PYG{o}{\PYGZhy{}}\PYG{l+m+mi}{2}\PYG{p}{,} \PYG{l+m+mi}{3}\PYG{p}{)}
\end{sphinxVerbatim}

\sphinxAtStartPar
Você pode fazer as operações padrões entre pontos normalmente.

\begin{sphinxVerbatim}[commandchars=\\\{\}]
\PYG{n}{B} \PYG{o}{\PYGZhy{}} \PYG{n}{A} \PYG{c+c1}{\PYGZsh{}\PYGZsh{} AB}
\end{sphinxVerbatim}
\begin{equation*}
\begin{split}\displaystyle Point2D\left(2, -6\right)\end{split}
\end{equation*}
\sphinxAtStartPar
Contudo, o indicado é utilizar as classes, que já terão suas propriedades a fácil acesso.

\begin{sphinxVerbatim}[commandchars=\\\{\}]
\PYG{n}{Segment}\PYG{p}{(}\PYG{n}{A}\PYG{p}{,}\PYG{n}{B}\PYG{p}{)} \PYG{c+c1}{\PYGZsh{}\PYGZsh{} AB Simbólicamente}
\end{sphinxVerbatim}
\begin{equation*}
\begin{split}\displaystyle Segment2D\left(Point2D\left(1, 2\right), Point2D\left(3, -4\right)\right)\end{split}
\end{equation*}
\begin{sphinxVerbatim}[commandchars=\\\{\}]
\PYG{n}{AC} \PYG{o}{=} \PYG{n}{Segment}\PYG{p}{(}\PYG{n}{A}\PYG{p}{,}\PYG{n}{C}\PYG{p}{)}
\PYG{n}{AC}\PYG{o}{.}\PYG{n}{slope} \PYG{c+c1}{\PYGZsh{}\PYGZsh{} inclinação}
\end{sphinxVerbatim}
\begin{equation*}
\begin{split}\displaystyle - \frac{1}{3}\end{split}
\end{equation*}
\begin{sphinxVerbatim}[commandchars=\\\{\}]
\PYG{n}{AC}\PYG{o}{.}\PYG{n}{length} \PYG{c+c1}{\PYGZsh{}\PYGZsh{} comprimento}
\end{sphinxVerbatim}
\begin{equation*}
\begin{split}\displaystyle \sqrt{10}\end{split}
\end{equation*}
\begin{sphinxVerbatim}[commandchars=\\\{\}]
\PYG{n}{AC}\PYG{o}{.}\PYG{n}{midpoint} \PYG{c+c1}{\PYGZsh{}\PYGZsh{} ponto médio}
\end{sphinxVerbatim}
\begin{equation*}
\begin{split}\displaystyle Point2D\left(- \frac{1}{2}, \frac{5}{2}\right)\end{split}
\end{equation*}
\begin{sphinxVerbatim}[commandchars=\\\{\}]
\PYG{n}{AC}\PYG{o}{.}\PYG{n}{contains}\PYG{p}{(}\PYG{n}{A}\PYG{p}{)} \PYG{c+c1}{\PYGZsh{}\PYGZsh{} Contém A?}
\end{sphinxVerbatim}
\begin{equation*}
\begin{split}\displaystyle \text{True}\end{split}
\end{equation*}
\begin{sphinxVerbatim}[commandchars=\\\{\}]
\PYG{n}{AC}\PYG{o}{.}\PYG{n}{distance}\PYG{p}{(}\PYG{n}{B}\PYG{p}{)} \PYG{c+c1}{\PYGZsh{}\PYGZsh{} Menor distância ao ponto B}
\end{sphinxVerbatim}
\begin{equation*}
\begin{split}\displaystyle 2 \sqrt{10}\end{split}
\end{equation*}
\sphinxAtStartPar
Nós podemos criar linhas também

\begin{sphinxVerbatim}[commandchars=\\\{\}]
\PYG{n}{Line}\PYG{p}{(}\PYG{n}{A}\PYG{p}{,}\PYG{n}{B}\PYG{p}{)}
\end{sphinxVerbatim}
\begin{equation*}
\begin{split}\displaystyle Line2D\left(Point2D\left(1, 2\right), Point2D\left(3, -4\right)\right)\end{split}
\end{equation*}
\begin{sphinxVerbatim}[commandchars=\\\{\}]
\PYG{n}{l1} \PYG{o}{=} \PYG{n}{Line}\PYG{p}{(}\PYG{n}{A}\PYG{p}{,}\PYG{n}{B}\PYG{p}{)}
\PYG{n}{l1}\PYG{o}{.}\PYG{n}{equation}\PYG{p}{(}\PYG{p}{)} \PYG{c+c1}{\PYGZsh{}\PYGZsh{} Equação da reta = 0}
\end{sphinxVerbatim}
\begin{equation*}
\begin{split}\displaystyle 6 x + 2 y - 10\end{split}
\end{equation*}
\begin{sphinxVerbatim}[commandchars=\\\{\}]
\PYG{n}{l1}\PYG{o}{.}\PYG{n}{coefficients} \PYG{c+c1}{\PYGZsh{}\PYGZsh{} Coeficientes da reta}
\end{sphinxVerbatim}
\begin{equation*}
\begin{split}\displaystyle \left( 6, \  2, \  -10\right)\end{split}
\end{equation*}
\sphinxAtStartPar
Podemos criar uma reta ao dar um ponto inicial e uma inclinação, lembrando que: \(y - y_0 = m(x-x_0)\)

\begin{sphinxVerbatim}[commandchars=\\\{\}]
\PYG{n}{l2} \PYG{o}{=} \PYG{n}{Line}\PYG{p}{(}\PYG{n}{C}\PYG{p}{,} \PYG{n}{slope} \PYG{o}{=} \PYG{l+m+mi}{3}\PYG{p}{)}
\PYG{n}{l2}\PYG{o}{.}\PYG{n}{equation}\PYG{p}{(}\PYG{p}{)}
\end{sphinxVerbatim}
\begin{equation*}
\begin{split}\displaystyle - 3 x + y - 9\end{split}
\end{equation*}
\begin{sphinxVerbatim}[commandchars=\\\{\}]
\PYG{n}{l3} \PYG{o}{=} \PYG{n}{l2}\PYG{o}{.}\PYG{n}{perpendicular\PYGZus{}line}\PYG{p}{(}\PYG{n}{A}\PYG{p}{)} \PYG{c+c1}{\PYGZsh{}\PYGZsh{} Retorna uma reta perpendicular que passa pelo ponto dado}
\PYG{n}{l3}\PYG{o}{.}\PYG{n}{equation}\PYG{p}{(}\PYG{p}{)}
\end{sphinxVerbatim}
\begin{equation*}
\begin{split}\displaystyle - x - 3 y + 7\end{split}
\end{equation*}
\begin{sphinxVerbatim}[commandchars=\\\{\}]
\PYG{n}{l3}\PYG{o}{.}\PYG{n}{slope} \PYG{c+c1}{\PYGZsh{} \PYGZhy{}m\PYGZca{}\PYGZhy{}1}
\end{sphinxVerbatim}
\begin{equation*}
\begin{split}\displaystyle - \frac{1}{3}\end{split}
\end{equation*}
\sphinxAtStartPar
E nós podemos ver a intersecção entre duas entidades geométricas.

\begin{sphinxVerbatim}[commandchars=\\\{\}]
\PYG{n}{intersection}\PYG{p}{(}\PYG{n}{l2}\PYG{p}{,}\PYG{n}{l3}\PYG{p}{)}
\end{sphinxVerbatim}
\begin{equation*}
\begin{split}\displaystyle \left[ Point2D\left(-2, 3\right)\right]\end{split}
\end{equation*}
\begin{sphinxVerbatim}[commandchars=\\\{\}]
\PYG{n}{intersection}\PYG{p}{(}\PYG{n}{l1}\PYG{p}{,} \PYG{n}{l3}\PYG{p}{)} \PYG{c+c1}{\PYGZsh{} Ponto A}
\end{sphinxVerbatim}
\begin{equation*}
\begin{split}\displaystyle \left[ Point2D\left(1, 2\right)\right]\end{split}
\end{equation*}
\sphinxAtStartPar
Para plotar, de modo geral, fazemos o uso do que aprendemos no último capítulo, a função \sphinxcode{\sphinxupquote{plot\_implicit}}.

\begin{sphinxVerbatim}[commandchars=\\\{\}]
\PYG{n}{plot\PYGZus{}implicit}\PYG{p}{(}\PYG{n}{l1}\PYG{o}{.}\PYG{n}{equation}\PYG{p}{(}\PYG{p}{)}\PYG{p}{)} 
\end{sphinxVerbatim}

\noindent\sphinxincludegraphics{{6_25_0}.png}

\begin{sphinxVerbatim}[commandchars=\\\{\}]
\PYGZlt{}sympy.plotting.plot.Plot at 0x7f9593fee1d0\PYGZgt{}
\end{sphinxVerbatim}

\sphinxAtStartPar
Podemos criar figuras geométricas e encontrar suas áreas e verficiar intersecções. Veja os exemplos:

\begin{sphinxVerbatim}[commandchars=\\\{\}]
\PYG{n}{trig} \PYG{o}{=} \PYG{n}{Triangle}\PYG{p}{(}\PYG{n}{A}\PYG{p}{,}\PYG{n}{B}\PYG{p}{,}\PYG{n}{C}\PYG{p}{)} \PYG{c+c1}{\PYGZsh{} Cria um Triângulo}
\PYG{n}{trig}
\end{sphinxVerbatim}
\begin{equation*}
\begin{split}\displaystyle Triangle\left(Point2D\left(1, 2\right), Point2D\left(3, -4\right), Point2D\left(-2, 3\right)\right)\end{split}
\end{equation*}
\begin{sphinxVerbatim}[commandchars=\\\{\}]
\PYG{n}{trig}\PYG{o}{.}\PYG{n}{area} \PYG{c+c1}{\PYGZsh{}\PYGZsh{} Não utiliza valores absolutos}
\end{sphinxVerbatim}
\begin{equation*}
\begin{split}\displaystyle -8\end{split}
\end{equation*}
\begin{sphinxVerbatim}[commandchars=\\\{\}]
\PYG{n+nb}{abs}\PYG{p}{(}\PYG{n}{trig}\PYG{o}{.}\PYG{n}{area}\PYG{p}{)} \PYG{c+c1}{\PYGZsh{}\PYGZsh{} Correto}
\end{sphinxVerbatim}
\begin{equation*}
\begin{split}\displaystyle 8\end{split}
\end{equation*}
\begin{sphinxVerbatim}[commandchars=\\\{\}]
\PYG{n}{trig}\PYG{o}{.}\PYG{n}{perimeter} \PYG{c+c1}{\PYGZsh{}\PYGZsh{} Perímetro}
\end{sphinxVerbatim}
\begin{equation*}
\begin{split}\displaystyle \sqrt{74} + 3 \sqrt{10}\end{split}
\end{equation*}
\begin{sphinxVerbatim}[commandchars=\\\{\}]
\PYG{n}{trig}\PYG{o}{.}\PYG{n}{orthocenter} \PYG{c+c1}{\PYGZsh{}\PYGZsh{} Centro Ortogonal}
\end{sphinxVerbatim}
\begin{equation*}
\begin{split}\displaystyle Point2D\left(\frac{25}{4}, \frac{23}{4}\right)\end{split}
\end{equation*}
\begin{sphinxVerbatim}[commandchars=\\\{\}]
\PYG{n}{trig}\PYG{o}{.}\PYG{n}{circumcenter} \PYG{c+c1}{\PYGZsh{}\PYGZsh{} Circuncentro}
\end{sphinxVerbatim}
\begin{equation*}
\begin{split}\displaystyle Point2D\left(- \frac{17}{8}, - \frac{19}{8}\right)\end{split}
\end{equation*}
\begin{sphinxVerbatim}[commandchars=\\\{\}]
\PYG{n}{trig}\PYG{o}{.}\PYG{n}{altitudes} \PYG{c+c1}{\PYGZsh{}\PYGZsh{} Alturas}
\end{sphinxVerbatim}
\begin{equation*}
\begin{split}\displaystyle \left\{ Point2D\left(-2, 3\right) : Segment2D\left(Point2D\left(-2, 3\right), Point2D\left(\frac{2}{5}, \frac{19}{5}\right)\right), \  Point2D\left(1, 2\right) : Segment2D\left(Point2D\left(1, 2\right), Point2D\left(- \frac{19}{37}, \frac{34}{37}\right)\right), \  Point2D\left(3, -4\right) : Segment2D\left(Point2D\left(3, -4\right), Point2D\left(\frac{23}{5}, \frac{4}{5}\right)\right)\right\}\end{split}
\end{equation*}
\begin{sphinxVerbatim}[commandchars=\\\{\}]
\PYG{n}{trig}\PYG{o}{.}\PYG{n}{incircle} \PYG{c+c1}{\PYGZsh{}\PYGZsh{} Círculo interno}
\end{sphinxVerbatim}
\begin{equation*}
\begin{split}\displaystyle Circle\left(Point2D\left(\frac{- \sqrt{10} + \sqrt{74}}{\sqrt{74} + 3 \sqrt{10}}, \frac{2 \left(\sqrt{10} + \sqrt{74}\right)}{\sqrt{74} + 3 \sqrt{10}}\right), - \frac{16}{\sqrt{74} + 3 \sqrt{10}}\right)\end{split}
\end{equation*}
\begin{sphinxVerbatim}[commandchars=\\\{\}]
\PYG{n}{trig}\PYG{o}{.}\PYG{n}{incircle}\PYG{o}{.}\PYG{n}{equation}\PYG{p}{(}\PYG{p}{)}
\end{sphinxVerbatim}
\begin{equation*}
\begin{split}\displaystyle \left(x - \frac{- \sqrt{10} + \sqrt{74}}{\sqrt{74} + 3 \sqrt{10}}\right)^{2} + \left(y - \frac{2 \left(\sqrt{10} + \sqrt{74}\right)}{\sqrt{74} + 3 \sqrt{10}}\right)^{2} - \frac{256}{\left(\sqrt{74} + 3 \sqrt{10}\right)^{2}}\end{split}
\end{equation*}
\begin{sphinxVerbatim}[commandchars=\\\{\}]
\PYG{n}{trig}\PYG{o}{.}\PYG{n}{bisectors}\PYG{p}{(}\PYG{p}{)} \PYG{c+c1}{\PYGZsh{}\PYGZsh{} Bissetrizes}
\end{sphinxVerbatim}
\begin{equation*}
\begin{split}\displaystyle \left\{ Point2D\left(-2, 3\right) : Segment2D\left(Point2D\left(-2, 3\right), Point2D\left(\frac{11}{16} + \frac{\sqrt{185}}{16}, \frac{47}{16} - \frac{3 \sqrt{185}}{16}\right)\right), \  Point2D\left(1, 2\right) : Segment2D\left(Point2D\left(1, 2\right), Point2D\left(- \frac{1}{3}, \frac{2}{3}\right)\right), \  Point2D\left(3, -4\right) : Segment2D\left(Point2D\left(3, -4\right), Point2D\left(\frac{77}{17} - \frac{6 \sqrt{185}}{17}, \frac{14}{17} + \frac{2 \sqrt{185}}{17}\right)\right)\right\}\end{split}
\end{equation*}
\begin{sphinxVerbatim}[commandchars=\\\{\}]
\PYG{n}{trig}\PYG{o}{.}\PYG{n}{bisectors}\PYG{p}{(}\PYG{p}{)}\PYG{p}{[}\PYG{n}{A}\PYG{p}{]} \PYG{c+c1}{\PYGZsh{}\PYGZsh{} Bissetriz que passa no ponto A}
\end{sphinxVerbatim}
\begin{equation*}
\begin{split}\displaystyle Segment2D\left(Point2D\left(1, 2\right), Point2D\left(- \frac{1}{3}, \frac{2}{3}\right)\right)\end{split}
\end{equation*}
\begin{sphinxVerbatim}[commandchars=\\\{\}]
\PYG{n}{trig}\PYG{o}{.}\PYG{n}{is\PYGZus{}right}\PYG{p}{(}\PYG{p}{)} \PYG{c+c1}{\PYGZsh{}\PYGZsh{} É triângulo retângulo?}
\end{sphinxVerbatim}

\begin{sphinxVerbatim}[commandchars=\\\{\}]
False
\end{sphinxVerbatim}

\begin{sphinxVerbatim}[commandchars=\\\{\}]
\PYG{n}{trig}\PYG{o}{.}\PYG{n}{is\PYGZus{}scalene}\PYG{p}{(}\PYG{p}{)} \PYG{c+c1}{\PYGZsh{}\PYGZsh{} É triângulo escaleno?}
\end{sphinxVerbatim}

\begin{sphinxVerbatim}[commandchars=\\\{\}]
True
\end{sphinxVerbatim}

\begin{sphinxVerbatim}[commandchars=\\\{\}]
\PYG{n}{circ} \PYG{o}{=} \PYG{n}{Circle}\PYG{p}{(}\PYG{n}{A}\PYG{p}{,} \PYG{l+m+mi}{3}\PYG{p}{)} \PYG{c+c1}{\PYGZsh{}\PYGZsh{} Centro e Raio}
\PYG{n}{circ}
\end{sphinxVerbatim}
\begin{equation*}
\begin{split}\displaystyle Circle\left(Point2D\left(1, 2\right), 3\right)\end{split}
\end{equation*}
\begin{sphinxVerbatim}[commandchars=\\\{\}]
\PYG{n}{circ}\PYG{o}{.}\PYG{n}{equation}\PYG{p}{(}\PYG{p}{)}
\end{sphinxVerbatim}
\begin{equation*}
\begin{split}\displaystyle \left(x - 1\right)^{2} + \left(y - 2\right)^{2} - 9\end{split}
\end{equation*}
\begin{sphinxVerbatim}[commandchars=\\\{\}]
\PYG{n}{circ}\PYG{o}{.}\PYG{n}{circumference}
\end{sphinxVerbatim}
\begin{equation*}
\begin{split}\displaystyle 6 \pi\end{split}
\end{equation*}
\begin{sphinxVerbatim}[commandchars=\\\{\}]
\PYG{n}{circ}\PYG{o}{.}\PYG{n}{area}
\end{sphinxVerbatim}
\begin{equation*}
\begin{split}\displaystyle 9 \pi\end{split}
\end{equation*}
\begin{sphinxVerbatim}[commandchars=\\\{\}]
\PYG{n}{intersection}\PYG{p}{(}\PYG{n}{trig}\PYG{p}{,}\PYG{n}{circ}\PYG{p}{)}
\end{sphinxVerbatim}
\begin{equation*}
\begin{split}\displaystyle \left[ Point2D\left(- \frac{19}{37} + \frac{5 \sqrt{410}}{74}, \frac{34}{37} - \frac{7 \sqrt{410}}{74}\right), \  Point2D\left(1 - \frac{9 \sqrt{10}}{10}, \frac{3 \sqrt{10}}{10} + 2\right), \  Point2D\left(\frac{3 \sqrt{10}}{10} + 1, 2 - \frac{9 \sqrt{10}}{10}\right), \  Point2D\left(- \frac{5 \sqrt{410}}{74} - \frac{19}{37}, \frac{34}{37} + \frac{7 \sqrt{410}}{74}\right)\right]\end{split}
\end{equation*}
\begin{sphinxVerbatim}[commandchars=\\\{\}]
\PYG{n}{elips} \PYG{o}{=} \PYG{n}{Ellipse}\PYG{p}{(}\PYG{n}{B}\PYG{p}{,} \PYG{l+m+mi}{3}\PYG{p}{,} \PYG{l+m+mi}{2}\PYG{p}{)} \PYG{c+c1}{\PYGZsh{}\PYGZsh{} Centro, Raio Horizontal, Raio Vertical}
\PYG{n}{elips}
\end{sphinxVerbatim}
\begin{equation*}
\begin{split}\displaystyle Ellipse\left(Point2D\left(3, -4\right), 3, 2\right)\end{split}
\end{equation*}
\begin{sphinxVerbatim}[commandchars=\\\{\}]
\PYG{n}{elips}\PYG{o}{.}\PYG{n}{equation}\PYG{p}{(}\PYG{p}{)}
\end{sphinxVerbatim}
\begin{equation*}
\begin{split}\displaystyle \left(\frac{x}{3} - 1\right)^{2} + \left(\frac{y}{2} + 2\right)^{2} - 1\end{split}
\end{equation*}
\begin{sphinxVerbatim}[commandchars=\\\{\}]
\PYG{n}{elips}\PYG{o}{.}\PYG{n}{circumference} \PYG{c+c1}{\PYGZsh{}\PYGZsh{} Não há formulas}
\end{sphinxVerbatim}
\begin{equation*}
\begin{split}\displaystyle 12 E\left(\frac{5}{9}\right)\end{split}
\end{equation*}
\begin{sphinxVerbatim}[commandchars=\\\{\}]
\PYG{n}{elips}\PYG{o}{.}\PYG{n}{circumference}\PYG{o}{.}\PYG{n}{evalf}\PYG{p}{(}\PYG{p}{)} \PYG{c+c1}{\PYGZsh{}\PYGZsh{} Valor numérico}
\end{sphinxVerbatim}
\begin{equation*}
\begin{split}\displaystyle 15.8654395892906\end{split}
\end{equation*}
\begin{sphinxVerbatim}[commandchars=\\\{\}]
\PYG{n}{elips}\PYG{o}{.}\PYG{n}{area}
\end{sphinxVerbatim}
\begin{equation*}
\begin{split}\displaystyle 6 \pi\end{split}
\end{equation*}
\begin{sphinxVerbatim}[commandchars=\\\{\}]
\PYG{n}{elips}\PYG{o}{.}\PYG{n}{eccentricity}
\end{sphinxVerbatim}
\begin{equation*}
\begin{split}\displaystyle \frac{\sqrt{5}}{3}\end{split}
\end{equation*}
\begin{sphinxVerbatim}[commandchars=\\\{\}]
\PYG{n}{elips}\PYG{o}{.}\PYG{n}{foci} \PYG{c+c1}{\PYGZsh{}\PYGZsh{} Focos}
\end{sphinxVerbatim}
\begin{equation*}
\begin{split}\displaystyle \left( Point2D\left(3 - \sqrt{5}, -4\right), \  Point2D\left(\sqrt{5} + 3, -4\right)\right)\end{split}
\end{equation*}
\begin{sphinxVerbatim}[commandchars=\\\{\}]
\PYG{n}{elips}\PYG{o}{.}\PYG{n}{focus\PYGZus{}distance} \PYG{c+c1}{\PYGZsh{}\PYGZsh{} Distância Focal}
\end{sphinxVerbatim}
\begin{equation*}
\begin{split}\displaystyle \sqrt{5}\end{split}
\end{equation*}
\begin{sphinxVerbatim}[commandchars=\\\{\}]
\PYG{n}{D} \PYG{o}{=} \PYG{n}{Point}\PYG{p}{(}\PYG{l+m+mi}{0}\PYG{p}{,}\PYG{l+m+mi}{10}\PYG{p}{)}
\PYG{n}{quad} \PYG{o}{=} \PYG{n}{Polygon}\PYG{p}{(}\PYG{n}{A}\PYG{p}{,}\PYG{n}{B}\PYG{p}{,}\PYG{n}{C}\PYG{p}{,}\PYG{n}{D}\PYG{p}{)} \PYG{c+c1}{\PYGZsh{}\PYGZsh{} Criando Polígono de N vértices}
\PYG{n}{quad}
\end{sphinxVerbatim}
\begin{equation*}
\begin{split}\displaystyle Polygon\left(Point2D\left(1, 2\right), Point2D\left(3, -4\right), Point2D\left(-2, 3\right), Point2D\left(0, 10\right)\right)\end{split}
\end{equation*}
\begin{sphinxVerbatim}[commandchars=\\\{\}]
\PYG{n+nb}{abs}\PYG{p}{(}\PYG{n}{quad}\PYG{o}{.}\PYG{n}{area}\PYG{p}{)}
\end{sphinxVerbatim}
\begin{equation*}
\begin{split}\displaystyle \frac{39}{2}\end{split}
\end{equation*}
\begin{sphinxVerbatim}[commandchars=\\\{\}]
\PYG{n}{quad}\PYG{o}{.}\PYG{n}{angles}
\end{sphinxVerbatim}
\begin{equation*}
\begin{split}\displaystyle \left\{ Point2D\left(-2, 3\right) : - \operatorname{acos}{\left(- \frac{39 \sqrt{3922}}{3922} \right)} + 2 \pi, \  Point2D\left(0, 10\right) : - \operatorname{acos}{\left(\frac{54 \sqrt{3445}}{3445} \right)} + 2 \pi, \  Point2D\left(1, 2\right) : \operatorname{acos}{\left(- \frac{5 \sqrt{26}}{26} \right)}, \  Point2D\left(3, -4\right) : - \operatorname{acos}{\left(\frac{13 \sqrt{185}}{185} \right)} + 2 \pi\right\}\end{split}
\end{equation*}
\begin{sphinxVerbatim}[commandchars=\\\{\}]
\PYG{n}{quad}\PYG{o}{.}\PYG{n}{angles}\PYG{p}{[}\PYG{n}{A}\PYG{p}{]} \PYG{c+c1}{\PYGZsh{}\PYGZsh{} No ponto A}
\end{sphinxVerbatim}
\begin{equation*}
\begin{split}\displaystyle \operatorname{acos}{\left(- \frac{5 \sqrt{26}}{26} \right)}\end{split}
\end{equation*}
\begin{sphinxVerbatim}[commandchars=\\\{\}]
\PYG{k+kn}{from} \PYG{n+nn}{sympy}\PYG{n+nn}{.}\PYG{n+nn}{physics}\PYG{n+nn}{.}\PYG{n+nn}{units} \PYG{k+kn}{import} \PYG{n}{degree} \PYG{c+c1}{\PYGZsh{}\PYGZsh{} Importação das unidades}
\PYG{p}{(}\PYG{n}{quad}\PYG{o}{.}\PYG{n}{angles}\PYG{p}{[}\PYG{n}{A}\PYG{p}{]}\PYG{o}{/}\PYG{n}{degree}\PYG{o}{.}\PYG{n}{scale\PYGZus{}factor}\PYG{p}{)}\PYG{o}{.}\PYG{n}{evalf}\PYG{p}{(}\PYG{p}{)} \PYG{c+c1}{\PYGZsh{}\PYGZsh{} Transforma em Graus}
\end{sphinxVerbatim}
\begin{equation*}
\begin{split}\displaystyle 168.69006752598\end{split}
\end{equation*}
\begin{sphinxVerbatim}[commandchars=\\\{\}]
\PYG{n}{reg} \PYG{o}{=} \PYG{n}{RegularPolygon}\PYG{p}{(}\PYG{n}{A}\PYG{p}{,}\PYG{l+m+mi}{1}\PYG{p}{,}\PYG{l+m+mi}{4}\PYG{p}{)} \PYG{c+c1}{\PYGZsh{} Centro, Raio, Qtd. Lados}
\PYG{n}{reg}
\end{sphinxVerbatim}
\begin{equation*}
\begin{split}\displaystyle RegularPolygon\left(Point2D\left(1, 2\right), 1, 4, 0\right)\end{split}
\end{equation*}
\begin{sphinxVerbatim}[commandchars=\\\{\}]
\PYG{n}{reg}\PYG{o}{.}\PYG{n}{angles} \PYG{c+c1}{\PYGZsh{} Retângulo}
\end{sphinxVerbatim}
\begin{equation*}
\begin{split}\displaystyle \left\{ Point2D\left(0, 2\right) : \frac{\pi}{2}, \  Point2D\left(1, 1\right) : \frac{\pi}{2}, \  Point2D\left(1, 3\right) : \frac{\pi}{2}, \  Point2D\left(2, 2\right) : \frac{\pi}{2}\right\}\end{split}
\end{equation*}
\begin{sphinxVerbatim}[commandchars=\\\{\}]
\PYG{n}{reg}\PYG{o}{.}\PYG{n}{vertices} \PYG{c+c1}{\PYGZsh{} Vértices}
\end{sphinxVerbatim}
\begin{equation*}
\begin{split}\displaystyle \left[ Point2D\left(2, 2\right), \  Point2D\left(1, 3\right), \  Point2D\left(0, 2\right), \  Point2D\left(1, 1\right)\right]\end{split}
\end{equation*}
\sphinxAtStartPar
Para finalizar com a Geometria, é importante relembrar que é possível fazer tudo isso com valores simbólicos. Por exemplo, um quadrado em função de um lado \(x\):

\begin{sphinxVerbatim}[commandchars=\\\{\}]
\PYG{n}{sim\PYGZus{}quad} \PYG{o}{=} \PYG{n}{Polygon}\PYG{p}{(}\PYG{n}{Point}\PYG{p}{(}\PYG{n}{x}\PYG{o}{/}\PYG{l+m+mi}{2}\PYG{p}{,} \PYG{n}{x}\PYG{o}{/}\PYG{l+m+mi}{2}\PYG{p}{)}\PYG{p}{,} \PYG{n}{Point}\PYG{p}{(}\PYG{o}{\PYGZhy{}}\PYG{n}{x}\PYG{o}{/}\PYG{l+m+mi}{2}\PYG{p}{,} \PYG{n}{x}\PYG{o}{/}\PYG{l+m+mi}{2}\PYG{p}{)}\PYG{p}{,}\PYG{n}{Point}\PYG{p}{(}\PYG{o}{\PYGZhy{}}\PYG{n}{x}\PYG{o}{/}\PYG{l+m+mi}{2}\PYG{p}{,} \PYG{o}{\PYGZhy{}}\PYG{n}{x}\PYG{o}{/}\PYG{l+m+mi}{2}\PYG{p}{)}\PYG{p}{,}\PYG{n}{Point}\PYG{p}{(}\PYG{n}{x}\PYG{o}{/}\PYG{l+m+mi}{2}\PYG{p}{,} \PYG{o}{\PYGZhy{}}\PYG{n}{x}\PYG{o}{/}\PYG{l+m+mi}{2}\PYG{p}{)}\PYG{p}{)}
\PYG{n}{sim\PYGZus{}quad}
\end{sphinxVerbatim}
\begin{equation*}
\begin{split}\displaystyle Polygon\left(Point2D\left(\frac{x}{2}, \frac{x}{2}\right), Point2D\left(- \frac{x}{2}, \frac{x}{2}\right), Point2D\left(- \frac{x}{2}, - \frac{x}{2}\right), Point2D\left(\frac{x}{2}, - \frac{x}{2}\right)\right)\end{split}
\end{equation*}
\begin{sphinxVerbatim}[commandchars=\\\{\}]
\PYG{n}{sim\PYGZus{}quad}\PYG{o}{.}\PYG{n}{area}
\end{sphinxVerbatim}
\begin{equation*}
\begin{split}\displaystyle x^{2}\end{split}
\end{equation*}

\subsection{3D}
\label{\detokenize{chapters/6:id1}}
\sphinxAtStartPar
Para a terceira dimensão, podemos utilizar os Pontos com três coordenadas para gerar nossas formas.

\begin{sphinxVerbatim}[commandchars=\\\{\}]
\PYG{n}{M} \PYG{o}{=} \PYG{n}{Point}\PYG{p}{(}\PYG{l+m+mi}{1}\PYG{p}{,} \PYG{l+m+mi}{2}\PYG{p}{,} \PYG{l+m+mi}{3}\PYG{p}{)}
\PYG{n}{N} \PYG{o}{=} \PYG{n}{Point}\PYG{p}{(}\PYG{o}{\PYGZhy{}}\PYG{l+m+mi}{2}\PYG{p}{,} \PYG{l+m+mi}{3}\PYG{p}{,} \PYG{l+m+mi}{4}\PYG{p}{)}
\PYG{n}{P} \PYG{o}{=} \PYG{n}{Point}\PYG{p}{(}\PYG{l+m+mi}{5}\PYG{p}{,} \PYG{o}{\PYGZhy{}}\PYG{l+m+mi}{8}\PYG{p}{,} \PYG{l+m+mi}{10}\PYG{p}{)}
\PYG{n}{Line}\PYG{p}{(}\PYG{n}{M}\PYG{p}{,}\PYG{n}{N}\PYG{p}{)}
\end{sphinxVerbatim}
\begin{equation*}
\begin{split}\displaystyle Line3D\left(Point3D\left(1, 2, 3\right), Point3D\left(-2, 3, 4\right)\right)\end{split}
\end{equation*}
\begin{sphinxVerbatim}[commandchars=\\\{\}]
\PYG{n}{Line}\PYG{p}{(}\PYG{n}{M}\PYG{p}{,}\PYG{n}{N}\PYG{p}{)}\PYG{o}{.}\PYG{n}{equation}\PYG{p}{(}\PYG{p}{)}
\end{sphinxVerbatim}
\begin{equation*}
\begin{split}\displaystyle \left( x + 3 y - 7, \  x + 3 z - 10\right)\end{split}
\end{equation*}
\begin{sphinxVerbatim}[commandchars=\\\{\}]
\PYG{n}{Plane}\PYG{p}{(}\PYG{n}{M}\PYG{p}{,}\PYG{n}{N}\PYG{p}{,}\PYG{n}{P}\PYG{p}{)} \PYG{c+c1}{\PYGZsh{} Plano}
\end{sphinxVerbatim}
\begin{equation*}
\begin{split}\displaystyle Plane\left(Point3D\left(1, 2, 3\right), \left( 17, \  25, \  26\right)\right)\end{split}
\end{equation*}
\begin{sphinxVerbatim}[commandchars=\\\{\}]
\PYG{n}{Plane}\PYG{p}{(}\PYG{n}{M}\PYG{p}{,}\PYG{n}{N}\PYG{p}{,}\PYG{n}{P}\PYG{p}{)}\PYG{o}{.}\PYG{n}{equation}\PYG{p}{(}\PYG{p}{)}
\end{sphinxVerbatim}
\begin{equation*}
\begin{split}\displaystyle 17 x + 25 y + 26 z - 145\end{split}
\end{equation*}

\section{Reações em Vigas (Mecânica)}
\label{\detokenize{chapters/6:reacoes-em-vigas-mecanica}}
\sphinxAtStartPar
Nós podemos analisar as tensões em vigas utilizando o \sphinxcode{\sphinxupquote{sympy.physics.continuum\_mechanics}}. Caso você procure aplicar carregamento em uma viga e então avaliar as reações e seus gráficos, certamente isso vai te auxiliar.

\sphinxAtStartPar
Veja um exemplo (no caso, para fazer sentido, leva os mesmos valores de um exemplo da documentação):

\begin{sphinxVerbatim}[commandchars=\\\{\}]
\PYG{k+kn}{from} \PYG{n+nn}{sympy}\PYG{n+nn}{.}\PYG{n+nn}{physics}\PYG{n+nn}{.}\PYG{n+nn}{continuum\PYGZus{}mechanics}\PYG{n+nn}{.}\PYG{n+nn}{beam} \PYG{k+kn}{import} \PYG{n}{Beam}
\PYG{n}{E}\PYG{p}{,} \PYG{n}{I} \PYG{o}{=} \PYG{n}{symbols}\PYG{p}{(}\PYG{l+s+s1}{\PYGZsq{}}\PYG{l+s+s1}{E I}\PYG{l+s+s1}{\PYGZsq{}}\PYG{p}{)} \PYG{c+c1}{\PYGZsh{}\PYGZsh{} Símbolos para o Módulo de Elasticidade e o Momento de Inércia}
\PYG{n}{R1}\PYG{p}{,} \PYG{n}{R2} \PYG{o}{=} \PYG{n}{symbols}\PYG{p}{(}\PYG{l+s+s1}{\PYGZsq{}}\PYG{l+s+s1}{R1 R2}\PYG{l+s+s1}{\PYGZsq{}}\PYG{p}{)} \PYG{c+c1}{\PYGZsh{}\PYGZsh{} Símbolos para as forças}
\PYG{n}{b} \PYG{o}{=} \PYG{n}{Beam}\PYG{p}{(}\PYG{l+m+mi}{50}\PYG{p}{,} \PYG{l+m+mi}{20}\PYG{p}{,} \PYG{l+m+mi}{30}\PYG{p}{)} \PYG{c+c1}{\PYGZsh{}\PYGZsh{} Criando a viga (comprimento, E, I)}
\PYG{n}{b}\PYG{o}{.}\PYG{n}{apply\PYGZus{}load}\PYG{p}{(}\PYG{n}{R1}\PYG{p}{,} \PYG{l+m+mi}{0}\PYG{p}{,} \PYG{o}{\PYGZhy{}}\PYG{l+m+mi}{1}\PYG{p}{)} \PYG{c+c1}{\PYGZsh{}\PYGZsh{} Aplicando carregamentos (intensidade, início, ordem)}
\PYG{c+c1}{\PYGZsh{}\PYGZsh{}\PYGZsh{}}
\PYG{c+c1}{\PYGZsh{}\PYGZsh{}\PYGZsh{} Momentos, order = \PYGZhy{}2}
\PYG{c+c1}{\PYGZsh{}\PYGZsh{}\PYGZsh{} Forças Pontuais, order =\PYGZhy{}1}
\PYG{c+c1}{\PYGZsh{}\PYGZsh{}\PYGZsh{} Forças distribuídas linearmente, order = 0}
\PYG{c+c1}{\PYGZsh{}\PYGZsh{}\PYGZsh{} Veja os outros na documentação}
\PYG{c+c1}{\PYGZsh{}\PYGZsh{}\PYGZsh{}}
\PYG{n}{b}\PYG{o}{.}\PYG{n}{apply\PYGZus{}load}\PYG{p}{(}\PYG{n}{R1}\PYG{p}{,} \PYG{l+m+mi}{10}\PYG{p}{,} \PYG{o}{\PYGZhy{}}\PYG{l+m+mi}{1}\PYG{p}{)}
\PYG{n}{b}\PYG{o}{.}\PYG{n}{apply\PYGZus{}load}\PYG{p}{(}\PYG{n}{R2}\PYG{p}{,} \PYG{l+m+mi}{30}\PYG{p}{,} \PYG{o}{\PYGZhy{}}\PYG{l+m+mi}{1}\PYG{p}{)}
\PYG{n}{b}\PYG{o}{.}\PYG{n}{apply\PYGZus{}load}\PYG{p}{(}\PYG{l+m+mi}{90}\PYG{p}{,} \PYG{l+m+mi}{5}\PYG{p}{,} \PYG{l+m+mi}{0}\PYG{p}{,} \PYG{l+m+mi}{23}\PYG{p}{)}
\PYG{n}{b}\PYG{o}{.}\PYG{n}{apply\PYGZus{}load}\PYG{p}{(}\PYG{l+m+mi}{10}\PYG{p}{,} \PYG{l+m+mi}{30}\PYG{p}{,} \PYG{l+m+mi}{1}\PYG{p}{,} \PYG{l+m+mi}{50}\PYG{p}{)}
\PYG{n}{b}\PYG{o}{.}\PYG{n}{load} \PYG{c+c1}{\PYGZsh{}\PYGZsh{} Carregamento}
\end{sphinxVerbatim}
\begin{equation*}
\begin{split}\displaystyle R_{1} {\left\langle x \right\rangle}^{-1} + R_{1} {\left\langle x - 10 \right\rangle}^{-1} + R_{2} {\left\langle x - 30 \right\rangle}^{-1} + 90 {\left\langle x - 5 \right\rangle}^{0} - 90 {\left\langle x - 23 \right\rangle}^{0} + 10 {\left\langle x - 30 \right\rangle}^{1} - 200 {\left\langle x - 50 \right\rangle}^{0} - 10 {\left\langle x - 50 \right\rangle}^{1}\end{split}
\end{equation*}
\begin{sphinxVerbatim}[commandchars=\\\{\}]
\PYG{n}{b}\PYG{o}{.}\PYG{n}{shear\PYGZus{}force}\PYG{p}{(}\PYG{p}{)} \PYG{c+c1}{\PYGZsh{}\PYGZsh{} Força Cortante}
\end{sphinxVerbatim}
\begin{equation*}
\begin{split}\displaystyle - M_{0} {\left\langle x \right\rangle}^{-1} - R_{0} {\left\langle x \right\rangle}^{0} - R_{20} {\left\langle x - 20 \right\rangle}^{0} - R_{50} {\left\langle x - 50 \right\rangle}^{0} - \frac{224 {\left\langle x \right\rangle}^{0}}{15} - 90 {\left\langle x - 5 \right\rangle}^{1} - \frac{224 {\left\langle x - 10 \right\rangle}^{0}}{15} + 90 {\left\langle x - 23 \right\rangle}^{1} + \frac{54748 {\left\langle x - 30 \right\rangle}^{0}}{15} - 5 {\left\langle x - 30 \right\rangle}^{2} + 200 {\left\langle x - 50 \right\rangle}^{1} + 5 {\left\langle x - 50 \right\rangle}^{2}\end{split}
\end{equation*}
\begin{sphinxVerbatim}[commandchars=\\\{\}]
\PYG{n}{b}\PYG{o}{.}\PYG{n}{bending\PYGZus{}moment}\PYG{p}{(}\PYG{p}{)} \PYG{c+c1}{\PYGZsh{}\PYGZsh{} Momento Fletor}
\end{sphinxVerbatim}
\begin{equation*}
\begin{split}\displaystyle - M_{0} {\left\langle x \right\rangle}^{0} - R_{0} {\left\langle x \right\rangle}^{1} - R_{20} {\left\langle x - 20 \right\rangle}^{1} - R_{50} {\left\langle x - 50 \right\rangle}^{1} - \frac{224 {\left\langle x \right\rangle}^{1}}{15} - 45 {\left\langle x - 5 \right\rangle}^{2} - \frac{224 {\left\langle x - 10 \right\rangle}^{1}}{15} + 45 {\left\langle x - 23 \right\rangle}^{2} + \frac{54748 {\left\langle x - 30 \right\rangle}^{1}}{15} - \frac{5 {\left\langle x - 30 \right\rangle}^{3}}{3} + 100 {\left\langle x - 50 \right\rangle}^{2} + \frac{5 {\left\langle x - 50 \right\rangle}^{3}}{3}\end{split}
\end{equation*}
\begin{sphinxVerbatim}[commandchars=\\\{\}]
\PYG{n}{p} \PYG{o}{=} \PYG{n}{b}\PYG{o}{.}\PYG{n}{draw}\PYG{p}{(}\PYG{p}{)}
\PYG{n}{p}\PYG{o}{.}\PYG{n}{show}\PYG{p}{(}\PYG{p}{)} \PYG{c+c1}{\PYGZsh{}\PYGZsh{} Ilustração Gráfica}
\end{sphinxVerbatim}

\noindent\sphinxincludegraphics{{6_73_0}.png}

\begin{sphinxVerbatim}[commandchars=\\\{\}]
\PYG{n}{b}\PYG{o}{.}\PYG{n}{solve\PYGZus{}for\PYGZus{}reaction\PYGZus{}loads}\PYG{p}{(}\PYG{n}{R1}\PYG{p}{,} \PYG{n}{R2}\PYG{p}{)} \PYG{c+c1}{\PYGZsh{}\PYGZsh{} Solucionando}
\PYG{n}{b}\PYG{o}{.}\PYG{n}{plot\PYGZus{}bending\PYGZus{}moment}\PYG{p}{(}\PYG{p}{)} \PYG{c+c1}{\PYGZsh{}\PYGZsh{} Plot Momento Fletor}
\end{sphinxVerbatim}

\noindent\sphinxincludegraphics{{6_74_0}.png}

\begin{sphinxVerbatim}[commandchars=\\\{\}]
\PYGZlt{}sympy.plotting.plot.Plot at 0x7f9576b37cd0\PYGZgt{}
\end{sphinxVerbatim}

\begin{sphinxVerbatim}[commandchars=\\\{\}]
\PYG{n}{b}\PYG{o}{.}\PYG{n}{plot\PYGZus{}shear\PYGZus{}force}\PYG{p}{(}\PYG{p}{)} \PYG{c+c1}{\PYGZsh{}\PYGZsh{} Plotando Força Cortante}
\end{sphinxVerbatim}

\noindent\sphinxincludegraphics{{6_75_0}.png}

\begin{sphinxVerbatim}[commandchars=\\\{\}]
\PYGZlt{}sympy.plotting.plot.Plot at 0x7f9576c90d50\PYGZgt{}
\end{sphinxVerbatim}

\begin{sphinxVerbatim}[commandchars=\\\{\}]
\PYG{n}{b}\PYG{o}{.}\PYG{n}{apply\PYGZus{}support}\PYG{p}{(}\PYG{l+m+mi}{50}\PYG{p}{,} \PYG{l+s+s1}{\PYGZsq{}}\PYG{l+s+s1}{pin}\PYG{l+s+s1}{\PYGZsq{}}\PYG{p}{)} \PYG{c+c1}{\PYGZsh{}\PYGZsh{} Criando apoios}
\PYG{n}{b}\PYG{o}{.}\PYG{n}{apply\PYGZus{}support}\PYG{p}{(}\PYG{l+m+mi}{0}\PYG{p}{,} \PYG{l+s+s1}{\PYGZsq{}}\PYG{l+s+s1}{fixed}\PYG{l+s+s1}{\PYGZsq{}}\PYG{p}{)}
\PYG{n}{b}\PYG{o}{.}\PYG{n}{apply\PYGZus{}support}\PYG{p}{(}\PYG{l+m+mi}{20}\PYG{p}{,} \PYG{l+s+s1}{\PYGZsq{}}\PYG{l+s+s1}{roller}\PYG{l+s+s1}{\PYGZsq{}}\PYG{p}{)}
\PYG{n}{b}\PYG{o}{.}\PYG{n}{load}
\end{sphinxVerbatim}
\begin{equation*}
\begin{split}\displaystyle M_{0} {\left\langle x \right\rangle}^{-2} + R_{0} {\left\langle x \right\rangle}^{-1} + R_{20} {\left\langle x - 20 \right\rangle}^{-1} + R_{50} {\left\langle x - 50 \right\rangle}^{-1} + \frac{224 {\left\langle x \right\rangle}^{-1}}{15} + 90 {\left\langle x - 5 \right\rangle}^{0} + \frac{224 {\left\langle x - 10 \right\rangle}^{-1}}{15} - 90 {\left\langle x - 23 \right\rangle}^{0} - \frac{54748 {\left\langle x - 30 \right\rangle}^{-1}}{15} + 10 {\left\langle x - 30 \right\rangle}^{1} - 200 {\left\langle x - 50 \right\rangle}^{0} - 10 {\left\langle x - 50 \right\rangle}^{1}\end{split}
\end{equation*}
\begin{sphinxVerbatim}[commandchars=\\\{\}]
\PYG{n}{p} \PYG{o}{=} \PYG{n}{b}\PYG{o}{.}\PYG{n}{draw}\PYG{p}{(}\PYG{p}{)}
\PYG{n}{p}\PYG{o}{.}\PYG{n}{show}\PYG{p}{(}\PYG{p}{)}
\end{sphinxVerbatim}

\noindent\sphinxincludegraphics{{6_77_0}.png}







\renewcommand{\indexname}{Index}
\printindex
\end{document}